


\noindent This chapter presents an audit of Everscalend's smart contracts and lists the issues that were encoured in the source code.



\listoffigures





\section{General remarks}

In this section we present some recurrent issues that were encountered in the source code and some general good practices that should be respected. 

%\subsection{Typography of Static Variables}
%\label{readability:static}

%A good coding convention is to use typography to visually discriminate static variables from other variables, for example using a prefix such as {\tt s\_}.

%\subsection{Typography of Global Variables}
%\label{readability:global}

%A good coding convention is to use typography to visually discriminate global variables from local variables, for example using a prefix such as {\tt m\_} or {\tt g\_}.

\subsection{Typography of Internal Functions}
\label{readability:internal}

A good coding convention is to use typography to visually discriminate public functions and internal functions, for example using a prefix such as {\tt \_}.

%\subsection{Accept Methods without Checks}
%\label{accept:all}

%Public methods using {\tt tvm.accept()} without any prior check should not exist. Indeed, such methods could be used by attackers to drain the balance of the contracts, even with minor amounts but unlimited number of messages.

\subsection{Constructors without checks}
\label{constructor:check}

Contract constructors should always at the very least verify that the contract's public key is set and that the deployer is the owner of the contract. This is important especially in the case in which the contract has arguments that set the state variables. If it is not done, it opens the gate to various kinds of attacks.


\newcommand{\undefinedFunction}[1]{\issueMinor{Undefined function: {\tt #1}}{Undefined and unused function.}}

\newcommand{\unusedFunction}[1]{\issueMinor{Unused function: {\tt #1}}{Unused function.}}

\newcommand{\unusedModifier}[1]{\issueMinor{Unused modifier: {\tt #1}}{Unused modifier.}}

%\newcommand{\libraryFunctionMutability}[1]{\issueMinor{Warning in {\tt #1}}{Library functions must have default mutability.}} %too many

\newcommand{\internalFunctionName}[1]{\issueMinor{Readability issue in \{\tt #1}{See \ref{readability:internal}. Internal function names should start with {\tt \_}.}}

\section{Contract deployment from Platform}

\issueCritical{Unprotected constructors in many contracts}{
  See \ref{constructor:check}. 
  
  Other than the `RootTokenContract` and `TONTokenWallet` contracts, all the other contracts have unprotected constructors and a comment that says that the contract will be deployed from the Platform. That does mean that is no longer necessary to check that the deployer of the contract is the owner of the contract. It is especially dangerous in the contracts which set the owner through the constructor like: `MarketAggregator`, `BorrowModule`, `LiquidationModule`, `RepayModule`, `SupplyModule`, `WithdrawModule`, `Oracle`, `TIP3TokenDeployer`, `UserAccount`, `UserAccountManager`, `Platform` and `WalletController`.
}

\subsection{Possible attack}

It makes it possible to perform phishing attacks by deploying fake contracts with which the users can interact with. So instead of interacting with the real contract they interact with the fake. 

If a malicious user deploys a fake `UserAccountManager` which will deploy user's accounts. And one of the users requests a withdraw of their tokens. The owner of the fake `UserAccountManager` can either block the transaction stopping the user's from withdrawing their tokens, or ask them for a fee before processing their request.

%In a less likely way, if another user has the address of the owner, they can deploy the contract as if they were the owner before he does and impersonate him, making them able to control a fake system to their own interest. 

%\section{Address setting and updating}

%\issueCritical{No checks before setting}{}

%\subsection{Possible attack}

\section{Internal function names ({\tt TODO} regroup them here)}

\section{Undefined functions ({\tt TODO} regroup them here)}

\section{Unused functions ({\tt TODO} regroup them here)}

\section{Unused modifiers ({\tt TODO} regroup them here)}



\section{Module "Giver.sol"}


\subsection{Pragmas}


\noindent\begin{tabular}{|l|l|p{5cm}|}\hline
ton & -solidity $>$= 0.39.0 &\\\hline
AbiHeader &  time &\\\hline
AbiHeader &  expire &\\\hline
\end{tabular}


\subsection{Contract Definitions}

\begin{itemize}
\item Giver
\end{itemize}
 %ignored for now

\section{Library MarketMath}

In file {\tt MarketMath.sol}


\subsection{Function calculateUtilizationRate}

\begin{lstlisting}[firstnumber=4]
    function calculateUtilizationRate(uint256 currentPool, uint256 totalBorrowed) internal pure returns (uint256) {

    }
\end{lstlisting}

\noindent\begin{itemize}
  \item \undefinedFunction{MarketMath.calculateUtilizationRate}
\end{itemize}


\subsection{Function calculateBorrowingRate}

\begin{lstlisting}[firstnumber=8]
    function calculateBorrowingRate(uint256 currentPool, uint256 totalBorrowed, uint256 totalReserves, uint256 totalSupply) 
        internal pure returns (uint256) 
    {

    }
\end{lstlisting}

\noindent\begin{itemize}
  \item \undefinedFunction{MarketMath.calculateBorrowingRate}
\end{itemize}


\subsection{Function calculateExchangeRate}

\begin{lstlisting}[firstnumber=14]
    function calculateExchangeRate(uint256 currentPool, uint256 totalBorrowed, uint256 totalReserves, uint256 totalSupply)
        internal pure returns (uint256) 
    {
        return math.div(currentPool - totalReserves + totalBorrowed, totalSupply);
    }
\end{lstlisting}

\noindent\begin{itemize}
  \item \undefinedFunction{MarketMath.calculateExchangeRate}
  \item \issueMinor{Syntax Error in {\tt MarketMath.calculateExchangeRate}}{{\tt math.div} does not exist. It should use the infix division operator {\tt /}.}
\end{itemize}


\subsection{Function recalculateState}

\begin{lstlisting}[firstnumber=20]
    function recalculateState(uint256 currentPool, uint256 totalBorrowed, uint256 totalReserves, uint256 totalSupply)
        internal pure 
    {
        // uint256 exchangeRate = calculateExchangeRate(currentPool, totalBorrowed, totalReserves, totalSupply);
    }
\end{lstlisting}

\noindent\begin{itemize}
  \item \undefinedFunction{MarketMath.recalculateState}
\end{itemize}






\section{Library MarketOperations}

In file {\tt MarketOperations.sol}


\subsection{Function calculateU}

\begin{lstlisting}[firstnumber=9]
    function calculateU(uint256 totalBorrowed, uint256 realTokens) internal pure returns (fraction) {
        return fraction(totalBorrowed, totalBorrowed + realTokens);
    }
\end{lstlisting}

\noindent\begin{itemize}
  \item \unusedFunction{MarketOperations.calculateU}
\end{itemize}


\subsection{Function calculateTotalReserves}

\begin{lstlisting}[firstnumber=28]
    function calculateTotalReserves(uint256 totalReserve, uint256 totalBorrowed, fraction r, fraction reserveFactor, uint256 t) internal returns (fraction) {
        fraction tr;
        tr = r.fNumMul(t);
        tr = tr.fMul(reserveFactor);
        tr = tr.fNumMul(totalBorrowed);
        tr = tr.fNumAdd(totalReserve);
        return tr;
    }
\end{lstlisting}

\noindent\begin{itemize}
  \item \unusedFunction{MarketOperations.calculateTotalReserves}
\end{itemize}


\subsection{Function calculateNewIndex}

\begin{lstlisting}[firstnumber=37]
    function calculateNewIndex(fraction index, fraction bir, uint256 dt) internal returns (fraction) {
        fraction index_;
        index_ = bir.fNumMul(dt);
        index_ = index_.fNumAdd(1);
        index_ = index_.fAdd(index);
        return index_;
    }
\end{lstlisting}

\noindent\begin{itemize}
  \item \unusedFunction{MarketOperations.calculateNewIndex}
\end{itemize}


\subsection{Function calculateTotalBorrowed}

\begin{lstlisting}[firstnumber=45]
    function calculateTotalBorrowed(uint256 totalBorrowed, fraction oldIndex, fraction newIndex) internal returns (uint256) {
        fraction tb_;
        tb_ = totalBorrowed.numFDiv(oldIndex);
        tb_ = tb_.fMul(newIndex);
        return tb_.toNum();
    }
\end{lstlisting}

\noindent\begin{itemize}
  \item \unusedFunction{MarketOperations.calculateTotalBorrowed}
\end{itemize}


\subsection{Function calculateReserves}

\begin{lstlisting}[firstnumber=52]
    function calculateReserves(uint256 reserveOld, uint256 totalBorrowedOld, fraction bir, fraction reserveFactor, uint256 dt) internal returns (uint256) {
        fraction res = bir;
        res = res.fNumMul(dt);
        res = res.fMul(reserveFactor);
        res = res.fNumMul(totalBorrowedOld);
        res = res.fNumAdd(reserveOld);
        return res.toNum();
    }
\end{lstlisting}

\noindent\begin{itemize}
  \item \unusedFunction{MarketOperations.calculateReserves}
\end{itemize}


\section{Library MarketToUserPayloads}

In file {\tt MarketPayloads.sol}

\issueMinor{Unused functions}{All the functions in the file are unused.}



\section{Contract MarketAggregator}

In file {\tt MarketsAggregator.sol}

\subsection{Function performOperation}

\begin{lstlisting}[firstnumber=361]
    function performOperation(TvmCell args) internal view {
        TvmSlice ts = args.toSlice();

        uint8 operationId = ts.decode(uint8);
        if (operationId != OperationCodes.NO_OP) {
            uint32 marketId = ts.decode(uint32);
            TvmCell moduleArgs = ts.loadRef();
            IModule(modules[operationId]).performAction{
                flag: MsgFlag.REMAINING_GAS
            }(marketId, moduleArgs, markets, tokenPrices);
        } else {
            address(_owner).transfer({value: 0, flag: MsgFlag.REMAINING_GAS});
        }
    }
\end{lstlisting}

\noindent\begin{itemize}
  \item \internalFunctionName{MarketAggregator.performOperation}
\end{itemize}


\subsection{Function updatePrice}

\begin{lstlisting}[firstnumber=428]
    function updatePrice(address tokenRoot, TvmCell payload) internal view {
        IOracleReturnPrices(oracle).getTokenPrice{
            flag: MsgFlag.REMAINING_GAS,
            callback: this.receiveUpdatedPrice
        }(tokenRoot, payload);
    }
\end{lstlisting}

\noindent\begin{itemize}
  \item \internalFunctionName{MarketAggregator.updatePrice}
\end{itemize}
  
  
\subsection{Function \_{}calculateBorrowInfo}

\begin{lstlisting}[firstnumber=516]
    function _calculateBorrowInfo(mapping(uint32 => BorrowInfo) borrowInfo, mapping(uint32 => fraction) updatedIndexes) internal returns(mapping (uint32=>uint256) userBorrowInfo) {
        for ((uint32 marketId, BorrowInfo bi): borrowInfo) {
            if (bi.tokensBorrowed != 0) {
                fraction tmpf = borrowInfo[marketId].tokensBorrowed.numFMul(updatedIndexes[marketId]);
                tmpf = tmpf.fDiv(bi.index);
                userBorrowInfo[marketId] = tmpf.toNum();
            } else {
                userBorrowInfo[marketId] = 0;
            }
        }
    }
\end{lstlisting}

\noindent\begin{itemize}
  \item \unusedFunction{MarketAggregator.\_calculateBorrowInfo}
\end{itemize}


\subsection{Modifier onlySelf}

\begin{lstlisting}[firstnumber=510]
    modifier onlySelf() {
        require(msg.sender == address(this), MarketErrorCodes.ERROR_MSG_SENDER_IS_NOT_SELF);
        _;
    }
\end{lstlisting}

\noindent\begin{itemize}
  \item \unusedModifier{MarketAggregator.onlySelf}
\end{itemize}


\subsection{Modifier onlyRealTokenRoot}

\begin{lstlisting}[firstnumber=533]
    modifier onlyRealTokenRoot() {
        require(realTokenRoots.exists(msg.sender));
        _;
    }
\end{lstlisting}

\noindent\begin{itemize}
  \item \unusedModifier{MarketAggregator.onlyRealTokenRoot}
\end{itemize}


\subsection{Modifier onlyExecutor}

\begin{lstlisting}[firstnumber=543]
    modifier onlyExecutor() {
        require(
            (msg.sender == userAccountManager) ||
            (isModule.exists(msg.sender))
        );
        _;
    }
\end{lstlisting}

\noindent\begin{itemize}
  \item \unusedModifier{MarketAggregator.onlyExecutor}
\end{itemize}





\section{Library Utilities}

In file {\tt IModule.sol}

\subsection{Function calculateSupplyBorrow}

\noindent\begin{itemize}
\item \issueMinor{Naming}{The function is called ``calculateSupplyBorrow'' but it calculates a user's account health. It should be named accordingly, e.g. ``calculateAccountHealth''.}
\end{itemize}

\begin{lstlisting}[firstnumber=90]
    function calculateSupplyBorrow(
        mapping(uint32 => uint256) supplyInfo,
        mapping(uint32 => BorrowInfo) borrowInfo,
        mapping(uint32 => MarketInfo) marketInfo,
        mapping(address => fraction) tokenPrices
    ) internal returns (fraction) {
        fraction accountHealth = fraction(0, 0);
        fraction tmp;
        fraction nom = fraction(0, 1);
        fraction denom = fraction(0, 1);

        // Supply:
        // 1. Calculate real token amount: vToken*exchangeRate
        // 2. Calculate real token amount in USD: realTokens/tokenPrice
        // 3. Multiply by collateral factor: usdValue*collateralFactor
        for ((uint32 marketId, uint256 supplied): supplyInfo) {
            tmp = supplied.numFMul(marketInfo[marketId].exchangeRate);
            tmp = tmp.fDiv(tokenPrices[marketInfo[marketId].token]);
            tmp = tmp.fMul(marketInfo[marketId].collateralFactor);
            nom = nom.fAdd(tmp);
            nom = nom.simplify();
        }

        // Borrow:
        // 1. Recalculate borrow amount according to new index
        // 2. Calculate borrow value in USD
        // NOTE: no conversion from vToken to real tokens required, as value is stored in real tokens
        for ((uint32 marketId, BorrowInfo _bi): borrowInfo) {
            if (_bi.tokensBorrowed != 0) {
                if (!_bi.index.eq(marketInfo[marketId].index)) {
                    tmp = borrowInfo[marketId].tokensBorrowed.numFMul(marketInfo[marketId].index);
                    tmp = tmp.fDiv(borrowInfo[marketId].index);
                } else {
                    tmp = borrowInfo[marketId].tokensBorrowed.toF();
                }
                tmp = tmp.fDiv(tokenPrices[marketInfo[marketId].token]);
                tmp = tmp.simplify();
                denom = denom.fAdd(tmp);
                denom = denom.simplify();
            }
        }

        accountHealth = nom.fDiv(denom);

        return accountHealth;
    }
\end{lstlisting}


\bigskip

\section{Contract BorrowModule}

In file {\tt BorrowModule.sol}

\subsection{Function borrowTokensFromMarket}

\begin{lstlisting}[firstnumber=74]
    function borrowTokensFromMarket(
        address tonWallet,
        address userTip3Wallet,
        uint256 tokensToBorrow,
        uint32 marketId,
        mapping (uint32 => uint256) supplyInfo,
        mapping (uint32 => BorrowInfo) borrowInfo
    ) external override onlyUserAccountManager {
        tvm.rawReserve(msg.value, 2);
        mapping(uint32 => MarketDelta) marketsDelta;
        MarketDelta marketDelta;
        
        // Borrow:
        // 1. Check that market has enough tokens for lending
        // 2. Calculate user account health
        // 3. Calculate USD value of tokens to borrow
        // 4. Check if there is enough (collateral - borrowed) for new token borrow
        // 5. Increase user's borrowed amount

        MarketInfo mi = marketInfo[marketId];

        if (tokensToBorrow < mi.realTokenBalance - mi.totalReserve) {
            fraction accountHealth = Utilities.calculateSupplyBorrow(supplyInfo, borrowInfo, marketInfo, tokenPrices);
            if (accountHealth.nom > accountHealth.denom) {
                uint256 healthDelta = accountHealth.nom - accountHealth.denom;
                fraction tmp = healthDelta.numFMul(tokenPrices[marketInfo[marketId].token]);
                uint256 possibleTokenWithdraw = tmp.toNum();
                if (possibleTokenWithdraw >= tokensToBorrow) {
                    marketDelta.totalBorrowed.delta = tokensToBorrow;
                    marketDelta.totalBorrowed.positive = true;
                    marketDelta.realTokenBalance.delta = tokensToBorrow;
                    marketDelta.realTokenBalance.positive = false;

                    marketsDelta[marketId] = marketDelta;

                    TvmBuilder tb;
                    tb.store(marketId);
                    tb.store(tonWallet);
                    tb.store(userTip3Wallet);
                    tb.store(tokensToBorrow);

                    emit TokenBorrow(marketId, marketDelta, tonWallet, tokensToBorrow);

                    IContractStateCacheRoot(marketAddress).receiveCacheDelta{
                        flag: MsgFlag.REMAINING_GAS
                    }(marketsDelta, tb.toCell());
                } else {
                    IUAMUserAccount(userAccountManager).writeBorrowInformation{
                        flag: MsgFlag.REMAINING_GAS
                    }(tonWallet, userTip3Wallet, 0, marketId, marketInfo[marketId].index);
                }
            } else {
                IUAMUserAccount(userAccountManager).writeBorrowInformation{
                    flag: MsgFlag.REMAINING_GAS
                }(tonWallet, userTip3Wallet, 0, marketId, marketInfo[marketId].index);
            }
        } else {
            address(tonWallet).transfer({value: 0, flag: MsgFlag.REMAINING_GAS});
        }
    }
\end{lstlisting}

\issueCritical{Math error in {\tt BorrowModule.borrowTokensFromMarket}}{Line 99. To caculate the amount of tokens that it is possible to withdraw, the health delta needs to be divided by the price of the token not multiplied by it.}



\chapter{Contract LiquidationModule}

\minitoc

\section{Overview}


In file {\tt LiquidationModule.sol}

\section{Contract Inheritance}


\noindent\begin{tabular}{|l|p{5cm}|}\hline
IRoles & \\\hline
IModule & \\\hline
IContractStateCache & \\\hline
IContractAddressSG & \\\hline
ILiquidationModule & \\\hline
IUpgradableContract & \\\hline
\end{tabular}


\section{Event Definitions}


\begin{lstlisting}[firstnumber=16]
    event TokensLiquidated(uint32 marketId, mapping(uint32 => MarketDelta) marketDeltas, address liquidator, address targetUser, uint256 tokensLiquidated, uint256 vTokensSeized);
\end{lstlisting}

\section{Variable Definitions}


\ifsoltables
\noindent\begin{tabular}{|l|l|p{5cm}|}\hline
address & marketAddress &  \\\hline
 & & used in @7.LiquidationModule.upgradeContractCode\\\hline
 & & assigned in @7.LiquidationModule.setMarketAddress\\\hline
 & & used in @7.LiquidationModule.setMarketAddress\\\hline
 & & assigned in @7.LiquidationModule.onCodeUpgrade\\\hline
 & & used in @7.LiquidationModule.onCodeUpgrade\\\hline
 & & used in @7.LiquidationModule.liquidate\\\hline
 & & used in @7.LiquidationModule.getContractAddresses\\\hline
address & userAccountManager &  \\\hline
 & & used in @7.LiquidationModule.upgradeContractCode\\\hline
 & & assigned in @7.LiquidationModule.setUserAccountManager\\\hline
 & & used in @7.LiquidationModule.setUserAccountManager\\\hline
 & & used in @7.LiquidationModule.resumeOperation\\\hline
 & & used in @7.LiquidationModule.performAction\\\hline
 & & assigned in @7.LiquidationModule.onCodeUpgrade\\\hline
 & & used in @7.LiquidationModule.onCodeUpgrade\\\hline
 & & used in @7.LiquidationModule.liquidate\\\hline
 & & used in @7.LiquidationModule.liquidate\\\hline
 & & used in @7.LiquidationModule.getContractAddresses\\\hline
uint32 & contractCodeVersion &  \\\hline
 & & assigned in @7.LiquidationModule.onCodeUpgrade\\\hline
 & & used in @7.LiquidationModule.onCodeUpgrade\\\hline
mapping (uint32 =$>$ MarketInfo) & marketInfo &  \\\hline
 & & used in @7.LiquidationModule.upgradeContractCode\\\hline
 & & assigned in @7.LiquidationModule.updateCache\\\hline
 & & used in @7.LiquidationModule.updateCache\\\hline
 & & assigned in @7.LiquidationModule.resumeOperation\\\hline
 & & used in @7.LiquidationModule.resumeOperation\\\hline
 & & assigned in @7.LiquidationModule.performAction\\\hline
 & & used in @7.LiquidationModule.performAction\\\hline
 & & assigned in @7.LiquidationModule.onCodeUpgrade\\\hline
 & & used in @7.LiquidationModule.onCodeUpgrade\\\hline
 & & used in @7.LiquidationModule.liquidate\\\hline
 & & used in @7.LiquidationModule.liquidate\\\hline
 & & used in @7.LiquidationModule.liquidate\\\hline
 & & used in @7.LiquidationModule.liquidate\\\hline
 & & used in @7.LiquidationModule.liquidate\\\hline
 & & used in @7.LiquidationModule.liquidate\\\hline
 & & used in @7.LiquidationModule.liquidate\\\hline
 & & used in @7.LiquidationModule.liquidate\\\hline
 & & used in @7.LiquidationModule.liquidate\\\hline
 & & used in @7.LiquidationModule.liquidate\\\hline
 & & used in @7.LiquidationModule.getModuleState\\\hline
 & & used in @7.LiquidationModule.\_{}createUpdatedIndexes\\\hline
mapping (address =$>$ fraction) & tokenPrices &  \\\hline
 & & used in @7.LiquidationModule.upgradeContractCode\\\hline
 & & assigned in @7.LiquidationModule.updateCache\\\hline
 & & used in @7.LiquidationModule.updateCache\\\hline
 & & assigned in @7.LiquidationModule.resumeOperation\\\hline
 & & used in @7.LiquidationModule.resumeOperation\\\hline
 & & assigned in @7.LiquidationModule.performAction\\\hline
 & & used in @7.LiquidationModule.performAction\\\hline
 & & assigned in @7.LiquidationModule.onCodeUpgrade\\\hline
 & & used in @7.LiquidationModule.onCodeUpgrade\\\hline
 & & used in @7.LiquidationModule.liquidate\\\hline
 & & used in @7.LiquidationModule.liquidate\\\hline
 & & used in @7.LiquidationModule.liquidate\\\hline
 & & used in @7.LiquidationModule.liquidate\\\hline
 & & used in @7.LiquidationModule.liquidate\\\hline
 & & used in @7.LiquidationModule.getModuleState\\\hline
\end{tabular}
\fi


\begin{lstlisting}[firstnumber=9]
    address marketAddress;
\end{lstlisting}

\begin{lstlisting}[firstnumber=10]
    address userAccountManager;
\end{lstlisting}

\begin{lstlisting}[firstnumber=11]
    uint32 public contractCodeVersion;
\end{lstlisting}

\begin{lstlisting}[firstnumber=13]
    mapping (uint32 => MarketInfo) marketInfo;
\end{lstlisting}

\begin{lstlisting}[firstnumber=14]
    mapping (address => fraction) tokenPrices;
\end{lstlisting}

\section{Modifier Definitions}


\subsection{Modifier onlyUserAccountManager}


\begin{lstlisting}[firstnumber=235]
    modifier onlyUserAccountManager() {
        require(msg.sender == userAccountManager);
        _;
    }
\end{lstlisting}

\subsection{Modifier onlyMarket}


\begin{lstlisting}[firstnumber=240]
    modifier onlyMarket() {
        require(msg.sender == marketAddress);
        tvm.rawReserve(msg.value, 2);
        _;
    }
\end{lstlisting}

\section{Constructor Definitions}


\subsection{Constructor}

\issueCritical{Constructor for LiquidationModule (fake)}{loren ipsum  loren ipsum  loren ipsum loren ipsum loren ipsum loren ipsum loren ipsum loren ipsum loren ipsum loren ipsum loren ipsum loren ipsum loren ipsum loren ipsum loren ipsum loren ipsum loren ipsum loren ipsum

loren ipsum loren ipsum loren ipsum loren ipsum loren ipsum loren ipsum
loren ipsum loren ipsum loren ipsum }
\noindent\begin{itemize}
\item TODO
\end{itemize}

\begin{lstlisting}[firstnumber=18]
    constructor(address _newOwner) public {
        tvm.accept();
        _owner = _newOwner;
    }
\end{lstlisting}

\section{Public Method Definitions}


\subsection{Function getContractAddresses}

\noindent\begin{itemize}
\item TODO
\end{itemize}

\begin{lstlisting}[firstnumber=77]
    function getContractAddresses() external override view responsible returns(address _owner, address _marketAddress, address _userAccountManager) {
        return {flag: MsgFlag.REMAINING_GAS} (_owner, marketAddress, userAccountManager);
    }
\end{lstlisting}

\subsection{Function getModuleState}

\noindent\begin{itemize}
\item TODO
\end{itemize}

\begin{lstlisting}[firstnumber=61]
    function getModuleState() external override view returns(mapping(uint32 => MarketInfo), mapping(address => fraction)) {
        return(marketInfo, tokenPrices);
    }
\end{lstlisting}

\subsection{Function liquidate}

\noindent\begin{itemize}
\item TODO
\end{itemize}

\begin{lstlisting}[firstnumber=101]
    function liquidate(
        address tonWallet, 
        address targetUser, 
        address tip3UserWallet, 
        uint32 marketId, 
        uint32 marketToLiquidate,
        uint256 tokensProvided, 
        mapping(uint32 => uint256) supplyInfo, 
        mapping(uint32 => BorrowInfo) borrowInfo
    ) external override onlyUserAccountManager {
        tvm.rawReserve(msg.value, 2);
        // Liquidation:
        // 1. Calculate user account health to check if liquidation is required
        // 2. Calculate max values
        // 3. Choose minimal value of all max values
        // 4. Based on min value calculate rest of parameters, it is guaranteed that:
        // - User will not exceed tokens that he provided for liquidation (providingLimit)
        // - User will not exceed tokens that are available for liquidation (borrowLimit)
        // - User will not exceed vToken balance of user that is liquidated (vTokenLimit)

        fraction health = Utilities.calculateSupplyBorrow(supplyInfo, borrowInfo, marketInfo, tokenPrices);
        if (health.nom <= health.denom) {
            uint256 tokensToLiquidate = math.min(
                borrowInfo[marketId].tokensBorrowed,
                tokensProvided
            );

            // Calculating USD value of liquidation
            fraction ftokensToLiquidateUSD = tokensToLiquidate.numFMul(marketInfo[marketId].liquidationMultiplier);
            ftokensToLiquidateUSD = ftokensToLiquidateUSD.fDiv(tokenPrices[marketInfo[marketId].token]);

            // Calculating USD value of collateral
            fraction fvTokensCollateralUSD = supplyInfo[marketToLiquidate].numFMul(marketInfo[marketToLiquidate].exchangeRate);
            fvTokensCollateralUSD = fvTokensCollateralUSD.fDiv(tokenPrices[marketInfo[marketToLiquidate].token]);

            uint256 tokensToSeize;
            uint256 tokensToReturn;
            uint256 tokensFromReserve;

            // Calculating how much of collateral tokens to seize
            fraction fvTokensCollateral = fvTokensCollateralUSD.getMin(ftokensToLiquidateUSD);
            fraction ftokensToSeize = fvTokensCollateral.fMul(tokenPrices[marketInfo[marketToLiquidate].token]);
            ftokensToSeize = ftokensToSeize.fDiv(marketInfo[marketToLiquidate].exchangeRate);
            tokensToSeize = ftokensToSeize.toNum();

            tokensToReturn = tokensProvided - tokensToLiquidate;
            mapping(uint32 => MarketDelta) marketDeltas;
            MarketDelta collateralMarketDelta;
            MarketDelta liquidationMarketDelta;

            liquidationMarketDelta.totalBorrowed.delta = tokensToLiquidate;
            liquidationMarketDelta.totalBorrowed.positive = false;
            liquidationMarketDelta.realTokenBalance.delta = tokensToLiquidate;
            liquidationMarketDelta.realTokenBalance.positive = true;

            if (fvTokensCollateralUSD.lessThan(ftokensToLiquidateUSD)) {
                // Using reserves from market to compensate liquidity absence
                fraction freservesUsageUSD = ftokensToLiquidateUSD.fSub(fvTokensCollateralUSD);
                freservesUsageUSD = freservesUsageUSD.simplify();
                fraction freservesUsageTokens = freservesUsageUSD.fMul(tokenPrices[marketInfo[marketToLiquidate].token]);
                uint256 reservesUsageTokens = freservesUsageTokens.toNum();
                if (reservesUsageTokens < marketInfo[marketId].totalReserve) {
                    tokensFromReserve = reservesUsageTokens;
                    collateralMarketDelta.totalReserve.delta = tokensFromReserve;
                    collateralMarketDelta.totalReserve.positive = false;
                } else {
                    // abort liquidation
                    IUAMUserAccount(userAccountManager).requestTokenPayout{
                        flag: MsgFlag.REMAINING_GAS
                    }(
                        tonWallet, tip3UserWallet, marketId, tokensProvided
                    );
                    tvm.exit();
                }
            }

            marketDeltas[marketId] = liquidationMarketDelta;
            marketDeltas[marketToLiquidate] = collateralMarketDelta;

            emit TokensLiquidated(marketId, marketDeltas, tonWallet, targetUser, tokensToLiquidate, tokensToSeize);

            BorrowInfo userBorrowInfo = BorrowInfo(borrowInfo[marketId].tokensBorrowed - tokensToLiquidate, marketInfo[marketId].index);

            TvmBuilder tb;
            TvmBuilder addressStorage;
            addressStorage.store(tonWallet);
            addressStorage.store(targetUser);
            addressStorage.store(tip3UserWallet);
            TvmBuilder valueStorage;
            valueStorage.store(marketId);
            valueStorage.store(marketToLiquidate);
            valueStorage.store(tokensToSeize);
            valueStorage.store(tokensToReturn);
            valueStorage.store(tokensFromReserve);
            TvmBuilder borrowInfoStorage;
            borrowInfoStorage.store(userBorrowInfo);
            tb.store(addressStorage.toCell());
            tb.store(valueStorage.toCell());
            tb.store(borrowInfoStorage.toCell());

            IContractStateCacheRoot(marketAddress).receiveCacheDelta{
                flag: MsgFlag.REMAINING_GAS
            }(marketDeltas, tb.toCell());
        } else {
            IUAMUserAccount(userAccountManager).requestTokenPayout{
                flag: MsgFlag.REMAINING_GAS
            }(
                tonWallet, tip3UserWallet, marketId, tokensProvided
            );
        }
    }
\end{lstlisting}

\subsection{Function performAction}

\noindent\begin{itemize}
\item TODO
\end{itemize}

\begin{lstlisting}[firstnumber=87]
    function performAction(uint32 marketId, TvmCell args, mapping (uint32 => MarketInfo) _marketInfo, mapping (address => fraction) _tokenPrices) external override onlyMarket {
        tvm.rawReserve(msg.value, 2);
        marketInfo = _marketInfo;
        tokenPrices = _tokenPrices;
        TvmSlice ts = args.toSlice();
        (address tonWallet, address targetUser, address tip3UserWallet) = ts.decode(address, address, address);
        TvmSlice amountTS = ts.loadRefAsSlice();
        (uint32 marketToLiquidate, uint256 tokenAmount) = amountTS.decode(uint32, uint256);
        mapping(uint32 => fraction) updatedIndexes = _createUpdatedIndexes();
        IUAMUserAccount(userAccountManager).requestLiquidationInformation{
            flag: MsgFlag.REMAINING_GAS
        }(tonWallet, targetUser, tip3UserWallet, marketId, marketToLiquidate, tokenAmount, updatedIndexes);
    }
\end{lstlisting}

\subsection{Function resumeOperation}

\noindent\begin{itemize}
\item TODO
\end{itemize}

\begin{lstlisting}[firstnumber=213]
    function resumeOperation(TvmCell args, mapping(uint32 => MarketInfo) _marketInfo, mapping (address => fraction) _tokenPrices) external override onlyMarket {
        tvm.rawReserve(msg.value, 2);
        marketInfo = _marketInfo;
        tokenPrices = _tokenPrices;
        TvmSlice ts = args.toSlice();
        TvmSlice addressStorage = ts.loadRefAsSlice();
        (address tonWallet, address targetUser, address tip3UserWallet) = addressStorage.decode(address, address, address);
        TvmSlice valueStorage = ts.loadRefAsSlice();
        (uint32 marketId, uint32 marketToLiquidate, uint256 tokensToSeize, uint256 tokensToReturn, uint256 tokensFromReserve) = valueStorage.decode(uint32, uint32, uint256, uint256, uint256);
        TvmSlice borrowInfoStorage = ts.loadRefAsSlice();
        (BorrowInfo borrowInfo) = borrowInfoStorage.decode(BorrowInfo);
        IUAMUserAccount(userAccountManager).seizeTokens{
            flag: MsgFlag.REMAINING_GAS
        }(tonWallet, targetUser, tip3UserWallet, marketId, marketToLiquidate, tokensToSeize, tokensToReturn, tokensFromReserve, borrowInfo);
    }
\end{lstlisting}

\subsection{Function sendActionId}

\noindent\begin{itemize}
\item TODO
\end{itemize}

\begin{lstlisting}[firstnumber=57]
    function sendActionId() external override view responsible returns(uint8) {
        return {flag: MsgFlag.REMAINING_GAS} OperationCodes.LIQUIDATE_TOKENS;
    }
\end{lstlisting}

\subsection{Function setMarketAddress}

\noindent\begin{itemize}
\item TODO
\end{itemize}

\begin{lstlisting}[firstnumber=65]
    function setMarketAddress(address _marketAddress) external override canChangeParams {
        tvm.rawReserve(msg.value, 2);
        marketAddress = _marketAddress;
        address(_owner).transfer({value: 0, flag: MsgFlag.REMAINING_GAS});
    }
\end{lstlisting}

\subsection{Function setUserAccountManager}

\noindent\begin{itemize}
\item TODO
\end{itemize}

\begin{lstlisting}[firstnumber=71]
    function setUserAccountManager(address _userAccountManager) external override canChangeParams {
        tvm.rawReserve(msg.value, 2);
        userAccountManager = _userAccountManager;
        address(_owner).transfer({value: 0, flag: MsgFlag.REMAINING_GAS});
    }
\end{lstlisting}

\subsection{Function updateCache}

\noindent\begin{itemize}
\item TODO
\end{itemize}

\begin{lstlisting}[firstnumber=81]
    function updateCache(address tonWallet, mapping (uint32 => MarketInfo) _marketInfo, mapping (address => fraction) _tokenPrices) external override onlyMarket {
        marketInfo = _marketInfo;
        tokenPrices = _tokenPrices;
        tonWallet.transfer({value: 0, flag: MsgFlag.REMAINING_GAS});
    }
\end{lstlisting}

\subsection{Function upgradeContractCode}

\noindent\begin{itemize}
\item TODO
\end{itemize}

\begin{lstlisting}[firstnumber=23]
    function upgradeContractCode(TvmCell code, TvmCell updateParams, uint32 codeVersion) external override canUpgrade {
        tvm.rawReserve(msg.value, 2);

        tvm.setcode(code);
        tvm.setCurrentCode(code);

        onCodeUpgrade (
            _owner,
            marketAddress,
            userAccountManager,
            marketInfo,
            tokenPrices,
            codeVersion
        );
    }
\end{lstlisting}

\section{Internal Method Definitions}


\subsection{Function \_{}createUpdatedIndexes}

\noindent\begin{itemize}
\item TODO
\end{itemize}

\begin{lstlisting}[firstnumber=229]
    function _createUpdatedIndexes() internal view returns(mapping(uint32 => fraction) updatedIndexes) {
        for ((uint32 marketId, MarketInfo mi): marketInfo) {
            updatedIndexes[marketId] = mi.index;
        }
    }
\end{lstlisting}

\subsection{Function onCodeUpgrade}

\noindent\begin{itemize}
\item TODO
\end{itemize}

\begin{lstlisting}[firstnumber=39]
    function onCodeUpgrade(
        address owner,
        address _marketAddress,
        address _userAccountManager,
        mapping(uint32 => MarketInfo) _marketInfo,
        mapping(address => fraction) _tokenPrices,
        uint32 _codeVersion
    ) private {
        tvm.accept();
        tvm.resetStorage();
        _owner = owner;
        marketAddress = _marketAddress;
        userAccountManager = _userAccountManager;
        marketInfo = _marketInfo;
        tokenPrices = _tokenPrices;
        contractCodeVersion = _codeVersion;
    }
\end{lstlisting}
\paragraph{Some functions inherited by using}


\section{Module "RepayModule.sol"}


\subsection{Pragmas}


\noindent\begin{tabular}{|l|l|p{5cm}|}\hline
ton & -solidity $>$= 0.47.0 &\\\hline
\end{tabular}


\subsection{Imports}


\noindent\begin{tabular}{|l|l|p{5cm}|}\hline
./interfaces/IModule.sol &\\\hline
../utils/libraries/MsgFlag.sol &\\\hline
\end{tabular}


\subsection{Contract Definitions}

\begin{itemize}
\item RepayModule
\end{itemize}


\chapter{Contract SupplyModule}

\minitoc

\section{Overview}


In file {\tt SupplyModule.sol}

\section{Contract Inheritance}


\noindent\begin{tabular}{|l|p{5cm}|}\hline
IRoles & \\\hline
IModule & \\\hline
IContractStateCache & \\\hline
IContractAddressSG & \\\hline
IUpgradableContract & \\\hline
\end{tabular}


\section{Event Definitions}


\begin{lstlisting}[firstnumber=19]
    event TokensSupplied(uint32 marketId, MarketDelta marketDelta, address tonWallet, uint256 tokensSupplied);
\end{lstlisting}

\section{Variable Definitions}


\ifsoltables
\noindent\begin{tabular}{|l|l|p{5cm}|}\hline
address & marketAddress &  \\\hline
 & & used in @9.SupplyModule.upgradeContractCode\\\hline
 & & assigned in @9.SupplyModule.setMarketAddress\\\hline
 & & used in @9.SupplyModule.setMarketAddress\\\hline
 & & used in @9.SupplyModule.performAction\\\hline
 & & assigned in @9.SupplyModule.onCodeUpgrade\\\hline
 & & used in @9.SupplyModule.onCodeUpgrade\\\hline
 & & used in @9.SupplyModule.getContractAddresses\\\hline
address & userAccountManager &  \\\hline
 & & used in @9.SupplyModule.upgradeContractCode\\\hline
 & & assigned in @9.SupplyModule.setUserAccountManager\\\hline
 & & used in @9.SupplyModule.setUserAccountManager\\\hline
 & & used in @9.SupplyModule.resumeOperation\\\hline
 & & assigned in @9.SupplyModule.onCodeUpgrade\\\hline
 & & used in @9.SupplyModule.onCodeUpgrade\\\hline
 & & used in @9.SupplyModule.getContractAddresses\\\hline
uint32 & contractCodeVersion &  \\\hline
 & & assigned in @9.SupplyModule.onCodeUpgrade\\\hline
 & & used in @9.SupplyModule.onCodeUpgrade\\\hline
mapping (uint32 =$>$ MarketInfo) & marketInfo &  \\\hline
 & & used in @9.SupplyModule.upgradeContractCode\\\hline
 & & assigned in @9.SupplyModule.updateCache\\\hline
 & & used in @9.SupplyModule.updateCache\\\hline
 & & used in @9.SupplyModule.resumeOperation\\\hline
 & & assigned in @9.SupplyModule.resumeOperation\\\hline
 & & used in @9.SupplyModule.resumeOperation\\\hline
 & & used in @9.SupplyModule.performAction\\\hline
 & & assigned in @9.SupplyModule.performAction\\\hline
 & & used in @9.SupplyModule.performAction\\\hline
 & & assigned in @9.SupplyModule.onCodeUpgrade\\\hline
 & & used in @9.SupplyModule.onCodeUpgrade\\\hline
 & & used in @9.SupplyModule.getModuleState\\\hline
mapping (address =$>$ fraction) & tokenPrices &  \\\hline
 & & used in @9.SupplyModule.upgradeContractCode\\\hline
 & & assigned in @9.SupplyModule.updateCache\\\hline
 & & used in @9.SupplyModule.updateCache\\\hline
 & & assigned in @9.SupplyModule.resumeOperation\\\hline
 & & used in @9.SupplyModule.resumeOperation\\\hline
 & & assigned in @9.SupplyModule.performAction\\\hline
 & & used in @9.SupplyModule.performAction\\\hline
 & & assigned in @9.SupplyModule.onCodeUpgrade\\\hline
 & & used in @9.SupplyModule.onCodeUpgrade\\\hline
 & & used in @9.SupplyModule.getModuleState\\\hline
\end{tabular}
\fi


\begin{lstlisting}[firstnumber=12]
    address marketAddress;
\end{lstlisting}

\begin{lstlisting}[firstnumber=13]
    address userAccountManager;
\end{lstlisting}

\begin{lstlisting}[firstnumber=14]
    uint32 public contractCodeVersion;
\end{lstlisting}

\begin{lstlisting}[firstnumber=16]
    mapping (uint32 => MarketInfo) marketInfo;
\end{lstlisting}

\begin{lstlisting}[firstnumber=17]
    mapping (address => fraction) tokenPrices;
\end{lstlisting}

\section{Modifier Definitions}


\subsection{Modifier onlyMarket}


\begin{lstlisting}[firstnumber=135]
    modifier onlyMarket() {
        require(msg.sender == marketAddress);
        _;
    }
\end{lstlisting}

\subsection{Modifier onlyUserAccountManager}


\begin{lstlisting}[firstnumber=140]
    modifier onlyUserAccountManager() {
        require(msg.sender == userAccountManager);
        tvm.rawReserve(msg.value, 2);
        _;
    }
\end{lstlisting}

\section{Constructor Definitions}


\subsection{Constructor}

\issueCritical{Constructor for SupplyModule (fake)}{loren ipsum  loren ipsum  loren ipsum loren ipsum loren ipsum loren ipsum loren ipsum loren ipsum loren ipsum loren ipsum loren ipsum loren ipsum loren ipsum loren ipsum loren ipsum loren ipsum loren ipsum loren ipsum

loren ipsum loren ipsum loren ipsum loren ipsum loren ipsum loren ipsum
loren ipsum loren ipsum loren ipsum }
\noindent\begin{itemize}
\item TODO
\end{itemize}

\begin{lstlisting}[firstnumber=21]
    constructor(address _newOwner) public {
        tvm.accept();
        _owner = _newOwner;
    }
\end{lstlisting}

\section{Public Method Definitions}


\subsection{Function getContractAddresses}

\noindent\begin{itemize}
\item TODO
\end{itemize}

\begin{lstlisting}[firstnumber=80]
    function getContractAddresses() external override view responsible returns(address _owner, address _marketAddress, address _userAccountManager) {
        return {flag: MsgFlag.REMAINING_GAS} (_owner, marketAddress, userAccountManager);
    }
\end{lstlisting}

\subsection{Function getModuleState}

\noindent\begin{itemize}
\item TODO
\end{itemize}

\begin{lstlisting}[firstnumber=64]
    function getModuleState() external override view returns(mapping(uint32 => MarketInfo), mapping(address => fraction)) {
        return(marketInfo, tokenPrices);
    }
\end{lstlisting}

\subsection{Function performAction}

\noindent\begin{itemize}
\item TODO
\end{itemize}

\begin{lstlisting}[firstnumber=91]
    function performAction(uint32 marketId, TvmCell args, mapping (uint32 => MarketInfo) _marketInfo, mapping (address => fraction) _tokenPrices) external override onlyMarket {
        tvm.rawReserve(msg.value, 2);
        marketInfo = _marketInfo;
        tokenPrices = _tokenPrices;
        TvmSlice ts = args.toSlice();
        (address tonWallet, uint256 tokenAmount) = ts.decode(address, uint256);

        // Supply process:
        // 1. Convert real tokens to vTokens by dividing real token amount by exchange rate
        fraction vTokensToProvide = tokenAmount.numFDiv(marketInfo[marketId].exchangeRate);

        MarketDelta marketDelta;
        mapping(uint32 => MarketDelta) marketsDelta;
        marketDelta.realTokenBalance.delta = tokenAmount;
        marketDelta.realTokenBalance.positive = true;
        marketDelta.vTokenBalance.delta = vTokensToProvide.toNum();
        marketDelta.vTokenBalance.positive = true;
        marketsDelta[marketId] = marketDelta;

        TvmBuilder tb;
        tb.store(marketId);
        tb.store(tonWallet);
        tb.store(vTokensToProvide.toNum());

        emit TokensSupplied(marketId, marketDelta, tonWallet, tokenAmount);

        IContractStateCacheRoot(marketAddress).receiveCacheDelta{
            flag: MsgFlag.REMAINING_GAS
        }(marketsDelta, tb.toCell());
    }
\end{lstlisting}

\subsection{Function resumeOperation}

\noindent\begin{itemize}
\item TODO
\end{itemize}

\begin{lstlisting}[firstnumber=122]
    function resumeOperation(TvmCell args, mapping(uint32 => MarketInfo) _marketInfo, mapping (address => fraction) _tokenPrices) external override onlyMarket {
        tvm.rawReserve(msg.value, 2);
        marketInfo = _marketInfo;
        tokenPrices = _tokenPrices;

        TvmSlice ts = args.toSlice();
        (uint32 marketId, address tonWallet, uint256 vTokensToProvide) = ts.decode(uint32, address, uint256);

        IUAMUserAccount(userAccountManager).writeSupplyInfo{
            flag: MsgFlag.REMAINING_GAS
        }(tonWallet, marketId, vTokensToProvide, marketInfo[marketId].index);
    }
\end{lstlisting}

\subsection{Function sendActionId}

\noindent\begin{itemize}
\item TODO
\end{itemize}

\begin{lstlisting}[firstnumber=60]
    function sendActionId() external override view responsible returns(uint8) {
        return {flag: MsgFlag.REMAINING_GAS} OperationCodes.SUPPLY_TOKENS;
    }
\end{lstlisting}

\subsection{Function setMarketAddress}

\noindent\begin{itemize}
\item TODO
\end{itemize}

\begin{lstlisting}[firstnumber=68]
    function setMarketAddress(address _marketAddress) external override canChangeParams {
        tvm.rawReserve(msg.value, 2);
        marketAddress = _marketAddress;
        address(_owner).transfer({value: 0, flag: MsgFlag.REMAINING_GAS});
    }
\end{lstlisting}

\subsection{Function setUserAccountManager}

\noindent\begin{itemize}
\item TODO
\end{itemize}

\begin{lstlisting}[firstnumber=74]
    function setUserAccountManager(address _userAccountManager) external override canChangeParams {
        tvm.rawReserve(msg.value, 2);
        userAccountManager = _userAccountManager;
        address(_owner).transfer({value: 0, flag: MsgFlag.REMAINING_GAS});
    }
\end{lstlisting}

\subsection{Function updateCache}

\noindent\begin{itemize}
\item TODO
\end{itemize}

\begin{lstlisting}[firstnumber=84]
    function updateCache(address tonWallet, mapping (uint32 => MarketInfo) _marketInfo, mapping (address => fraction) _tokenPrices) external override onlyMarket {
        tvm.rawReserve(msg.value, 2);
        marketInfo = _marketInfo;
        tokenPrices = _tokenPrices;
        tonWallet.transfer({value: 0, flag: MsgFlag.REMAINING_GAS});
    }
\end{lstlisting}

\subsection{Function upgradeContractCode}

\noindent\begin{itemize}
\item TODO
\end{itemize}

\begin{lstlisting}[firstnumber=26]
    function upgradeContractCode(TvmCell code, TvmCell updateParams, uint32 codeVersion) external override canUpgrade {
        tvm.rawReserve(msg.value, 2);

        tvm.setcode(code);
        tvm.setCurrentCode(code);

        onCodeUpgrade (
            _owner,
            marketAddress,
            userAccountManager,
            marketInfo,
            tokenPrices,
            codeVersion
        );
    }
\end{lstlisting}

\section{Internal Method Definitions}


\subsection{Function onCodeUpgrade}

\noindent\begin{itemize}
\item TODO
\end{itemize}

\begin{lstlisting}[firstnumber=42]
    function onCodeUpgrade(
        address owner,
        address _marketAddress,
        address _userAccountManager,
        mapping(uint32 => MarketInfo) _marketInfo,
        mapping(address => fraction) _tokenPrices,
        uint32 _codeVersion
    ) private {
        tvm.accept();
        tvm.resetStorage();
        _owner = owner;
        marketAddress = _marketAddress;
        userAccountManager = _userAccountManager;
        marketInfo = _marketInfo;
        tokenPrices = _tokenPrices;
        contractCodeVersion = _codeVersion;
    }
\end{lstlisting}
\paragraph{Some functions inherited by using}


\chapter{Contract WithdrawModule}

\minitoc

\section{Overview}


In file {\tt WithdrawModule.sol}

\section{Contract Inheritance}


\noindent\begin{tabular}{|l|p{5cm}|}\hline
IRoles & \\\hline
IModule & \\\hline
IContractStateCache & \\\hline
IContractAddressSG & \\\hline
IWithdrawModule & \\\hline
IUpgradableContract & \\\hline
\end{tabular}


\section{Event Definitions}


\begin{lstlisting}[firstnumber=18]
    event TokenWithdraw(uint32 marketId, MarketDelta marketDelta, address tonWallet, uint256 vTokensWithdrawn, uint256 realTokensWithdrawn);
\end{lstlisting}

\section{Variable Definitions}


\ifsoltables
\noindent\begin{tabular}{|l|l|p{5cm}|}\hline
address & marketAddress &  \\\hline
 & & used in @10.WithdrawModule.withdrawTokensFromMarket\\\hline
 & & used in @10.WithdrawModule.upgradeContractCode\\\hline
 & & assigned in @10.WithdrawModule.setMarketAddress\\\hline
 & & used in @10.WithdrawModule.setMarketAddress\\\hline
 & & assigned in @10.WithdrawModule.onCodeUpgrade\\\hline
 & & used in @10.WithdrawModule.onCodeUpgrade\\\hline
 & & used in @10.WithdrawModule.getContractAddresses\\\hline
address & userAccountManager &  \\\hline
 & & used in @10.WithdrawModule.withdrawTokensFromMarket\\\hline
 & & used in @10.WithdrawModule.withdrawTokensFromMarket\\\hline
 & & used in @10.WithdrawModule.upgradeContractCode\\\hline
 & & assigned in @10.WithdrawModule.setUserAccountManager\\\hline
 & & used in @10.WithdrawModule.setUserAccountManager\\\hline
 & & used in @10.WithdrawModule.resumeOperation\\\hline
 & & used in @10.WithdrawModule.performAction\\\hline
 & & assigned in @10.WithdrawModule.onCodeUpgrade\\\hline
 & & used in @10.WithdrawModule.onCodeUpgrade\\\hline
 & & used in @10.WithdrawModule.getContractAddresses\\\hline
uint32 & contractCodeVersion &  \\\hline
 & & assigned in @10.WithdrawModule.onCodeUpgrade\\\hline
 & & used in @10.WithdrawModule.onCodeUpgrade\\\hline
mapping (uint32 =$>$ MarketInfo) & marketInfo &  \\\hline
 & & used in @10.WithdrawModule.withdrawTokensFromMarket\\\hline
 & & used in @10.WithdrawModule.withdrawTokensFromMarket\\\hline
 & & used in @10.WithdrawModule.withdrawTokensFromMarket\\\hline
 & & used in @10.WithdrawModule.upgradeContractCode\\\hline
 & & assigned in @10.WithdrawModule.updateCache\\\hline
 & & used in @10.WithdrawModule.updateCache\\\hline
 & & assigned in @10.WithdrawModule.resumeOperation\\\hline
 & & used in @10.WithdrawModule.resumeOperation\\\hline
 & & assigned in @10.WithdrawModule.performAction\\\hline
 & & used in @10.WithdrawModule.performAction\\\hline
 & & assigned in @10.WithdrawModule.onCodeUpgrade\\\hline
 & & used in @10.WithdrawModule.onCodeUpgrade\\\hline
 & & used in @10.WithdrawModule.getModuleState\\\hline
 & & used in @10.WithdrawModule.\_{}createUpdatedIndexes\\\hline
mapping (address =$>$ fraction) & tokenPrices &  \\\hline
 & & used in @10.WithdrawModule.withdrawTokensFromMarket\\\hline
 & & used in @10.WithdrawModule.withdrawTokensFromMarket\\\hline
 & & used in @10.WithdrawModule.upgradeContractCode\\\hline
 & & assigned in @10.WithdrawModule.updateCache\\\hline
 & & used in @10.WithdrawModule.updateCache\\\hline
 & & assigned in @10.WithdrawModule.resumeOperation\\\hline
 & & used in @10.WithdrawModule.resumeOperation\\\hline
 & & assigned in @10.WithdrawModule.performAction\\\hline
 & & used in @10.WithdrawModule.performAction\\\hline
 & & assigned in @10.WithdrawModule.onCodeUpgrade\\\hline
 & & used in @10.WithdrawModule.onCodeUpgrade\\\hline
 & & used in @10.WithdrawModule.getModuleState\\\hline
\end{tabular}
\fi


\begin{lstlisting}[firstnumber=11]
    address marketAddress;
\end{lstlisting}

\begin{lstlisting}[firstnumber=12]
    address userAccountManager;
\end{lstlisting}

\begin{lstlisting}[firstnumber=13]
    uint32 public contractCodeVersion;
\end{lstlisting}

\begin{lstlisting}[firstnumber=15]
    mapping (uint32 => MarketInfo) marketInfo;
\end{lstlisting}

\begin{lstlisting}[firstnumber=16]
    mapping (address => fraction) tokenPrices;
\end{lstlisting}

\section{Modifier Definitions}


\subsection{Modifier onlyMarket}


\begin{lstlisting}[firstnumber=190]
    modifier onlyMarket() {
        require(msg.sender == marketAddress);
        tvm.rawReserve(msg.value, 2);
        _;
    }
\end{lstlisting}

\subsection{Modifier onlyUserAccountManager}


\begin{lstlisting}[firstnumber=196]
    modifier onlyUserAccountManager() {
        require(msg.sender == userAccountManager);
        _;
    }
\end{lstlisting}

\section{Constructor Definitions}


\subsection{Constructor}

\issueCritical{Constructor for WithdrawModule (fake)}{loren ipsum  loren ipsum  loren ipsum loren ipsum loren ipsum loren ipsum loren ipsum loren ipsum loren ipsum loren ipsum loren ipsum loren ipsum loren ipsum loren ipsum loren ipsum loren ipsum loren ipsum loren ipsum

loren ipsum loren ipsum loren ipsum loren ipsum loren ipsum loren ipsum
loren ipsum loren ipsum loren ipsum }
\noindent\begin{itemize}
\item TODO
\end{itemize}

\begin{lstlisting}[firstnumber=20]
    constructor(address _newOwner) public {
        tvm.accept();
        _owner = _newOwner;
    }
\end{lstlisting}

\section{Public Method Definitions}


\subsection{Function getContractAddresses}

\noindent\begin{itemize}
\item TODO
\end{itemize}

\begin{lstlisting}[firstnumber=79]
    function getContractAddresses() external override view responsible returns(address _owner, address _marketAddress, address _userAccountManager) {
        return {flag: MsgFlag.REMAINING_GAS} (_owner, marketAddress, userAccountManager);
    }
\end{lstlisting}

\subsection{Function getModuleState}

\noindent\begin{itemize}
\item TODO
\end{itemize}

\begin{lstlisting}[firstnumber=63]
    function getModuleState() external override view returns(mapping(uint32 => MarketInfo), mapping(address => fraction)) {
        return(marketInfo, tokenPrices);
    }
\end{lstlisting}

\subsection{Function performAction}

\noindent\begin{itemize}
\item TODO
\end{itemize}

\begin{lstlisting}[firstnumber=89]
    function performAction(uint32 marketId, TvmCell args, mapping (uint32 => MarketInfo) _marketInfo, mapping (address => fraction) _tokenPrices) external override onlyMarket {
        TvmSlice ts = args.toSlice();
        marketInfo = _marketInfo;
        tokenPrices = _tokenPrices;
        (address tonWallet, address userTip3Wallet, uint256 tokensToWithdraw) = ts.decode(address, address, uint256);
        mapping(uint32 => fraction) updatedIndexes = _createUpdatedIndexes();
        IUAMUserAccount(userAccountManager).requestWithdrawInfo{
            flag: MsgFlag.REMAINING_GAS
        }(tonWallet, userTip3Wallet, tokensToWithdraw, marketId, updatedIndexes);
    }
\end{lstlisting}

\subsection{Function resumeOperation}

\noindent\begin{itemize}
\item TODO
\end{itemize}

\begin{lstlisting}[firstnumber=177]
    function resumeOperation(TvmCell args, mapping(uint32 => MarketInfo) _marketInfo, mapping (address => fraction) _tokenPrices) external override onlyMarket {
        tvm.rawReserve(msg.value, 2);
        marketInfo = _marketInfo;
        tokenPrices = _tokenPrices;
        TvmSlice ts = args.toSlice();
        (uint32 marketId, address tonWallet, address userTip3Wallet) = ts.decode(uint32, address, address);
        TvmSlice values = ts.loadRefAsSlice();
        (uint256 tokensToWithdraw, uint256 tokensToSend) = values.decode(uint256, uint256);
        IUAMUserAccount(userAccountManager).writeWithdrawInfo{
            flag: MsgFlag.REMAINING_GAS
        }(tonWallet, userTip3Wallet, marketId, tokensToWithdraw, tokensToSend);
    }
\end{lstlisting}

\subsection{Function sendActionId}

\noindent\begin{itemize}
\item TODO
\end{itemize}

\begin{lstlisting}[firstnumber=59]
    function sendActionId() external override view responsible returns(uint8) {
        return {flag: MsgFlag.REMAINING_GAS} OperationCodes.WITHDRAW_TOKENS;
    }
\end{lstlisting}

\subsection{Function setMarketAddress}

\noindent\begin{itemize}
\item TODO
\end{itemize}

\begin{lstlisting}[firstnumber=67]
    function setMarketAddress(address _marketAddress) external override canChangeParams {
        tvm.rawReserve(msg.value, 2);
        marketAddress = _marketAddress;
        address(_owner).transfer({value: 0, flag: MsgFlag.REMAINING_GAS});
    }
\end{lstlisting}

\subsection{Function setUserAccountManager}

\noindent\begin{itemize}
\item TODO
\end{itemize}

\begin{lstlisting}[firstnumber=73]
    function setUserAccountManager(address _userAccountManager) external override canChangeParams {
        tvm.rawReserve(msg.value, 2);
        userAccountManager = _userAccountManager;
        address(_owner).transfer({value: 0, flag: MsgFlag.REMAINING_GAS});
    }
\end{lstlisting}

\subsection{Function updateCache}

\noindent\begin{itemize}
\item TODO
\end{itemize}

\begin{lstlisting}[firstnumber=83]
    function updateCache(address tonWallet, mapping (uint32 => MarketInfo) _marketInfo, mapping (address => fraction) _tokenPrices) external override onlyMarket {
        marketInfo = _marketInfo;
        tokenPrices = _tokenPrices;
        tonWallet.transfer({value: 0, flag: MsgFlag.REMAINING_GAS});
    }
\end{lstlisting}

\subsection{Function upgradeContractCode}

\noindent\begin{itemize}
\item TODO
\end{itemize}

\begin{lstlisting}[firstnumber=25]
    function upgradeContractCode(TvmCell code, TvmCell updateParams, uint32 codeVersion) external override canUpgrade {
        tvm.rawReserve(msg.value, 2);

        tvm.setcode(code);
        tvm.setCurrentCode(code);

        onCodeUpgrade (
            _owner,
            marketAddress,
            userAccountManager,
            marketInfo,
            tokenPrices,
            codeVersion
        );
    }
\end{lstlisting}

\subsection{Function withdrawTokensFromMarket}

\noindent\begin{itemize}
\item TODO
\end{itemize}

\begin{lstlisting}[firstnumber=106]
    function withdrawTokensFromMarket(
        address tonWallet, 
        address userTip3Wallet,
        uint256 tokensToWithdraw, 
        uint32 marketId, 
        mapping(uint32 => uint256) supplyInfo,
        mapping(uint32 => BorrowInfo) borrowInfo
    ) external override onlyUserAccountManager {
        tvm.rawReserve(msg.value, 2);
        MarketDelta marketDelta;
        mapping(uint32 => MarketDelta) marketsDelta;

        MarketInfo mi = marketInfo[marketId];

        // For token withdraw:
        // 1. Calculate account health
        // 2. Calculate USD amount for withdraw token
        // 3. Check if user can afford to withdraw required amount of real tokens

        fraction accountHealth = Utilities.calculateSupplyBorrow(supplyInfo, borrowInfo, marketInfo, tokenPrices);

        fraction fTokensToSend = tokensToWithdraw.numFMul(mi.exchangeRate);
        fraction fTokensToSendUSD = fTokensToSend.fDiv(tokenPrices[marketInfo[marketId].token]);

        // Check user balance in tokens just in case
        // There will be lock at user account for operation, unified for all operations
        // As all operations are finished with account health check, account will unlock after
        // Updating indexes
        if (
            (accountHealth.nom > accountHealth.denom) &&
            (supplyInfo[marketId] >= tokensToWithdraw)
        ) {
            if (
                accountHealth.nom - accountHealth.denom >= fTokensToSendUSD.toNum() &&
                fTokensToSend.toNum() <= mi.realTokenBalance - mi.totalReserve
            ) {
                uint256 tokensToSend = fTokensToSend.toNum();

                marketDelta.realTokenBalance.delta = tokensToSend;
                marketDelta.realTokenBalance.positive = false;
                marketDelta.vTokenBalance.delta = tokensToWithdraw;
                marketDelta.vTokenBalance.positive = false;

                marketsDelta[marketId] = marketDelta;

                emit TokenWithdraw(marketId, marketDelta, tonWallet, tokensToWithdraw, tokensToSend);

                TvmBuilder tb;
                tb.store(marketId);
                tb.store(tonWallet);
                tb.store(userTip3Wallet);
                TvmBuilder valueStorate;
                valueStorate.store(tokensToWithdraw);
                valueStorate.store(tokensToSend);
                tb.store(valueStorate.toCell());

                IContractStateCacheRoot(marketAddress).receiveCacheDelta{
                    flag: MsgFlag.REMAINING_GAS
                }(marketsDelta, tb.toCell());
            } else {
                IUAMUserAccount(userAccountManager).requestUserAccountHealthCalculation{
                    flag: MsgFlag.REMAINING_GAS
                }(tonWallet);
            }
        } else {
            IUAMUserAccount(userAccountManager).requestUserAccountHealthCalculation{
                flag: MsgFlag.REMAINING_GAS
            }(tonWallet);
        }
    }
\end{lstlisting}

\section{Internal Method Definitions}


\subsection{Function \_{}createUpdatedIndexes}

\noindent\begin{itemize}
\item TODO
\end{itemize}

\begin{lstlisting}[firstnumber=100]
    function _createUpdatedIndexes() internal view returns(mapping(uint32 => fraction) updatedIndexes) {
        for ((uint32 marketId, MarketInfo mi): marketInfo) {
            updatedIndexes[marketId] = mi.index;
        }
    }
\end{lstlisting}

\subsection{Function onCodeUpgrade}

\noindent\begin{itemize}
\item TODO
\end{itemize}

\begin{lstlisting}[firstnumber=41]
    function onCodeUpgrade(
        address owner,
        address _marketAddress,
        address _userAccountManager,
        mapping(uint32 => MarketInfo) _marketInfo,
        mapping(address => fraction) _tokenPrices,
        uint32 _codeVersion
    ) private {
        tvm.accept();
        tvm.resetStorage();
        _owner = owner;
        marketAddress = _marketAddress;
        userAccountManager = _userAccountManager;
        marketInfo = _marketInfo;
        tokenPrices = _tokenPrices;
        contractCodeVersion = _codeVersion;
    }
\end{lstlisting}
\paragraph{Some functions inherited by using}



\chapter{Contract Oracle}

\minitoc

\section{Overview}


In file {\tt Oracle.sol}

\section{Contract Inheritance}


\noindent\begin{tabular}{|l|p{5cm}|}\hline
IRoles & \\\hline
IOracleService & \\\hline
IOracleUpdatePrices & \\\hline
IOracleReturnPrices & \\\hline
IOracleManageTokens & \\\hline
IUpgradableContract & \\\hline
\end{tabular}


\section{Static Variable Definitions}


\ifsoltables
\noindent\begin{tabular}{|l|l|p{5cm}|}\hline
uint256 & nonce &  \\\hline
 & & used in @11.Oracle.upgradeContractCode\\\hline
 & & assigned in @11.Oracle.onCodeUpgrade\\\hline
 & & used in @11.Oracle.onCodeUpgrade\\\hline
\end{tabular}
\fi


\begin{lstlisting}[firstnumber=22]
    uint256 static nonce;
\end{lstlisting}

\section{Variable Definitions}


\ifsoltables
\noindent\begin{tabular}{|l|l|p{5cm}|}\hline
mapping (address =$>$ MarketPriceInfo) & prices &  \\\hline
 & & used in @11.Oracle.upgradeContractCode\\\hline
 & & assigned in @11.Oracle.removeToken\\\hline
 & & used in @11.Oracle.removeToken\\\hline
 & & used in @11.Oracle.removeToken\\\hline
 & & assigned in @11.Oracle.onCodeUpgrade\\\hline
 & & used in @11.Oracle.onCodeUpgrade\\\hline
 & & used in @11.Oracle.internalUpdatePrice\\\hline
 & & assigned in @11.Oracle.internalGetUpdatedPrice\\\hline
 & & used in @11.Oracle.internalGetUpdatedPrice\\\hline
 & & used in @11.Oracle.internalGetUpdatedPrice\\\hline
 & & assigned in @11.Oracle.internalGetUpdatedPrice\\\hline
 & & used in @11.Oracle.internalGetUpdatedPrice\\\hline
 & & used in @11.Oracle.internalGetUpdatedPrice\\\hline
 & & used in @11.Oracle.getTokenPrice\\\hline
 & & used in @11.Oracle.getTokenPrice\\\hline
 & & used in @11.Oracle.getAllTokenPrices\\\hline
 & & assigned in @11.Oracle.externalUpdatePrice\\\hline
 & & used in @11.Oracle.externalUpdatePrice\\\hline
 & & assigned in @11.Oracle.externalUpdatePrice\\\hline
 & & used in @11.Oracle.externalUpdatePrice\\\hline
 & & assigned in @11.Oracle.addToken\\\hline
 & & used in @11.Oracle.addToken\\\hline
mapping (address =$>$ address) & swapPairToTokenRoot &  \\\hline
 & & used in @11.Oracle.upgradeContractCode\\\hline
 & & assigned in @11.Oracle.removeToken\\\hline
 & & used in @11.Oracle.removeToken\\\hline
 & & assigned in @11.Oracle.onCodeUpgrade\\\hline
 & & used in @11.Oracle.onCodeUpgrade\\\hline
 & & used in @11.Oracle.internalGetUpdatedPrice\\\hline
 & & assigned in @11.Oracle.addToken\\\hline
 & & used in @11.Oracle.addToken\\\hline
uint32 & contractCodeVersion &  \\\hline
 & & assigned in @11.Oracle.onCodeUpgrade\\\hline
 & & used in @11.Oracle.onCodeUpgrade\\\hline
 & & used in @11.Oracle.getVersion\\\hline
 & & used in @11.Oracle.getDetails\\\hline
\end{tabular}
\fi


\begin{lstlisting}[firstnumber=26]
    mapping(address => MarketPriceInfo) prices;
\end{lstlisting}

\begin{lstlisting}[firstnumber=28]
    mapping(address => address) swapPairToTokenRoot;
\end{lstlisting}

\begin{lstlisting}[firstnumber=31]
    uint32 contractCodeVersion;
\end{lstlisting}

\section{Modifier Definitions}


\subsection{Modifier onlyTrustedSwapPair}


\begin{lstlisting}[firstnumber=198]
    modifier onlyTrustedSwapPair() {
        require(swapPairToTokenRoot.exists(msg.sender), OracleErrorCodes.ERROR_NOT_KNOWN_SWAP_PAIR);
        _;
    }
\end{lstlisting}

\subsection{Modifier onlyKnownTokenRoot}


\begin{lstlisting}[firstnumber=203]
    modifier onlyKnownTokenRoot(address _tokenRoot) {
        require(prices.exists(_tokenRoot), OracleErrorCodes.ERROR_NOT_KNOWN_TOKEN_ROOT);
        _;
    }
\end{lstlisting}

\section{Constructor Definitions}


\subsection{Constructor}

\issueCritical{Constructor for Oracle (fake)}{loren ipsum  loren ipsum  loren ipsum loren ipsum loren ipsum loren ipsum loren ipsum loren ipsum loren ipsum loren ipsum loren ipsum loren ipsum loren ipsum loren ipsum loren ipsum loren ipsum loren ipsum loren ipsum

loren ipsum loren ipsum loren ipsum loren ipsum loren ipsum loren ipsum
loren ipsum loren ipsum loren ipsum }
\noindent\begin{itemize}
\item TODO
\end{itemize}

\begin{lstlisting}[firstnumber=36]
    constructor(address _newOwner) public {
        tvm.accept();
        _owner = _newOwner;
    }
\end{lstlisting}

\section{Public Method Definitions}


\subsection{Function addToken}

\noindent\begin{itemize}
\item TODO
\end{itemize}

\begin{lstlisting}[firstnumber=179]
    function addToken(address tokenRoot, address swapPairAddress, bool isLeft) override external canChangeParams {
        tvm.accept();
        swapPairToTokenRoot[swapPairAddress] = tokenRoot;
        prices[tokenRoot] = MarketPriceInfo(swapPairAddress, isLeft, 0, 0);
        this.internalUpdatePrice{value: CostConstants.TOKEN_INITIAL_UPDATE_PRICE, bounce: false}(tokenRoot);
    }
\end{lstlisting}

\subsection{Function externalUpdatePrice}

\noindent\begin{itemize}
\item TODO
\end{itemize}

\begin{lstlisting}[firstnumber=117]
    function externalUpdatePrice(address tokenRoot, uint128 tokens, uint128 usd) override external canChangeParams onlyKnownTokenRoot(tokenRoot) {
        if (msg.sender.value == 0) {
            tvm.accept();
        } else {
            tvm.rawReserve(msg.value, 2);
        }

        prices[tokenRoot].tokens = tokens;
        prices[tokenRoot].usd = usd;

        address(msg.sender).transfer({value: 0, flag: 64});
    }
\end{lstlisting}

\subsection{Function getAllTokenPrices}

\noindent\begin{itemize}
\item TODO
\end{itemize}

\begin{lstlisting}[firstnumber=167]
    function getAllTokenPrices(TvmCell payload) override external responsible view returns (mapping(address => MarketPriceInfo), TvmCell) {
        tvm.rawReserve(msg.value, 2);
        return { value: 0, bounce: false, flag: MsgFlag.REMAINING_GAS } (prices, payload);
    }
\end{lstlisting}

\subsection{Function getDetails}

\noindent\begin{itemize}
\item TODO
\end{itemize}

\begin{lstlisting}[firstnumber=104]
    function getDetails() override external responsible view returns (OracleServiceInformation) {
        tvm.rawReserve(msg.value, 2);
        return { value: 0, bounce: false, flag: MsgFlag.REMAINING_GAS } OracleServiceInformation(contractCodeVersion, _owner);
    }
\end{lstlisting}

\subsection{Function getTokenPrice}

\noindent\begin{itemize}
\item TODO
\end{itemize}

\begin{lstlisting}[firstnumber=159]
    function getTokenPrice(address tokenRoot, TvmCell payload) override external responsible view returns(address, uint128, uint128, TvmCell) {
        tvm.rawReserve(msg.value, 2);
        return { value: 0, bounce: false, flag: MsgFlag.REMAINING_GAS } (tokenRoot, prices[tokenRoot].tokens, prices[tokenRoot].usd, payload);
    }
\end{lstlisting}

\subsection{Function getVersion}

\noindent\begin{itemize}
\item TODO
\end{itemize}

\begin{lstlisting}[firstnumber=99]
    function getVersion() override external responsible view returns (uint32) { 
        tvm.rawReserve(msg.value, 2);
        return { value: 0, bounce: false, flag: MsgFlag.REMAINING_GAS } contractCodeVersion;
    }
\end{lstlisting}

\subsection{Function internalGetUpdatedPrice}

\noindent\begin{itemize}
\item TODO
\end{itemize}

\begin{lstlisting}[firstnumber=146]
    function internalGetUpdatedPrice(IDexPairBalances updatedPrice) override external onlyTrustedSwapPair {
        tvm.rawReserve(msg.value, 2);
        address affectedToken = swapPairToTokenRoot[msg.sender];
        prices[affectedToken].tokens = prices[affectedToken].isLeft ? updatedPrice.left_balance : updatedPrice.right_balance;
        prices[affectedToken].usd = prices[affectedToken].isLeft ? updatedPrice.right_balance : updatedPrice.left_balance;
    }
\end{lstlisting}

\subsection{Function internalUpdatePrice}

\noindent\begin{itemize}
\item TODO
\end{itemize}

\begin{lstlisting}[firstnumber=133]
    function internalUpdatePrice(address tokenRoot) override external onlyKnownTokenRoot(tokenRoot) {
        tvm.rawReserve(msg.value, 2);
        IDexPair(prices[tokenRoot].swapPair).getBalances{
            value: 0, 
            bounce: true, 
            flag: MsgFlag.REMAINING_GAS,
            callback: this.internalGetUpdatedPrice
        }();
    }
\end{lstlisting}

\subsection{Function removeToken}

\noindent\begin{itemize}
\item TODO
\end{itemize}

\begin{lstlisting}[firstnumber=189]
    function removeToken(address tokenRoot) override external canChangeParams {
        tvm.accept();
        delete swapPairToTokenRoot[prices[tokenRoot].swapPair];
        delete prices[tokenRoot];
    }
\end{lstlisting}

\subsection{Function upgradeContractCode}

\noindent\begin{itemize}
\item TODO
\end{itemize}

\begin{lstlisting}[firstnumber=62]
    function upgradeContractCode(TvmCell code, TvmCell updateParams, uint32 codeVersion) override external canUpgrade {
        tvm.accept();

        tvm.setcode(code);
        tvm.setCurrentCode(code);

        onCodeUpgrade(
            nonce,
            prices,
            swapPairToTokenRoot,
            0,
            _owner,
            updateParams,
            codeVersion
        );
    }
\end{lstlisting}

\section{Internal Method Definitions}


\subsection{Function onCodeUpgrade}

\noindent\begin{itemize}
\item TODO
\end{itemize}

\begin{lstlisting}[firstnumber=79]
    function onCodeUpgrade(
        uint256 _nonce,
        mapping(address => MarketPriceInfo) _prices,
        mapping(address => address) _swapPairToTokenRoot,
        uint256,
        address _ownerAddress,
        TvmCell,
        uint32 _codeVersion
    ) private {
        tvm.accept();
        tvm.resetStorage();
        nonce = _nonce;
        prices = _prices;
        swapPairToTokenRoot = _swapPairToTokenRoot;
        _owner = _ownerAddress;
        contractCodeVersion = _codeVersion;
    }
\end{lstlisting}


\section{Module "TIP3Deployer.sol"}


\subsection{Pragmas}


\noindent\begin{tabular}{|l|l|p{5cm}|}\hline
ton & -solidity $>$= 0.39.0 &\\\hline
AbiHeader &  pubkey &\\\hline
AbiHeader &  expire &\\\hline
AbiHeader &  time &\\\hline
\end{tabular}


\subsection{Imports}


\noindent\begin{tabular}{|l|l|p{5cm}|}\hline
./interfaces/ITIP3Deployer.sol &\\\hline
./interfaces/ITIP3DeployerManageCode.sol &\\\hline
./interfaces/ITIP3DeployerServiceInfo.sol &\\\hline
./libraries/TIP3DeployerErrorCodes.sol &\\\hline
../utils/libraries/MsgFlag.sol &\\\hline
../utils/interfaces/IUpgradableContract.sol &\\\hline
../utils/TIP3/RootTokenContract.sol &\\\hline
\end{tabular}


\subsection{Contract Definitions}

\begin{itemize}
\item TIP3TokenDeployer
\end{itemize}


\chapter{Contract UserAccount}

\minitoc

\section{Overview}


In file {\tt UserAccount.sol}

\section{Contract Inheritance}


\noindent\begin{tabular}{|l|p{5cm}|}\hline
IUserAccount & \\\hline
IUserAccountData & \\\hline
IUpgradableContract & \\\hline
IUserAccountGetters & \\\hline
\end{tabular}


\section{Static Variable Definitions}


\ifsoltables
\noindent\begin{tabular}{|l|l|p{5cm}|}\hline
address & owner &  \\\hline
 & & used in @13.UserAccount.writeWithdrawInfo\\\hline
 & & used in @13.UserAccount.writeWithdrawInfo\\\hline
 & & used in @13.UserAccount.writeSupplyInfo\\\hline
 & & used in @13.UserAccount.writeRepayInformation\\\hline
 & & used in @13.UserAccount.writeRepayInformation\\\hline
 & & used in @13.UserAccount.writeRepayInformation\\\hline
 & & used in @13.UserAccount.writeBorrowInformation\\\hline
 & & used in @13.UserAccount.writeBorrowInformation\\\hline
 & & used in @13.UserAccount.writeBorrowInformation\\\hline
 & & used in @13.UserAccount.withdrawExtraTons\\\hline
 & & used in @13.UserAccount.withdraw\\\hline
 & & used in @13.UserAccount.withdraw\\\hline
 & & used in @13.UserAccount.upgradeContractCode\\\hline
 & & used in @13.UserAccount.sendRepayInfo\\\hline
 & & used in @13.UserAccount.requestWithdrawInfo\\\hline
 & & used in @13.UserAccount.requestWithdrawInfo\\\hline
 & & used in @13.UserAccount.requestLiquidationInformation\\\hline
 & & assigned in @13.UserAccount.onCodeUpgrade\\\hline
 & & used in @13.UserAccount.onCodeUpgrade\\\hline
 & & used in @13.UserAccount.liquidateVTokens\\\hline
 & & used in @13.UserAccount.grantVTokens\\\hline
 & & used in @13.UserAccount.grantVTokens\\\hline
 & & used in @13.UserAccount.grantVTokens\\\hline
 & & used in @13.UserAccount.grantVTokens\\\hline
 & & used in @13.UserAccount.getOwner\\\hline
 & & used in @13.UserAccount.enterMarket\\\hline
 & & used in @13.UserAccount.borrowUpdateIndexes\\\hline
 & & used in @13.UserAccount.borrow\\\hline
 & & used in @13.UserAccount.borrow\\\hline
 & & used in @13.UserAccount.abortLiquidation\\\hline
 & & used in @13.UserAccount.abortLiquidation\\\hline
 & & used in @13.UserAccount.\_{}checkUserAccountHealth\\\hline
\end{tabular}
\fi


\begin{lstlisting}[firstnumber=23]
    address static public owner;
\end{lstlisting}

\section{Variable Definitions}


\ifsoltables
\noindent\begin{tabular}{|l|l|p{5cm}|}\hline
bool & borrowLock &  \\\hline
 & & assigned in @13.UserAccount.writeBorrowInformation\\\hline
 & & used in @13.UserAccount.writeBorrowInformation\\\hline
 & & used in @13.UserAccount.upgradeContractCode\\\hline
 & & used in @13.UserAccount.upgradeContractCode\\\hline
 & & assigned in @13.UserAccount.updateUserAccountHealth\\\hline
 & & used in @13.UserAccount.updateUserAccountHealth\\\hline
 & & assigned in @13.UserAccount.onCodeUpgrade\\\hline
 & & used in @13.UserAccount.onCodeUpgrade\\\hline
 & & assigned in @13.UserAccount.disableBorrowLock\\\hline
 & & used in @13.UserAccount.disableBorrowLock\\\hline
 & & assigned in @13.UserAccount.borrow\\\hline
 & & used in @13.UserAccount.borrow\\\hline
 & & used in @13.UserAccount.borrow\\\hline
bool & liquidationLock &  \\\hline
 & & used in @13.UserAccount.withdraw\\\hline
 & & used in @13.UserAccount.upgradeContractCode\\\hline
 & & assigned in @13.UserAccount.updateUserAccountHealth\\\hline
 & & used in @13.UserAccount.updateUserAccountHealth\\\hline
 & & assigned in @13.UserAccount.onCodeUpgrade\\\hline
 & & used in @13.UserAccount.onCodeUpgrade\\\hline
 & & used in @13.UserAccount.borrow\\\hline
address & userAccountManager &  \\\hline
 & & used in @13.UserAccount.withdraw\\\hline
 & & used in @13.UserAccount.upgradeContractCode\\\hline
 & & used in @13.UserAccount.updateUserAccountHealth\\\hline
 & & used in @13.UserAccount.sendRepayInfo\\\hline
 & & used in @13.UserAccount.requestWithdrawInfo\\\hline
 & & used in @13.UserAccount.requestLiquidationInformation\\\hline
 & & used in @13.UserAccount.removeMarket\\\hline
 & & assigned in @13.UserAccount.onCodeUpgrade\\\hline
 & & used in @13.UserAccount.onCodeUpgrade\\\hline
 & & used in @13.UserAccount.liquidateVTokens\\\hline
 & & used in @13.UserAccount.grantVTokens\\\hline
 & & used in @13.UserAccount.disableBorrowLock\\\hline
 & & used in @13.UserAccount.borrowUpdateIndexes\\\hline
 & & used in @13.UserAccount.borrow\\\hline
 & & used in @13.UserAccount.\_{}checkUserAccountHealth\\\hline
 & & assigned in @13.UserAccount.:constructor\\\hline
 & & used in @13.UserAccount.:constructor\\\hline
uint32 & contractCodeVersion &  \\\hline
 & & assigned in @13.UserAccount.onCodeUpgrade\\\hline
 & & used in @13.UserAccount.onCodeUpgrade\\\hline
fraction & accountHealth &  \\\hline
 & & used in @13.UserAccount.upgradeContractCode\\\hline
 & & used in @13.UserAccount.updateUserAccountHealth\\\hline
 & & used in @13.UserAccount.updateUserAccountHealth\\\hline
 & & used in @13.UserAccount.updateUserAccountHealth\\\hline
 & & used in @13.UserAccount.updateUserAccountHealth\\\hline
 & & assigned in @13.UserAccount.updateUserAccountHealth\\\hline
 & & used in @13.UserAccount.updateUserAccountHealth\\\hline
 & & used in @13.UserAccount.requestWithdrawInfo\\\hline
 & & used in @13.UserAccount.requestWithdrawInfo\\\hline
 & & assigned in @13.UserAccount.onCodeUpgrade\\\hline
 & & used in @13.UserAccount.onCodeUpgrade\\\hline
 & & used in @13.UserAccount.borrow\\\hline
 & & used in @13.UserAccount.borrow\\\hline
mapping (uint32 =$>$ bool) & knownMarkets &  \\\hline
 & & used in @13.UserAccount.upgradeContractCode\\\hline
 & & assigned in @13.UserAccount.removeMarket\\\hline
 & & used in @13.UserAccount.removeMarket\\\hline
 & & assigned in @13.UserAccount.onCodeUpgrade\\\hline
 & & used in @13.UserAccount.onCodeUpgrade\\\hline
 & & used in @13.UserAccount.getKnownMarkets\\\hline
 & & assigned in @13.UserAccount.enterMarket\\\hline
 & & used in @13.UserAccount.enterMarket\\\hline
 & & used in @13.UserAccount.enterMarket\\\hline
mapping (uint32 =$>$ UserMarketInfo) & markets &  \\\hline
 & & assigned in @13.UserAccount.writeWithdrawInfo\\\hline
 & & used in @13.UserAccount.writeWithdrawInfo\\\hline
 & & assigned in @13.UserAccount.writeSupplyInfo\\\hline
 & & used in @13.UserAccount.writeSupplyInfo\\\hline
 & & assigned in @13.UserAccount.writeRepayInformation\\\hline
 & & used in @13.UserAccount.writeRepayInformation\\\hline
 & & assigned in @13.UserAccount.writeBorrowInformation\\\hline
 & & used in @13.UserAccount.writeBorrowInformation\\\hline
 & & used in @13.UserAccount.withdraw\\\hline
 & & used in @13.UserAccount.upgradeContractCode\\\hline
 & & used in @13.UserAccount.sendRepayInfo\\\hline
 & & assigned in @13.UserAccount.removeMarket\\\hline
 & & used in @13.UserAccount.removeMarket\\\hline
 & & assigned in @13.UserAccount.onCodeUpgrade\\\hline
 & & used in @13.UserAccount.onCodeUpgrade\\\hline
 & & assigned in @13.UserAccount.liquidateVTokens\\\hline
 & & used in @13.UserAccount.liquidateVTokens\\\hline
 & & assigned in @13.UserAccount.liquidateVTokens\\\hline
 & & used in @13.UserAccount.liquidateVTokens\\\hline
 & & assigned in @13.UserAccount.grantVTokens\\\hline
 & & used in @13.UserAccount.grantVTokens\\\hline
 & & used in @13.UserAccount.getMarketInfo\\\hline
 & & used in @13.UserAccount.getAllMarketsInfo\\\hline
 & & assigned in @13.UserAccount.enterMarket\\\hline
 & & used in @13.UserAccount.enterMarket\\\hline
 & & assigned in @13.UserAccount.enterMarket\\\hline
 & & used in @13.UserAccount.enterMarket\\\hline
 & & assigned in @13.UserAccount.enterMarket\\\hline
 & & used in @13.UserAccount.enterMarket\\\hline
 & & assigned in @13.UserAccount.\_{}updateMarketInfo\\\hline
 & & used in @13.UserAccount.\_{}updateMarketInfo\\\hline
 & & used in @13.UserAccount.\_{}updateMarketInfo\\\hline
 & & used in @13.UserAccount.\_{}updateMarketInfo\\\hline
 & & used in @13.UserAccount.\_{}getBorrowSupplyInfo\\\hline
\end{tabular}
\fi


\begin{lstlisting}[firstnumber=20]
    bool public borrowLock;
\end{lstlisting}

\begin{lstlisting}[firstnumber=21]
    bool public liquidationLock;
\end{lstlisting}

\begin{lstlisting}[firstnumber=26]
    address public userAccountManager;
\end{lstlisting}

\begin{lstlisting}[firstnumber=29]
    uint32 public contractCodeVersion;
\end{lstlisting}

\begin{lstlisting}[firstnumber=31]
    fraction public accountHealth;
\end{lstlisting}

\begin{lstlisting}[firstnumber=33]
    mapping(uint32 => bool) knownMarkets;
\end{lstlisting}

\begin{lstlisting}[firstnumber=34]
    mapping(uint32 => UserMarketInfo) markets;
\end{lstlisting}

\section{Modifier Definitions}


\subsection{Modifier onlyOwner}


\begin{lstlisting}[firstnumber=412]
    modifier onlyOwner() {
        require(msg.sender == owner);
        _;
    }
\end{lstlisting}

\subsection{Modifier onlyUserAccountManager}


\begin{lstlisting}[firstnumber=417]
    modifier onlyUserAccountManager() {
        require(msg.sender == userAccountManager);
        _;
    }
\end{lstlisting}

\subsection{Modifier onlySelf}


\begin{lstlisting}[firstnumber=422]
    modifier onlySelf() {
        require(msg.sender == address(this));
        _;
    }
\end{lstlisting}

\subsection{Modifier onlyExecutor}


\begin{lstlisting}[firstnumber=427]
    modifier onlyExecutor() {
        require(
            msg.sender == userAccountManager ||
            msg.sender == owner ||
            msg.sender == address(this)
        );
        _;
    }
\end{lstlisting}

\section{Constructor Definitions}


\subsection{Constructor}

\issueCritical{Constructor for UserAccount (fake)}{loren ipsum  loren ipsum  loren ipsum loren ipsum loren ipsum loren ipsum loren ipsum loren ipsum loren ipsum loren ipsum loren ipsum loren ipsum loren ipsum loren ipsum loren ipsum loren ipsum loren ipsum loren ipsum

loren ipsum loren ipsum loren ipsum loren ipsum loren ipsum loren ipsum
loren ipsum loren ipsum loren ipsum }
\noindent\begin{itemize}
\item TODO
\end{itemize}

\begin{lstlisting}[firstnumber=49]
    constructor() public { 
        tvm.accept();
        userAccountManager = msg.sender;
    }
\end{lstlisting}

\section{Public Method Definitions}


\subsection{Function abortLiquidation}

\noindent\begin{itemize}
\item TODO
\end{itemize}

\begin{lstlisting}[firstnumber=327]
    function abortLiquidation(address tonWallet, address tip3UserWallet, uint32 marketId, uint256 tokensProvided) external override onlyUserAccountManager {
        if (tokensProvided != 0) { 
            _checkUserAccountHealth(owner, _createTokenPayoutPayload(tonWallet, tip3UserWallet, marketId, tokensProvided));
        } else {
            _checkUserAccountHealth(owner, _createNoOpPayload());
        }
    }
\end{lstlisting}

\subsection{Function borrow}

\noindent\begin{itemize}
\item TODO
\end{itemize}

\begin{lstlisting}[firstnumber=169]
    function borrow(uint32 marketId, uint256 amountToBorrow, address userTip3Wallet) external override onlyOwner {
        tvm.rawReserve(msg.value, 2);
        if (
            (!borrowLock) &&
            (accountHealth.nom > accountHealth.denom) &&
            !liquidationLock
        ) {
            borrowLock = true;
            TvmBuilder tb;
            tb.store(owner);
            tb.store(userTip3Wallet);
            tb.store(amountToBorrow);
            IUAMUserAccount(userAccountManager).requestIndexUpdate{
                flag: MsgFlag.REMAINING_GAS
            }(owner, marketId, tb.toCell());
        } else {
            address(msg.sender).transfer({value: 0, flag: MsgFlag.REMAINING_GAS});
        }
    }
\end{lstlisting}

\subsection{Function borrowUpdateIndexes}

\noindent\begin{itemize}
\item TODO
\end{itemize}

\begin{lstlisting}[firstnumber=189]
    function borrowUpdateIndexes(uint32 marketId, mapping(uint32 => fraction) updatedIndexes, address userTip3Wallet, uint256 toBorrow) external override onlyUserAccountManager {
        tvm.rawReserve(msg.value, 2);

        _updateIndexes(updatedIndexes);

        mapping(uint32 => BorrowInfo) borrowInfo;
        mapping(uint32 => uint256) supplyInfo;

        (borrowInfo, supplyInfo) = _getBorrowSupplyInfo();

        IUAMUserAccount(userAccountManager).passBorrowInformation{
            flag: MsgFlag.REMAINING_GAS
        }(owner, userTip3Wallet, marketId, toBorrow, supplyInfo, borrowInfo);
    }
\end{lstlisting}

\subsection{Function checkUserAccountHealth}

\noindent\begin{itemize}
\item TODO
\end{itemize}

\begin{lstlisting}[firstnumber=249]
    function checkUserAccountHealth(address gasTo) external override onlyExecutor {
        tvm.rawReserve(msg.value, 2);
        TvmBuilder no_op;
        no_op.store(OperationCodes.NO_OP);

        _checkUserAccountHealth(gasTo, no_op.toCell());
    }
\end{lstlisting}

\subsection{Function disableBorrowLock}

\noindent\begin{itemize}
\item TODO
\end{itemize}

\begin{lstlisting}[firstnumber=379]
    function disableBorrowLock() external override onlyUserAccountManager {
        tvm.rawReserve(msg.value, 2);
        borrowLock = false;
        address(userAccountManager).transfer({value: 0, flag: MsgFlag.REMAINING_GAS});
    }
\end{lstlisting}

\subsection{Function enterMarket}

\noindent\begin{itemize}
\item TODO
\end{itemize}

\begin{lstlisting}[firstnumber=391]
    function enterMarket(uint32 marketId) external override onlyOwner {
        tvm.rawReserve(msg.value, 2);
        if (!knownMarkets[marketId]) {
            knownMarkets[marketId] = true;

            markets[marketId].exists = true;
            markets[marketId]._marketId = marketId;
            markets[marketId].suppliedTokens = 0;
        }
        address(owner).transfer({value: 0, flag: MsgFlag.REMAINING_GAS});
    }
\end{lstlisting}

\subsection{Function getAllMarketsInfo}

\noindent\begin{itemize}
\item TODO
\end{itemize}

\begin{lstlisting}[firstnumber=40]
    function getAllMarketsInfo() external override view responsible returns(mapping(uint32 => UserMarketInfo)) {
        return {flag: MsgFlag.REMAINING_GAS} markets;
    }
\end{lstlisting}

\subsection{Function getKnownMarkets}

\noindent\begin{itemize}
\item TODO
\end{itemize}

\begin{lstlisting}[firstnumber=36]
    function getKnownMarkets() external override view responsible returns(mapping(uint32 => bool)) {
        return {flag: MsgFlag.REMAINING_GAS} knownMarkets;
    }
\end{lstlisting}

\subsection{Function getMarketInfo}

\noindent\begin{itemize}
\item TODO
\end{itemize}

\begin{lstlisting}[firstnumber=44]
    function getMarketInfo(uint32 marketId) external override view responsible returns(UserMarketInfo) {
        return {flag: MsgFlag.REMAINING_GAS} markets[marketId];
    }
\end{lstlisting}

\subsection{Function getOwner}

\noindent\begin{itemize}
\item TODO
\end{itemize}

\begin{lstlisting}[firstnumber=105]
    function getOwner() external override responsible view returns(address) {
        return { value: 0, bounce: false, flag: MsgFlag.REMAINING_GAS } owner;
    }
\end{lstlisting}

\subsection{Function grantVTokens}

\noindent\begin{itemize}
\item TODO
\end{itemize}

\begin{lstlisting}[firstnumber=312]
    function grantVTokens(address tip3UserWallet, uint32 marketId, uint32 marketToLiquidate, uint256 tokensToSeize, uint256 tokensToReturn, uint256 tokensFromReserve) external override onlyUserAccountManager {
        markets[marketToLiquidate].suppliedTokens += tokensToSeize;
        if (tokensFromReserve != 0) {
            IUAMUserAccount(userAccountManager).returnAndSupply{
                flag: MsgFlag.REMAINING_GAS
            }(owner, tip3UserWallet, marketId, marketToLiquidate, tokensToReturn, tokensFromReserve);
        } else {
            if (tokensToReturn != 0) {
                _checkUserAccountHealth(owner, _createTokenPayoutPayload(owner, tip3UserWallet, marketId, tokensToReturn));
            } else {
                _checkUserAccountHealth(owner, _createNoOpPayload());
            }
        }
    }
\end{lstlisting}

\subsection{Function liquidateVTokens}

\noindent\begin{itemize}
\item TODO
\end{itemize}

\begin{lstlisting}[firstnumber=303]
    function liquidateVTokens(address tonWallet, address tip3UserWallet, uint32 marketId, uint32 marketToLiquidate, uint256 tokensToSeize, uint256 tokensToReturn, uint256 tokensFromReserve, BorrowInfo borrowInfo) external override onlyUserAccountManager {
        markets[marketToLiquidate].suppliedTokens -= tokensToSeize;
        markets[marketId].borrowInfo = borrowInfo;

        IUAMUserAccount(userAccountManager).grantVTokens{
            flag: MsgFlag.REMAINING_GAS
        }(tonWallet, owner, tip3UserWallet, marketId, marketToLiquidate, tokensToSeize, tokensToReturn, tokensFromReserve);
    }
\end{lstlisting}

\subsection{Function removeMarket}

\noindent\begin{itemize}
\item TODO
\end{itemize}

\begin{lstlisting}[firstnumber=283]
    function removeMarket(uint32 marketId) external override onlyUserAccountManager {
        tvm.rawReserve(msg.value, 2);
        delete markets[marketId];
        delete knownMarkets[marketId];
        address(userAccountManager).transfer({value: 0, flag: MsgFlag.REMAINING_GAS});
    }
\end{lstlisting}

\subsection{Function requestLiquidationInformation}

\noindent\begin{itemize}
\item TODO
\end{itemize}

\begin{lstlisting}[firstnumber=293]
    function requestLiquidationInformation(address tonWallet, address tip3UserWallet, uint32 marketId, uint32 marketToLiquidate, uint256 tokensProvided, mapping(uint32 => fraction) updatedIndexes) external override onlyUserAccountManager {
        _updateIndexes(updatedIndexes);

        (mapping(uint32 => BorrowInfo) borrowInfo, mapping(uint32 => uint256) supplyInfo) = _getBorrowSupplyInfo();

        IUAMUserAccount(userAccountManager).receiveLiquidationInformation{
            flag: MsgFlag.REMAINING_GAS
        }(tonWallet, owner, tip3UserWallet, marketId, marketToLiquidate, tokensProvided, supplyInfo, borrowInfo);
    }
\end{lstlisting}

\subsection{Function requestWithdrawInfo}

\noindent\begin{itemize}
\item TODO
\end{itemize}

\begin{lstlisting}[firstnumber=138]
    function requestWithdrawInfo(address userTip3Wallet, uint32 marketId, uint256 tokensToWithdraw, mapping(uint32 => fraction) updatedIndexes) external override onlyUserAccountManager {
        tvm.rawReserve(msg.value, 2);
        if (
            accountHealth.nom > accountHealth.denom
        ) {
            for ((uint32 marketId_, fraction index): updatedIndexes) {
                _updateMarketInfo(marketId_, index);
            }

            mapping(uint32 => BorrowInfo) borrowInfo;
            mapping(uint32 => uint256) supplyInfo;

            (borrowInfo, supplyInfo) = _getBorrowSupplyInfo();

            IUAMUserAccount(userAccountManager).receiveWithdrawInfo{
                flag: MsgFlag.REMAINING_GAS
            }(owner, userTip3Wallet, tokensToWithdraw, marketId, supplyInfo, borrowInfo);
        } else {
            address(owner).transfer({value: 0, flag: MsgFlag.REMAINING_GAS});
        }
    }
\end{lstlisting}

\subsection{Function sendRepayInfo}

\noindent\begin{itemize}
\item TODO
\end{itemize}

\begin{lstlisting}[firstnumber=223]
    function sendRepayInfo(address userTip3Wallet, uint32 marketId, uint256 tokensForRepay, mapping(uint32 => fraction) updatedIndexes) external override onlyUserAccountManager {
        tvm.rawReserve(msg.value, 2);
        for ((uint32 marketId_, fraction index): updatedIndexes) {
            _updateMarketInfo(marketId_, index);
        }

        IUAMUserAccount(userAccountManager).receiveRepayInfo{
            flag: MsgFlag.REMAINING_GAS
        }(owner, userTip3Wallet, tokensForRepay, marketId, markets[marketId].borrowInfo);
    }
\end{lstlisting}

\subsection{Function updateUserAccountHealth}

\noindent\begin{itemize}
\item TODO
\end{itemize}

\begin{lstlisting}[firstnumber=266]
    function updateUserAccountHealth(address gasTo, fraction _accountHealth, mapping(uint32 => fraction) updatedIndexes, TvmCell dataToTransfer) external override onlyUserAccountManager {
        accountHealth = _accountHealth;
        liquidationLock = accountHealth.denom > accountHealth.nom;
        borrowLock = accountHealth.denom > accountHealth.nom;
        _updateIndexes(updatedIndexes);
        TvmSlice ts = dataToTransfer.toSlice();
        (uint8 operation) = ts.decode(uint8);
        if (operation == OperationCodes.REQUEST_TOKEN_PAYOUT) {
            (address tonWallet, address userTip3Wallet, uint32 marketId, uint256 tokensToPayout) = ts.decode(address, address, uint32, uint256);
            IUAMUserAccount(userAccountManager).requestTokenPayout{
                flag: MsgFlag.REMAINING_GAS
            }(tonWallet, userTip3Wallet, marketId, tokensToPayout);
        } else {
            address(gasTo).transfer({value: 0, flag: MsgFlag.REMAINING_GAS});
        }
    }
\end{lstlisting}

\subsection{Function upgradeContractCode}

\noindent\begin{itemize}
\item TODO
\end{itemize}

\begin{lstlisting}[firstnumber=54]
    function upgradeContractCode(TvmCell code, TvmCell updateParams, uint32 codeVersion) override external onlyUserAccountManager {
        require(!borrowLock);
        tvm.accept();

        bool _borrowLock = borrowLock;
        bool _liquidationLock = liquidationLock;
        address _owner = owner;
        address _userAccountManager = userAccountManager;
        mapping (uint32 => bool) _knownMarkets = knownMarkets;
        mapping (uint32 => UserMarketInfo) _markets = markets;
        fraction _accountHealth = accountHealth;

        tvm.setcode(code);
        tvm.setCurrentCode(code);

        onCodeUpgrade(
            _borrowLock,
            _liquidationLock,
            _owner,
            _userAccountManager,
            _knownMarkets,
            _markets,
            _accountHealth,
            updateParams,
            codeVersion
        );
    }
\end{lstlisting}

\subsection{Function withdraw}

\noindent\begin{itemize}
\item TODO
\end{itemize}

\begin{lstlisting}[firstnumber=123]
    function withdraw(address userTip3Wallet, uint32 marketId, uint256 tokensToWithdraw) external override view onlyOwner {
        if (
            !liquidationLock &&
            tokensToWithdraw <= markets[marketId].suppliedTokens
        ) {
            tvm.rawReserve(msg.value, 2);
            
            IUAMUserAccount(userAccountManager).requestWithdraw{
                flag: MsgFlag.REMAINING_GAS
            }(owner, userTip3Wallet, marketId, tokensToWithdraw);
        } else {
            address(owner).transfer({value: 0, flag: MsgFlag.REMAINING_GAS});
        }
    }
\end{lstlisting}

\subsection{Function withdrawExtraTons}

\noindent\begin{itemize}
\item TODO
\end{itemize}

\begin{lstlisting}[firstnumber=406]
    function withdrawExtraTons() external override view onlyOwner {
        address(owner).transfer({ value: 0, bounce: false, flag: MsgFlag.ALL_NOT_RESERVED });
    }
\end{lstlisting}

\subsection{Function writeBorrowInformation}

\noindent\begin{itemize}
\item TODO
\end{itemize}

\begin{lstlisting}[firstnumber=204]
    function writeBorrowInformation(uint32 marketId, uint256 toBorrow, address userTip3Wallet, fraction marketIndex) external override onlyUserAccountManager {
        tvm.rawReserve(msg.value, 2);
        if (toBorrow > 0) {
            _updateMarketInfo(marketId, marketIndex);
            markets[marketId].borrowInfo.tokensBorrowed += toBorrow;
        }

        borrowLock = false;

        if (toBorrow > 0) {
            _checkUserAccountHealth(owner, _createTokenPayoutPayload(owner, userTip3Wallet, marketId, toBorrow));
        } else {
            _checkUserAccountHealth(owner, _createNoOpPayload());
        }
    }
\end{lstlisting}

\subsection{Function writeRepayInformation}

\noindent\begin{itemize}
\item TODO
\end{itemize}

\begin{lstlisting}[firstnumber=234]
    function writeRepayInformation(address userTip3Wallet, uint32 marketId, uint256 tokensToReturn, BorrowInfo bi) external override onlyUserAccountManager {
        tvm.rawReserve(msg.value, 2);

        markets[marketId].borrowInfo = bi;
        
        if (tokensToReturn != 0) { 
            _checkUserAccountHealth(owner, _createTokenPayoutPayload(owner, userTip3Wallet, marketId, tokensToReturn));
        } else {
            _checkUserAccountHealth(owner, _createNoOpPayload());
        }
    }
\end{lstlisting}

\subsection{Function writeSupplyInfo}

\noindent\begin{itemize}
\item TODO
\end{itemize}

\begin{lstlisting}[firstnumber=112]
    function writeSupplyInfo(uint32 marketId, uint256 tokensToSupply, fraction index) external override onlyUserAccountManager {
        tvm.rawReserve(msg.value, 2);
        markets[marketId].suppliedTokens += tokensToSupply;
        _updateMarketInfo(marketId, index);

        _checkUserAccountHealth(owner, _createNoOpPayload());
    }
\end{lstlisting}

\subsection{Function writeWithdrawInfo}

\noindent\begin{itemize}
\item TODO
\end{itemize}

\begin{lstlisting}[firstnumber=160]
    function writeWithdrawInfo(address userTip3Wallet, uint32 marketId, uint256 tokensToWithdraw, uint256 tokensToSend) external override onlyUserAccountManager{
        tvm.rawReserve(msg.value, 2);
        markets[marketId].suppliedTokens -= tokensToWithdraw;
        _checkUserAccountHealth(owner, _createTokenPayoutPayload(owner, userTip3Wallet, marketId, tokensToSend));
    }
\end{lstlisting}

\section{Internal Method Definitions}


\subsection{Function \_{}checkUserAccountHealth}

\noindent\begin{itemize}
\item TODO
\end{itemize}

\begin{lstlisting}[firstnumber=257]
    function _checkUserAccountHealth(address gasTo, TvmCell dataToTransfer) internal view {
        mapping(uint32 => uint256) supplyInfo;
        mapping(uint32 => BorrowInfo) borrowInfo;
        (borrowInfo, supplyInfo) = _getBorrowSupplyInfo();
        IUAMUserAccount(userAccountManager).calculateUserAccountHealth{
            flag: MsgFlag.REMAINING_GAS
        }(owner, gasTo, supplyInfo, borrowInfo, dataToTransfer);
    }
\end{lstlisting}

\subsection{Function \_{}createNoOpPayload}

\noindent\begin{itemize}
\item TODO
\end{itemize}

\begin{lstlisting}[firstnumber=373]
    function _createNoOpPayload() internal pure returns (TvmCell) {
        TvmBuilder no_op;
        no_op.store(OperationCodes.NO_OP);
        return no_op.toCell();
    }
\end{lstlisting}

\subsection{Function \_{}createTokenPayoutPayload}

\noindent\begin{itemize}
\item TODO
\end{itemize}

\begin{lstlisting}[firstnumber=363]
    function _createTokenPayoutPayload(address tonWallet, address userTip3Wallet, uint32 marketId, uint256 tokensToSend) internal pure returns (TvmCell) {
        TvmBuilder op;
        op.store(OperationCodes.REQUEST_TOKEN_PAYOUT);
        op.store(tonWallet);
        op.store(userTip3Wallet);
        op.store(marketId);
        op.store(tokensToSend);
        return op.toCell();
    }
\end{lstlisting}

\subsection{Function \_{}getBorrowSupplyInfo}

\noindent\begin{itemize}
\item TODO
\end{itemize}

\begin{lstlisting}[firstnumber=356]
    function _getBorrowSupplyInfo() internal view returns(mapping(uint32 => BorrowInfo) borrowInfo, mapping(uint32 => uint256) supplyInfo) {
        for ((uint32 marketId, UserMarketInfo umi) : markets) {
            supplyInfo[marketId] = umi.suppliedTokens;
            borrowInfo[marketId] = umi.borrowInfo;
        }
    }
\end{lstlisting}

\subsection{Function \_{}updateIndexes}

\noindent\begin{itemize}
\item TODO
\end{itemize}

\begin{lstlisting}[firstnumber=338]
    function _updateIndexes(mapping(uint32 => fraction) updatedIndexes) internal {
        for ((uint32 marketId_, fraction index): updatedIndexes) {
            _updateMarketInfo(marketId_, index);
        }
    }
\end{lstlisting}

\subsection{Function \_{}updateMarketInfo}

\noindent\begin{itemize}
\item TODO
\end{itemize}

\begin{lstlisting}[firstnumber=344]
    function _updateMarketInfo(uint32 marketId, fraction index) internal {
        fraction tmpf;
        BorrowInfo bi = markets[marketId].borrowInfo;
        if (markets[marketId].borrowInfo.tokensBorrowed != 0) {
            tmpf = bi.tokensBorrowed.numFMul(index);
            tmpf = tmpf.fDiv(bi.index);
        } else {
            tmpf = fraction(0, 1);
        }
        markets[marketId].borrowInfo = BorrowInfo(tmpf.toNum(), index);
    }
\end{lstlisting}

\subsection{Function onCodeUpgrade}

\noindent\begin{itemize}
\item TODO
\end{itemize}

\begin{lstlisting}[firstnumber=82]
    function onCodeUpgrade(
        bool _borrowLock,
        bool _liquidationLock,
        address _owner,
        address _userAccountManager,
        mapping(uint32 => bool) _knownMarkets,
        mapping(uint32 => UserMarketInfo) _markets,
        fraction _accountHealth,
        TvmCell,
        uint32 codeVersion
    ) private {
        tvm.resetStorage();
        borrowLock = _borrowLock;
        liquidationLock = _liquidationLock;
        owner = _owner;
        userAccountManager = _userAccountManager;
        knownMarkets = _knownMarkets;
        markets = _markets;
        accountHealth = _accountHealth;

        contractCodeVersion = codeVersion;
    }
\end{lstlisting}
\paragraph{Some functions inherited by using}


\section{Module "UserAccountManager.sol"}


\subsection{Pragmas}


\noindent\begin{tabular}{|l|l|p{5cm}|}\hline
ton & -solidity $>$= 0.43.0 &\\\hline
AbiHeader &  pubkey &\\\hline
AbiHeader &  expire &\\\hline
AbiHeader &  time &\\\hline
\end{tabular}


\subsection{Imports}


\noindent\begin{tabular}{|l|l|p{5cm}|}\hline
./interfaces/IUserAccountManager.sol &\\\hline
./interfaces/IUAMUserAccount.sol &\\\hline
./interfaces/IUAMMarket.sol &\\\hline
./libraries/UserAccountErrorCodes.sol &\\\hline
./libraries/CostConstants.sol &\\\hline
../Market/interfaces/IMarketInterfaces.sol &\\\hline
../WalletController/libraries/OperationCodes.sol &\\\hline
../utils/interfaces/IUpgradableContract.sol &\\\hline
../utils/libraries/MsgFlag.sol &\\\hline
./UserAccount.sol &\\\hline
../ModulesForMarket/interfaces/IModule.sol &\\\hline
\end{tabular}


\subsection{Contract Definitions}

\begin{itemize}
\item UserAccountManager
\end{itemize}


\section{Contract WalletController}

In file {\tt WalletController.sol}


\subsection{Modifier onlyMarket}

\begin{lstlisting}[firstnumber=295]
    modifier onlyMarket() {
        require(msg.sender == marketAddress, WalletControllerErrorCodes.ERROR_MSG_SENDER_IS_NOT_MARKET);
        _;
    }
\end{lstlisting}

\noindent\begin{itemize}
  \item \unusedModifier{WalletController.onlyMarket}
\end{itemize}






\section{Abstract Contract IRoles}

In file {\tt IRoles.sol}




\bigskip

\section{Contract Platform}

In file {\tt Platform.sol}.

\subsection{Function initializeContract}

\begin{lstlisting}[firstnumber=18]
  function initializeContract(TvmCell contractCode, TvmCell params) private {
      tvm.accept();
      TvmBuilder builder;

      builder.store(root);
      builder.store(platformType);

      builder.store(platformCode); // ref 1
      builder.store(initialData);  // ref 2
      builder.store(params);       // ref 3

      tvm.setcode(contractCode);
      tvm.setCurrentCode(contractCode);

      onCodeUpgrade(builder.toCell());
  }
\end{lstlisting}

\issueMajor{{\tt tvm.accept} in a private function}{Private and internal functions should not have a {\tt tvm.accept}, especially without checks.}



\chapter{Contract RootTokenContract}

\minitoc

\section{Overview}


In file {\tt RootTokenContract.sol}

\section{Contract Inheritance}


\noindent\begin{tabular}{|l|p{5cm}|}\hline
IRootTokenContract & \\\hline
IBurnableTokenRootContract & \\\hline
IBurnableByRootTokenRootContract & \\\hline
IPausable & \\\hline
ITransferOwner & \\\hline
ISendSurplusGas & \\\hline
IVersioned & \\\hline
\end{tabular}


\section{Static Variable Definitions}


\ifsoltables
\noindent\begin{tabular}{|l|l|p{5cm}|}\hline
uint256 & \_{}randomNonce &  \\\hline
bytes & name &  \\\hline
 & & used in @17.RootTokenContract.getDetails\\\hline
bytes & symbol &  \\\hline
 & & used in @17.RootTokenContract.getDetails\\\hline
uint8 & decimals &  \\\hline
 & & used in @17.RootTokenContract.getDetails\\\hline
TvmCell & wallet\_{}code &  \\\hline
 & & used in @17.RootTokenContract.getWalletCode\\\hline
 & & used in @17.RootTokenContract.getExpectedWalletAddress\\\hline
 & & used in @17.RootTokenContract.getExpectedWalletAddress\\\hline
 & & used in @17.RootTokenContract.deployWallet\\\hline
 & & used in @17.RootTokenContract.deployWallet\\\hline
 & & used in @17.RootTokenContract.deployEmptyWallet\\\hline
\end{tabular}
\fi


\begin{lstlisting}[firstnumber=29]
    uint256 static _randomNonce;
\end{lstlisting}

\begin{lstlisting}[firstnumber=31]
    bytes public static name;
\end{lstlisting}

\begin{lstlisting}[firstnumber=32]
    bytes public static symbol;
\end{lstlisting}

\begin{lstlisting}[firstnumber=33]
    uint8 public static decimals;
\end{lstlisting}

\begin{lstlisting}[firstnumber=35]
    TvmCell static wallet_code;
\end{lstlisting}

\section{Variable Definitions}


\ifsoltables
\noindent\begin{tabular}{|l|l|p{5cm}|}\hline
uint128 & total\_{}supply &  \\\hline
 & & assigned in @17.RootTokenContract.tokensBurned\\\hline
 & & used in @17.RootTokenContract.tokensBurned\\\hline
 & & assigned in @17.RootTokenContract.mint\\\hline
 & & used in @17.RootTokenContract.mint\\\hline
 & & used in @17.RootTokenContract.getTotalSupply\\\hline
 & & used in @17.RootTokenContract.getDetails\\\hline
 & & assigned in @17.RootTokenContract.deployWallet\\\hline
 & & used in @17.RootTokenContract.deployWallet\\\hline
 & & assigned in @17.RootTokenContract.:onBounce\\\hline
 & & used in @17.RootTokenContract.:onBounce\\\hline
 & & assigned in @17.RootTokenContract.:constructor\\\hline
 & & used in @17.RootTokenContract.:constructor\\\hline
uint256 & root\_{}public\_{}key &  \\\hline
 & & assigned in @17.RootTokenContract.transferOwner\\\hline
 & & used in @17.RootTokenContract.transferOwner\\\hline
 & & used in @17.RootTokenContract.isExternalOwner\\\hline
 & & used in @17.RootTokenContract.isExternalOwner\\\hline
 & & used in @17.RootTokenContract.getDetails\\\hline
 & & assigned in @17.RootTokenContract.:constructor\\\hline
 & & used in @17.RootTokenContract.:constructor\\\hline
address & root\_{}owner\_{}address &  \\\hline
 & & assigned in @17.RootTokenContract.transferOwner\\\hline
 & & used in @17.RootTokenContract.transferOwner\\\hline
 & & used in @17.RootTokenContract.isInternalOwner\\\hline
 & & used in @17.RootTokenContract.isInternalOwner\\\hline
 & & used in @17.RootTokenContract.getDetails\\\hline
 & & used in @17.RootTokenContract.deployWallet\\\hline
 & & used in @17.RootTokenContract.deployWallet\\\hline
 & & assigned in @17.RootTokenContract.:constructor\\\hline
 & & used in @17.RootTokenContract.:constructor\\\hline
uint128 & start\_{}gas\_{}balance &  \\\hline
 & & used in @17.RootTokenContract.sendSurplusGas\\\hline
 & & used in @17.RootTokenContract.deployWallet\\\hline
 & & assigned in @17.RootTokenContract.:constructor\\\hline
 & & used in @17.RootTokenContract.:constructor\\\hline
bool & paused &  \\\hline
 & & used in @17.RootTokenContract.tokensBurned\\\hline
 & & assigned in @17.RootTokenContract.setPaused\\\hline
 & & used in @17.RootTokenContract.setPaused\\\hline
 & & used in @17.RootTokenContract.sendPausedCallbackTo\\\hline
 & & assigned in @17.RootTokenContract.:constructor\\\hline
 & & used in @17.RootTokenContract.:constructor\\\hline
\end{tabular}
\fi


\begin{lstlisting}[firstnumber=37]
    uint128 total_supply;
\end{lstlisting}

\begin{lstlisting}[firstnumber=39]
    uint256 root_public_key;
\end{lstlisting}

\begin{lstlisting}[firstnumber=40]
    address root_owner_address;
\end{lstlisting}

\begin{lstlisting}[firstnumber=41]
    uint128 public start_gas_balance;
\end{lstlisting}

\begin{lstlisting}[firstnumber=43]
    bool public paused;
\end{lstlisting}

\section{Modifier Definitions}


\subsection{Modifier onlyOwner}


\begin{lstlisting}[firstnumber=459]
    modifier onlyOwner() {
        require(isOwner(), RootTokenContractErrors.error_message_sender_is_not_my_owner);
        _;
    }
\end{lstlisting}

\subsection{Modifier onlyInternalOwner}


\begin{lstlisting}[firstnumber=464]
    modifier onlyInternalOwner() {
        require(isInternalOwner(), RootTokenContractErrors.error_message_sender_is_not_my_owner);
        _;
    }
\end{lstlisting}

\section{Constructor Definitions}


\subsection{Constructor}

\issueCritical{Constructor for RootTokenContract (fake)}{loren ipsum  loren ipsum  loren ipsum loren ipsum loren ipsum loren ipsum loren ipsum loren ipsum loren ipsum loren ipsum loren ipsum loren ipsum loren ipsum loren ipsum loren ipsum loren ipsum loren ipsum loren ipsum

loren ipsum loren ipsum loren ipsum loren ipsum loren ipsum loren ipsum
loren ipsum loren ipsum loren ipsum }
\noindent\begin{itemize}
\item TODO
\end{itemize}

\begin{lstlisting}[firstnumber=49]
    constructor(uint256 root_public_key_, address root_owner_address_) public {
        require((root_public_key_ != 0 && root_owner_address_.value == 0) ||
                (root_public_key_ == 0 && root_owner_address_.value != 0),
                RootTokenContractErrors.error_define_public_key_or_owner_address);
        tvm.accept();

        root_public_key = root_public_key_;
        root_owner_address = root_owner_address_;

        total_supply = 0;
        paused = false;

        start_gas_balance = address(this).balance;
    }
\end{lstlisting}

\section{Public Method Definitions}


\subsection{Fallback function}

\noindent\begin{itemize}
\item TODO
\end{itemize}

\begin{lstlisting}[firstnumber=524]
    fallback() external {
    }
\end{lstlisting}

\subsection{OnBounce function}

\noindent\begin{itemize}
\item TODO
\end{itemize}

\begin{lstlisting}[firstnumber=515]
    onBounce(TvmSlice slice) external {
        tvm.accept();
        uint32 functionId = slice.decode(uint32);
        if (functionId == tvm.functionId(ITONTokenWallet.accept)) {
            uint128 latest_bounced_tokens = slice.decode(uint128);
            total_supply -= latest_bounced_tokens;
        }
    }
\end{lstlisting}

\subsection{Function deployEmptyWallet}

\noindent\begin{itemize}
\item TODO
\end{itemize}

\begin{lstlisting}[firstnumber=238]
    function deployEmptyWallet(
        uint128 deploy_grams,
        uint256 wallet_public_key_,
        address owner_address_,
        address gas_back_address
    )
        override
        external
    returns (
        address
    ) {
        require((owner_address_.value != 0 && wallet_public_key_ == 0) ||
                (owner_address_.value == 0 && wallet_public_key_ != 0),
                RootTokenContractErrors.error_define_public_key_or_owner_address);

        tvm.rawReserve(address(this).balance - msg.value, 2);

        address wallet = new TONTokenWallet{
            value: deploy_grams,
            flag: 1,
            code: wallet_code,
            pubkey: wallet_public_key_,
            varInit: {
                root_address: address(this),
                code: wallet_code,
                wallet_public_key: wallet_public_key_,
                owner_address: owner_address_
            }
        }();

        if (gas_back_address.value != 0) {
            gas_back_address.transfer({ value: 0, flag: 128 });
        } else {
            msg.sender.transfer({ value: 0, flag: 128 });
        }

        return wallet;
    }
\end{lstlisting}

\subsection{Function deployWallet}

\noindent\begin{itemize}
\item TODO
\end{itemize}

\begin{lstlisting}[firstnumber=165]
    function deployWallet(
        uint128 tokens,
        uint128 deploy_grams,
        uint256 wallet_public_key_,
        address owner_address_,
        address gas_back_address
    )
        override
        external
        onlyOwner
    returns(
        address
    ) {
        require(tokens >= 0);
        require((owner_address_.value != 0 && wallet_public_key_ == 0) ||
                (owner_address_.value == 0 && wallet_public_key_ != 0),
                RootTokenContractErrors.error_define_public_key_or_owner_address);

        if(root_owner_address.value == 0) {
            tvm.accept();
        } else {
            tvm.rawReserve(math.max(start_gas_balance, address(this).balance - msg.value), 2);
        }

        TvmCell stateInit = tvm.buildStateInit({
            contr: TONTokenWallet,
            varInit: {
                root_address: address(this),
                code: wallet_code,
                wallet_public_key: wallet_public_key_,
                owner_address: owner_address_
            },
            pubkey: wallet_public_key_,
            code: wallet_code
        });

        address wallet;

        if(deploy_grams > 0) {
            wallet = new TONTokenWallet{
                stateInit: stateInit,
                value: deploy_grams,
                wid: address(this).wid,
                flag: 1
            }();
        } else {
            wallet = address(tvm.hash(stateInit));
        }

        ITONTokenWallet(wallet).accept(tokens);

        total_supply += tokens;

        if (root_owner_address.value != 0) {
            if (gas_back_address.value != 0) {
                gas_back_address.transfer({ value: 0, flag: 128 });
            } else {
                msg.sender.transfer({ value: 0, flag: 128 });
            }
        }

        return wallet;
    }
\end{lstlisting}

\subsection{Function getDetails}

\noindent\begin{itemize}
\item TODO
\end{itemize}

\begin{lstlisting}[firstnumber=78]
    function getDetails() override external view responsible returns (IRootTokenContractDetails) {
        return { value: 0, bounce: false, flag: 64 } IRootTokenContractDetails(
            name,
            symbol,
            decimals,
            root_public_key,
            root_owner_address,
            total_supply
        );
    }
\end{lstlisting}

\subsection{Function getTotalSupply}

\noindent\begin{itemize}
\item TODO
\end{itemize}

\begin{lstlisting}[firstnumber=93]
    function getTotalSupply() override external view responsible returns (uint128) {
        return { value: 0, bounce: false, flag: 64 } total_supply;
    }
\end{lstlisting}

\subsection{Function getVersion}

\noindent\begin{itemize}
\item TODO
\end{itemize}

\begin{lstlisting}[firstnumber=64]
    function getVersion() override external pure responsible returns (uint32) {
        return 4;
    }
\end{lstlisting}

\subsection{Function getWalletAddress}

\noindent\begin{itemize}
\item TODO
\end{itemize}

\begin{lstlisting}[firstnumber=112]
    function getWalletAddress(
        uint256 wallet_public_key_,
        address owner_address_
    )
        override
        external
        view
        responsible
    returns (
        address
    ) {
        require((owner_address_.value != 0 && wallet_public_key_ == 0) ||
                (owner_address_.value == 0 && wallet_public_key_ != 0),
                RootTokenContractErrors.error_define_public_key_or_owner_address);
        return { value: 0, bounce: false, flag: 64 } getExpectedWalletAddress(wallet_public_key_, owner_address_);
    }
\end{lstlisting}

\subsection{Function getWalletCode}

\noindent\begin{itemize}
\item TODO
\end{itemize}

\begin{lstlisting}[firstnumber=101]
    function getWalletCode() override external view responsible returns (TvmCell) {
        return { value: 0, bounce: false, flag: 64 } wallet_code;
    }
\end{lstlisting}

\subsection{Function mint}

\noindent\begin{itemize}
\item TODO
\end{itemize}

\begin{lstlisting}[firstnumber=283]
    function mint(
        uint128 tokens,
        address to
    )
        override
        external
        onlyOwner
    {
        tvm.accept();

        ITONTokenWallet(to).accept(tokens);

        total_supply += tokens;
    }
\end{lstlisting}

\subsection{Function proxyBurn}

\noindent\begin{itemize}
\item TODO
\end{itemize}

\begin{lstlisting}[firstnumber=308]
    function proxyBurn(
        uint128 tokens,
        address sender_address,
        address send_gas_to,
        address callback_address,
        TvmCell callback_payload
    )
        override
        external
        onlyInternalOwner
    {
        tvm.rawReserve(address(this).balance - msg.value, 2);

        address send_gas_to_ = send_gas_to;
        address expectedWalletAddress = getExpectedWalletAddress(0, sender_address);

        if (send_gas_to.value == 0) {
            send_gas_to_ = sender_address;
        }

        IBurnableByRootTokenWallet(expectedWalletAddress).burnByRoot{value: 0, flag: 128}(
            tokens,
            send_gas_to_,
            callback_address,
            callback_payload
        );
    }
\end{lstlisting}

\subsection{Function sendExpectedWalletAddress}

\noindent\begin{itemize}
\item TODO
\end{itemize}

\begin{lstlisting}[firstnumber=135]
    function sendExpectedWalletAddress(
        uint256 wallet_public_key_,
        address owner_address_,
        address to
    )
        override
        external
    {
        tvm.rawReserve(address(this).balance - msg.value, 2);

        address wallet = getExpectedWalletAddress(wallet_public_key_, owner_address_);
        IExpectedWalletAddressCallback(to).expectedWalletAddressCallback{value: 0, flag: 128}(
            wallet,
            wallet_public_key_,
            owner_address_
        );
    }
\end{lstlisting}

\subsection{Function sendPausedCallbackTo}

\noindent\begin{itemize}
\item TODO
\end{itemize}

\begin{lstlisting}[firstnumber=424]
    function sendPausedCallbackTo(
        uint64 callback_id,
        address callback_addr
    )
        override
        external
    {
        tvm.rawReserve(address(this).balance - msg.value, 2);
        IPausedCallback(callback_addr).pausedCallback{ value: 0, flag: 128 }(callback_id, paused);
    }
\end{lstlisting}

\subsection{Function sendSurplusGas}

\noindent\begin{itemize}
\item TODO
\end{itemize}

\begin{lstlisting}[firstnumber=387]
    function sendSurplusGas(
        address to
    )
        override
        external
        onlyInternalOwner
    {
        tvm.rawReserve(start_gas_balance, 2);
        IReceiveSurplusGas(to).receiveSurplusGas{ value: 0, flag: 128 }();
    }
\end{lstlisting}

\subsection{Function setPaused}

\noindent\begin{itemize}
\item TODO
\end{itemize}

\begin{lstlisting}[firstnumber=408]
    function setPaused(
        bool value
    )
        override
        external
        onlyOwner
    {
        tvm.accept();
        paused = value;
    }
\end{lstlisting}

\subsection{Function tokensBurned}

\noindent\begin{itemize}
\item TODO
\end{itemize}

\begin{lstlisting}[firstnumber=348]
    function tokensBurned(
        uint128 tokens,
        uint256 sender_public_key,
        address sender_address,
        address send_gas_to,
        address callback_address,
        TvmCell callback_payload
    ) override external {

        require(!paused, RootTokenContractErrors.error_paused);

        address expectedWalletAddress = getExpectedWalletAddress(sender_public_key, sender_address);

        require(msg.sender == expectedWalletAddress, RootTokenContractErrors.error_message_sender_is_not_good_wallet);

        tvm.rawReserve(address(this).balance - msg.value, 2);

        total_supply -= tokens;

        if (callback_address.value == 0) {
            send_gas_to.transfer({ value: 0, flag: 128 });
        } else {
            IBurnTokensCallback(callback_address).burnCallback{value: 0, flag: 128}(
                tokens,
                callback_payload,
                sender_public_key,
                sender_address,
                expectedWalletAddress,
                send_gas_to
            );
        }

    }
\end{lstlisting}

\subsection{Function transferOwner}

\noindent\begin{itemize}
\item TODO
\end{itemize}

\begin{lstlisting}[firstnumber=441]
    function transferOwner(
        uint256 root_public_key_,
        address root_owner_address_
    )
        override
        external
        onlyOwner
    {
        require((root_public_key_ != 0 && root_owner_address_.value == 0) ||
                (root_public_key_ == 0 && root_owner_address_.value != 0),
                RootTokenContractErrors.error_define_public_key_or_owner_address);
        tvm.accept();
        root_public_key = root_public_key_;
        root_owner_address = root_owner_address_;
    }
\end{lstlisting}

\section{Internal Method Definitions}


\subsection{Function getExpectedWalletAddress}

\noindent\begin{itemize}
\item TODO
\end{itemize}

\begin{lstlisting}[firstnumber=486]
    function getExpectedWalletAddress(
        uint256 wallet_public_key_,
        address owner_address_
    )
        private
        inline
        view
    returns (
        address
    ) {
        TvmCell stateInit = tvm.buildStateInit({
            contr: TONTokenWallet,
            varInit: {
                root_address: address(this),
                code: wallet_code,
                wallet_public_key: wallet_public_key_,
                owner_address: owner_address_
            },
            pubkey: wallet_public_key_,
            code: wallet_code
        });

        return address(tvm.hash(stateInit));
    }
\end{lstlisting}

\subsection{Function isExternalOwner}

\noindent\begin{itemize}
\item TODO
\end{itemize}

\begin{lstlisting}[firstnumber=477]
    function isExternalOwner() private inline view returns (bool) {
        return root_public_key != 0 && root_public_key == msg.pubkey();
    }
\end{lstlisting}

\subsection{Function isInternalOwner}

\noindent\begin{itemize}
\item TODO
\end{itemize}

\begin{lstlisting}[firstnumber=473]
    function isInternalOwner() private inline view returns (bool) {
        return root_owner_address.value != 0 && root_owner_address == msg.sender;
    }
\end{lstlisting}

\subsection{Function isOwner}

\noindent\begin{itemize}
\item TODO
\end{itemize}

\begin{lstlisting}[firstnumber=469]
    function isOwner() private inline view returns (bool) {
        return isInternalOwner() || isExternalOwner();
    }
\end{lstlisting}


\section{Module "TONTokenWallet.sol"}


\subsection{Pragmas}


\noindent\begin{tabular}{|l|l|p{5cm}|}\hline
ton & -solidity $>$= 0.39.0 &\\\hline
AbiHeader &  expire &\\\hline
AbiHeader &  pubkey &\\\hline
\end{tabular}


\subsection{Imports}


\noindent\begin{tabular}{|l|l|p{5cm}|}\hline
./interfaces/IDestroyable.sol &\\\hline
./interfaces/ITONTokenWallet.sol &\\\hline
./interfaces/IBurnableByOwnerTokenWallet.sol &\\\hline
./interfaces/IBurnableByRootTokenWallet.sol &\\\hline
./interfaces/IBurnableTokenRootContract.sol &\\\hline
./interfaces/ITokenWalletDeployedCallback.sol &\\\hline
./interfaces/ITokensReceivedCallback.sol &\\\hline
./interfaces/ITokensBouncedCallback.sol &\\\hline
./libraries/TONTokenWalletErrors.sol &\\\hline
./libraries/TONTokenWalletConstants.sol &\\\hline
./interfaces/IVersioned.sol &\\\hline
\end{tabular}


\subsection{Contract Definitions}

\begin{itemize}
\item TONTokenWallet
\end{itemize}




\bigskip

\section{Module "FloatingPointOperations.sol"}

\subsection{Struct fraction}

\begin{lstlisting}[firstnumber=3]
  struct fraction {
      uint256 nom;
      uint256 denom;
  }
\end{lstlisting}

\issueMinor{Unintuitive struct field name}{The name of the field {\tt nom} should be {\tt num} for ``numerator''.}

\bigskip

\section{Library FPO}

In file {\tt FloatingPointOperations.sol}

\subsection{Function eq}

\begin{lstlisting}[firstnumber=53]
    function eq(fraction a, fraction b) internal pure returns(bool) {
        return ((a.nom == b.nom) && (a.denom == b.denom));
    }
\end{lstlisting}

\issueMajor{Math error in {\tt FPO.eq}}{Comparing numerators and denominators when testing if fractions are equal is incorrect. $eq(\frac{a}{b}, \frac{a \times 2}{b \times 2})$ will return {\tt false} while it should return {\tt true}. The fractions need to be normalized before checking if they are equal.}

\bigskip

\subsection{Function simplify}

\begin{lstlisting}[firstnumber=69]
  function simplify(fraction a) internal pure returns(fraction) {
      // loosing ??? % of presicion at most
      if (a.nom / a.denom > 100e9) {
          return fraction(a.nom / a.denom, 1);
      } else {
          // using bitshift for simultaneos division
          // leaving up to 64 bits of information if nom & denom > 2^64
          if ( (a.nom >= bits224) && (a.denom >= bits224) ) {
              return fraction(a.nom / bits160, a.denom / bits160);
          }

          if ( (a.nom >= bits192) && (a.denom >= bits192) ) {
              return fraction(a.nom / bits128, a.denom / bits128);
          }

          if ( (a.nom >= bits160) && (a.denom >= bits160) ) {
              return fraction(a.nom / bits96, a.denom / bits96);
          }

          if ( (a.nom >= bits128) && (a.denom >= bits128) ) {
              return fraction(a.nom / bits64, a.denom / bits64);
          }

          if ( (a.nom >= bits96) && (a.denom >= bits96) ) {
              return fraction(a.nom / bits32, a.denom / bits32);
          }

          return a;
      }
  }
\end{lstlisting}

\issueMajor{Math issue in {\tt FPO.simplify}}{Dividing the numerator and denominator by their greatest common divisor might make it unnecessary to do the bitshift and avoid losing precision.}


\section{Library TvmCellOperations}

In file {\tt TvmCellOperations.sol}

\issueMinor{Unused functions}{All the functions in the file are unused.}

