


\noindent This chapter presents an audit of Everscalend's smart contracts and lists the issues that were encoured in the source code.


\begin{figure}[hbt!]

\listoffigures
\end{figure}




\section{General remarks}

In this section we present some recurrent issues that were encountered in the source code and some general good practices that should be respected. 

%\subsection{Typography of Static Variables}
%\label{readability:static}

%A good coding convention is to use typography to visually discriminate static variables from other variables, for example using a prefix such as {\tt s\_}.

%\subsection{Typography of Global Variables}
%\label{readability:global}

%A good coding convention is to use typography to visually discriminate global variables from local variables, for example using a prefix such as {\tt m\_} or {\tt g\_}.

\subsection{Typography of Internal Functions}
\label{readability:internal}

A good coding convention is to use typography to visually discriminate public functions and internal functions, for example using a prefix such as {\tt \_}.

%\subsection{Accept Methods without Checks}
%\label{accept:all}

%Public methods using {\tt tvm.accept()} without any prior check should not exist. Indeed, such methods could be used by attackers to drain the balance of the contracts, even with minor amounts but unlimited number of messages.

\subsection{Constructors without checks}
\label{constructor:check}

Contract constructors should always at the very least verify that the contract's public key is set and that the deployer is the owner of the contract. This is important especially in the case in which the contract has arguments that set the state variables. If it is not done, it opens the gate to various kinds of attacks.

\section{Contract deployment from Platform}

\issueCritical{Unprotected constructors in many contracts}{
  See \ref{constructor:check}. 
  
  Other than the `RootTokenContract` and `TONTokenWallet` contracts, all the other contracts have unprotected constructors and a comment that says that the contract will be deployed from the Platform. That does mean that is no longer necessary to check that the deployer of the contract is the owner of the contract. It is especially dangerous in the contracts which set the owner through the constructor like: `MarketAggregator`, `BorrowModule`, `LiquidationModule`, `RepayModule`, `SupplyModule`, `WithdrawModule`, `Oracle`, `TIP3TokenDeployer`, `UserAccount`, `UserAccountManager`, `Platform` and `WalletController`.
}

\subsection{Possible attack}

It makes it possible to perform phishing attacks by deploying fake contracts with which the users can interact with. So instead of interacting with the real contract they interact with the fake. 

If a malicious user deploys a fake `UserAccountManager` which will deploy user's accounts. And one of the users requests a withdraw of their tokens. The owner of the fake `UserAccountManager` can either block the transaction stopping the user's from withdrawing their tokens, or ask them for a fee before processing their request.


\section{Unsafe role assignement}

\issueCritical{Unsafe role assignement in {\tt IRoles.sol}}{
  The role setter functions {\tt setUpgrader} and {\tt setParamChanger} in the {\tt IRoles} abstract contract (file: "IRoles.sol") allow the contract owner to assign the upgrader or paramChanger role to the provided address or unassign it. The function should test that the provided address's value is not zero. Otherwise the owner can mistakingly give that right to all external users, which would make it possible for anyone to modify the functioning of the system. If the goal is to assign those roles to external users, then it's their pubkeys that should be stored, not adresses.
}

\section{Internal function names}

\issueMinor{Internal function names}{
  See \ref{readability:internal}. 

  The functions {\tt performOperation} and {\tt updatePrice} in the {\tt MarketAggregator} contract (file: "MarketsAggregator.sol"). are internal so their names should start with `\_`.
}

\section{Undefined functions}

\issueMinor{Undefined functions}{
  The functions {\tt calculateUtilizationRate}, {\tt calculateBorrowingRate} and {\tt calculateExchangeRate} in the `MarketMath` library (file: "MarketMath.sol") are undefined and unused. Defining and using them appropriately would significantly improve the readability of the source code.
}

\section{Unused functions}

\issueMinor{Unused functions}{
  The functions: 
  \begin{itemize}
    \item {\tt calculateExchangeRate} in the {\tt MarketMath} library (file: "MarketMath.sol")
    \item {\tt calculateU}, {\tt calculateTotalReserves}, {\tt calculateNewIndex}, {\tt calculateTotalBorrowed} and {\tt calculateReserves} in the {\tt MarketOperations} library (file: "MarketOperations.sol")
    \item {\tt \_calculateBorrowInfo} in the {\tt MarketAggregator} conract (file: "MarketsAggregator.sol")
  \end{itemize}
  And all the functions from the {\tt MarketToUserPayloads} library (file: "MarketPayloads.sol") and {\tt TvmCellOperations} library (file: "TvmCellOperations.sol") are unused. Unused functions should be removed from the code as they are useless and they clutter the code.
}

\section{Unused modifiers}

\issueMinor{Unused modifiers}{
  The modifiers: 
  \begin{itemize}
    \item {\tt onlySelf}, {\tt onlyRealTokenRoot} and {\tt onlyExecutor} in the {\tt MarketAggregator} contract (file: "MarketsAggregator.sol")
    \item {\tt onlyMarket} in the {\tt WalletController} conract (file: "WalletController.sol")
  \end{itemize}
  Are unused. Unused modifiers should be removed from the code as they are useless and they clutter the code.
}


\chapter{Contract Giver}

\minitoc

\section{Overview}


In file {\tt Giver.sol}

\section{Constructor Definitions}


\subsection{Constructor}

\issueCritical{Constructor for Giver (fake)}{loren ipsum  loren ipsum  loren ipsum loren ipsum loren ipsum loren ipsum loren ipsum loren ipsum loren ipsum loren ipsum loren ipsum loren ipsum loren ipsum loren ipsum loren ipsum loren ipsum loren ipsum loren ipsum

loren ipsum loren ipsum loren ipsum loren ipsum loren ipsum loren ipsum
loren ipsum loren ipsum loren ipsum }
\noindent\begin{itemize}
\item TODO
\end{itemize}

\begin{lstlisting}[firstnumber=6]
    constructor() public {
        tvm.accept();
    }
\end{lstlisting}

\section{Public Method Definitions}


\subsection{Function sendGrams}

\noindent\begin{itemize}
\item TODO
\end{itemize}

\begin{lstlisting}[firstnumber=10]
    function sendGrams(address dest, uint64 amount) external pure {
        tvm.accept();
        address(dest).transfer({value: amount, bounce: false});
    }
\end{lstlisting}
 %ignored for now

\section{Library MarketMath}

In file {\tt MarketMath.sol}


\subsection{Function calculateUtilizationRate}

\begin{lstlisting}[firstnumber=4]
    function calculateUtilizationRate(uint256 currentPool, uint256 totalBorrowed) internal pure returns (uint256) {

    }
\end{lstlisting}

\noindent\begin{itemize}
  \item \undefinedFunction{MarketMath.calculateUtilizationRate}
\end{itemize}


\subsection{Function calculateBorrowingRate}

\begin{lstlisting}[firstnumber=8]
    function calculateBorrowingRate(uint256 currentPool, uint256 totalBorrowed, uint256 totalReserves, uint256 totalSupply) 
        internal pure returns (uint256) 
    {

    }
\end{lstlisting}

\noindent\begin{itemize}
  \item \undefinedFunction{MarketMath.calculateBorrowingRate}
\end{itemize}


\subsection{Function calculateExchangeRate}

\begin{lstlisting}[firstnumber=14]
    function calculateExchangeRate(uint256 currentPool, uint256 totalBorrowed, uint256 totalReserves, uint256 totalSupply)
        internal pure returns (uint256) 
    {
        return math.div(currentPool - totalReserves + totalBorrowed, totalSupply);
    }
\end{lstlisting}

\noindent\begin{itemize}
  \item \undefinedFunction{MarketMath.calculateExchangeRate}
  \item \issueMinor{Syntax Error in {\tt MarketMath.calculateExchangeRate}}{{\tt math.div} does not exist. It should use the infix division operator {\tt /}.}
\end{itemize}


\subsection{Function recalculateState}

\begin{lstlisting}[firstnumber=20]
    function recalculateState(uint256 currentPool, uint256 totalBorrowed, uint256 totalReserves, uint256 totalSupply)
        internal pure 
    {
        // uint256 exchangeRate = calculateExchangeRate(currentPool, totalBorrowed, totalReserves, totalSupply);
    }
\end{lstlisting}

\noindent\begin{itemize}
  \item \undefinedFunction{MarketMath.recalculateState}
\end{itemize}






\section{Library MarketOperations}

In file {\tt MarketOperations.sol}


\subsection{Function calculateU}

\begin{lstlisting}[firstnumber=9]
    function calculateU(uint256 totalBorrowed, uint256 realTokens) internal pure returns (fraction) {
        return fraction(totalBorrowed, totalBorrowed + realTokens);
    }
\end{lstlisting}

\noindent\begin{itemize}
  \item \unusedFunction{MarketOperations.calculateU}
\end{itemize}


\subsection{Function calculateTotalReserves}

\begin{lstlisting}[firstnumber=28]
    function calculateTotalReserves(uint256 totalReserve, uint256 totalBorrowed, fraction r, fraction reserveFactor, uint256 t) internal returns (fraction) {
        fraction tr;
        tr = r.fNumMul(t);
        tr = tr.fMul(reserveFactor);
        tr = tr.fNumMul(totalBorrowed);
        tr = tr.fNumAdd(totalReserve);
        return tr;
    }
\end{lstlisting}

\noindent\begin{itemize}
  \item \unusedFunction{MarketOperations.calculateTotalReserves}
\end{itemize}


\subsection{Function calculateNewIndex}

\begin{lstlisting}[firstnumber=37]
    function calculateNewIndex(fraction index, fraction bir, uint256 dt) internal returns (fraction) {
        fraction index_;
        index_ = bir.fNumMul(dt);
        index_ = index_.fNumAdd(1);
        index_ = index_.fAdd(index);
        return index_;
    }
\end{lstlisting}

\noindent\begin{itemize}
  \item \unusedFunction{MarketOperations.calculateNewIndex}
\end{itemize}


\subsection{Function calculateTotalBorrowed}

\begin{lstlisting}[firstnumber=45]
    function calculateTotalBorrowed(uint256 totalBorrowed, fraction oldIndex, fraction newIndex) internal returns (uint256) {
        fraction tb_;
        tb_ = totalBorrowed.numFDiv(oldIndex);
        tb_ = tb_.fMul(newIndex);
        return tb_.toNum();
    }
\end{lstlisting}

\noindent\begin{itemize}
  \item \unusedFunction{MarketOperations.calculateTotalBorrowed}
\end{itemize}


\subsection{Function calculateReserves}

\begin{lstlisting}[firstnumber=52]
    function calculateReserves(uint256 reserveOld, uint256 totalBorrowedOld, fraction bir, fraction reserveFactor, uint256 dt) internal returns (uint256) {
        fraction res = bir;
        res = res.fNumMul(dt);
        res = res.fMul(reserveFactor);
        res = res.fNumMul(totalBorrowedOld);
        res = res.fNumAdd(reserveOld);
        return res.toNum();
    }
\end{lstlisting}

\noindent\begin{itemize}
  \item \unusedFunction{MarketOperations.calculateReserves}
\end{itemize}


\section{Library MarketToUserPayloads}

In file {\tt MarketPayloads.sol}

\issueMinor{Unused functions}{All the functions in the file are unused.}



\section{Contract MarketAggregator}

In file {\tt MarketsAggregator.sol}

\subsection{Function performOperation}

\begin{lstlisting}[firstnumber=361]
    function performOperation(TvmCell args) internal view {
        TvmSlice ts = args.toSlice();

        uint8 operationId = ts.decode(uint8);
        if (operationId != OperationCodes.NO_OP) {
            uint32 marketId = ts.decode(uint32);
            TvmCell moduleArgs = ts.loadRef();
            IModule(modules[operationId]).performAction{
                flag: MsgFlag.REMAINING_GAS
            }(marketId, moduleArgs, markets, tokenPrices);
        } else {
            address(_owner).transfer({value: 0, flag: MsgFlag.REMAINING_GAS});
        }
    }
\end{lstlisting}

\noindent\begin{itemize}
  \item \internalFunctionName{MarketAggregator.performOperation}
\end{itemize}


\subsection{Function updatePrice}

\begin{lstlisting}[firstnumber=428]
    function updatePrice(address tokenRoot, TvmCell payload) internal view {
        IOracleReturnPrices(oracle).getTokenPrice{
            flag: MsgFlag.REMAINING_GAS,
            callback: this.receiveUpdatedPrice
        }(tokenRoot, payload);
    }
\end{lstlisting}

\noindent\begin{itemize}
  \item \internalFunctionName{MarketAggregator.updatePrice}
\end{itemize}
  
  
\subsection{Function \_{}calculateBorrowInfo}

\begin{lstlisting}[firstnumber=516]
    function _calculateBorrowInfo(mapping(uint32 => BorrowInfo) borrowInfo, mapping(uint32 => fraction) updatedIndexes) internal returns(mapping (uint32=>uint256) userBorrowInfo) {
        for ((uint32 marketId, BorrowInfo bi): borrowInfo) {
            if (bi.tokensBorrowed != 0) {
                fraction tmpf = borrowInfo[marketId].tokensBorrowed.numFMul(updatedIndexes[marketId]);
                tmpf = tmpf.fDiv(bi.index);
                userBorrowInfo[marketId] = tmpf.toNum();
            } else {
                userBorrowInfo[marketId] = 0;
            }
        }
    }
\end{lstlisting}

\noindent\begin{itemize}
  \item \unusedFunction{MarketAggregator.\_calculateBorrowInfo}
\end{itemize}


\subsection{Modifier onlySelf}

\begin{lstlisting}[firstnumber=510]
    modifier onlySelf() {
        require(msg.sender == address(this), MarketErrorCodes.ERROR_MSG_SENDER_IS_NOT_SELF);
        _;
    }
\end{lstlisting}

\noindent\begin{itemize}
  \item \unusedModifier{MarketAggregator.onlySelf}
\end{itemize}


\subsection{Modifier onlyRealTokenRoot}

\begin{lstlisting}[firstnumber=533]
    modifier onlyRealTokenRoot() {
        require(realTokenRoots.exists(msg.sender));
        _;
    }
\end{lstlisting}

\noindent\begin{itemize}
  \item \unusedModifier{MarketAggregator.onlyRealTokenRoot}
\end{itemize}


\subsection{Modifier onlyExecutor}

\begin{lstlisting}[firstnumber=543]
    modifier onlyExecutor() {
        require(
            (msg.sender == userAccountManager) ||
            (isModule.exists(msg.sender))
        );
        _;
    }
\end{lstlisting}

\noindent\begin{itemize}
  \item \unusedModifier{MarketAggregator.onlyExecutor}
\end{itemize}





\section{Library Utilities}

In file {\tt IModule.sol}

\subsection{Function calculateSupplyBorrow}

\noindent\begin{itemize}
\item \issueMinor{Naming}{The function is called ``calculateSupplyBorrow'' but it calculates a user's account health. It should be named accordingly, e.g. ``calculateAccountHealth''.}
\end{itemize}

\begin{lstlisting}[firstnumber=90]
    function calculateSupplyBorrow(
        mapping(uint32 => uint256) supplyInfo,
        mapping(uint32 => BorrowInfo) borrowInfo,
        mapping(uint32 => MarketInfo) marketInfo,
        mapping(address => fraction) tokenPrices
    ) internal returns (fraction) {
        fraction accountHealth = fraction(0, 0);
        fraction tmp;
        fraction nom = fraction(0, 1);
        fraction denom = fraction(0, 1);

        // Supply:
        // 1. Calculate real token amount: vToken*exchangeRate
        // 2. Calculate real token amount in USD: realTokens/tokenPrice
        // 3. Multiply by collateral factor: usdValue*collateralFactor
        for ((uint32 marketId, uint256 supplied): supplyInfo) {
            tmp = supplied.numFMul(marketInfo[marketId].exchangeRate);
            tmp = tmp.fDiv(tokenPrices[marketInfo[marketId].token]);
            tmp = tmp.fMul(marketInfo[marketId].collateralFactor);
            nom = nom.fAdd(tmp);
            nom = nom.simplify();
        }

        // Borrow:
        // 1. Recalculate borrow amount according to new index
        // 2. Calculate borrow value in USD
        // NOTE: no conversion from vToken to real tokens required, as value is stored in real tokens
        for ((uint32 marketId, BorrowInfo _bi): borrowInfo) {
            if (_bi.tokensBorrowed != 0) {
                if (!_bi.index.eq(marketInfo[marketId].index)) {
                    tmp = borrowInfo[marketId].tokensBorrowed.numFMul(marketInfo[marketId].index);
                    tmp = tmp.fDiv(borrowInfo[marketId].index);
                } else {
                    tmp = borrowInfo[marketId].tokensBorrowed.toF();
                }
                tmp = tmp.fDiv(tokenPrices[marketInfo[marketId].token]);
                tmp = tmp.simplify();
                denom = denom.fAdd(tmp);
                denom = denom.simplify();
            }
        }

        accountHealth = nom.fDiv(denom);

        return accountHealth;
    }
\end{lstlisting}


\section{Contract BorrowModule}

In file {\tt BorrowModule.sol}

\subsection{Function borrowTokensFromMarket}

\begin{lstlisting}[firstnumber=74]
    function borrowTokensFromMarket(
        address tonWallet,
        address userTip3Wallet,
        uint256 tokensToBorrow,
        uint32 marketId,
        mapping (uint32 => uint256) supplyInfo,
        mapping (uint32 => BorrowInfo) borrowInfo
    ) external override onlyUserAccountManager {
        tvm.rawReserve(msg.value, 2);
        mapping(uint32 => MarketDelta) marketsDelta;
        MarketDelta marketDelta;
        
        // Borrow:
        // 1. Check that market has enough tokens for lending
        // 2. Calculate user account health
        // 3. Calculate USD value of tokens to borrow
        // 4. Check if there is enough (collateral - borrowed) for new token borrow
        // 5. Increase user's borrowed amount

        MarketInfo mi = marketInfo[marketId];

        if (tokensToBorrow < mi.realTokenBalance - mi.totalReserve) {
            fraction accountHealth = Utilities.calculateSupplyBorrow(supplyInfo, borrowInfo, marketInfo, tokenPrices);
            if (accountHealth.nom > accountHealth.denom) {
                uint256 healthDelta = accountHealth.nom - accountHealth.denom;
                fraction tmp = healthDelta.numFMul(tokenPrices[marketInfo[marketId].token]);
                uint256 possibleTokenWithdraw = tmp.toNum();
                if (possibleTokenWithdraw >= tokensToBorrow) {
                    marketDelta.totalBorrowed.delta = tokensToBorrow;
                    marketDelta.totalBorrowed.positive = true;
                    marketDelta.realTokenBalance.delta = tokensToBorrow;
                    marketDelta.realTokenBalance.positive = false;

                    marketsDelta[marketId] = marketDelta;

                    TvmBuilder tb;
                    tb.store(marketId);
                    tb.store(tonWallet);
                    tb.store(userTip3Wallet);
                    tb.store(tokensToBorrow);

                    emit TokenBorrow(marketId, marketDelta, tonWallet, tokensToBorrow);

                    IContractStateCacheRoot(marketAddress).receiveCacheDelta{
                        flag: MsgFlag.REMAINING_GAS
                    }(marketsDelta, tb.toCell());
                } else {
                    IUAMUserAccount(userAccountManager).writeBorrowInformation{
                        flag: MsgFlag.REMAINING_GAS
                    }(tonWallet, userTip3Wallet, 0, marketId, marketInfo[marketId].index);
                }
            } else {
                IUAMUserAccount(userAccountManager).writeBorrowInformation{
                    flag: MsgFlag.REMAINING_GAS
                }(tonWallet, userTip3Wallet, 0, marketId, marketInfo[marketId].index);
            }
        } else {
            address(tonWallet).transfer({value: 0, flag: MsgFlag.REMAINING_GAS});
        }
    }
\end{lstlisting}

\noindent\begin{itemize}
  \item \issueCritical{Math error}{Line 99. To caculate the amount of tokens that it is possible to withdraw, the health delta needs to be divided by the price of the token not multiplied by it.}
\end{itemize}



\section{Module "LiquidationModule.sol"}


\subsection{Pragmas}


\noindent\begin{tabular}{|l|l|p{5cm}|}\hline
ton & -solidity $>$= 0.47.0 &\\\hline
\end{tabular}


\subsection{Imports}


\noindent\begin{tabular}{|l|l|p{5cm}|}\hline
./interfaces/IModule.sol &\\\hline
\end{tabular}


\subsection{Contract Definitions}

\begin{itemize}
\item LiquidationModule
\end{itemize}


\section{Module "RepayModule.sol"}


\subsection{Pragmas}


\noindent\begin{tabular}{|l|l|p{5cm}|}\hline
ton & -solidity $>$= 0.47.0 &\\\hline
\end{tabular}


\subsection{Imports}


\noindent\begin{tabular}{|l|l|p{5cm}|}\hline
./interfaces/IModule.sol &\\\hline
../utils/libraries/MsgFlag.sol &\\\hline
\end{tabular}


\subsection{Contract Definitions}

\begin{itemize}
\item RepayModule
\end{itemize}


\section{Module "SupplyModule.sol"}


\subsection{Pragmas}


\noindent\begin{tabular}{|l|l|p{5cm}|}\hline
ton & -solidity $>$= 0.47.0 &\\\hline
\end{tabular}


\subsection{Imports}


\noindent\begin{tabular}{|l|l|p{5cm}|}\hline
./interfaces/IModule.sol &\\\hline
../utils/libraries/MsgFlag.sol &\\\hline
\end{tabular}


\subsection{Contract Definitions}

\begin{itemize}
\item SupplyModule
\end{itemize}


\section{Module "WithdrawModule.sol"}


\subsection{Pragmas}


\noindent\begin{tabular}{|l|l|p{5cm}|}\hline
ton & -solidity $>$= 0.47.0 &\\\hline
\end{tabular}


\subsection{Imports}


\noindent\begin{tabular}{|l|l|p{5cm}|}\hline
./interfaces/IModule.sol &\\\hline
../utils/libraries/MsgFlag.sol &\\\hline
\end{tabular}


\subsection{Contract Definitions}

\begin{itemize}
\item WithdrawModule
\end{itemize}



\section{Module "Oracle.sol"}


\subsection{Pragmas}


\noindent\begin{tabular}{|l|l|p{5cm}|}\hline
ton & -solidity $>$= 0.43.0 &\\\hline
AbiHeader &  time &\\\hline
AbiHeader &  expire &\\\hline
AbiHeader &  pubkey &\\\hline
\end{tabular}


\subsection{Imports}


\noindent\begin{tabular}{|l|l|p{5cm}|}\hline
./interfaces/IOracleService.sol &\\\hline
./interfaces/IOracleUpdatePrices.sol &\\\hline
./interfaces/IOracleReturnPrices.sol &\\\hline
./interfaces/IOracleManageTokens.sol &\\\hline
./libraries/CostConstants.sol &\\\hline
./libraries/OracleErrorCodes.sol &\\\hline
../utils/libraries/MsgFlag.sol &\\\hline
../utils/Dex/IDexPair.sol &\\\hline
../utils/interfaces/IUpgradableContract.sol &\\\hline
../utils/interfaces/IRoles.sol &\\\hline
\end{tabular}


\subsection{Contract Definitions}

\begin{itemize}
\item Oracle
\end{itemize}


\section{Module "TIP3Deployer.sol"}


\subsection{Pragmas}


\noindent\begin{tabular}{|l|l|p{5cm}|}\hline
ton & -solidity $>$= 0.39.0 &\\\hline
AbiHeader &  pubkey &\\\hline
AbiHeader &  expire &\\\hline
AbiHeader &  time &\\\hline
\end{tabular}


\subsection{Imports}


\noindent\begin{tabular}{|l|l|p{5cm}|}\hline
./interfaces/ITIP3Deployer.sol &\\\hline
./interfaces/ITIP3DeployerManageCode.sol &\\\hline
./interfaces/ITIP3DeployerServiceInfo.sol &\\\hline
./libraries/TIP3DeployerErrorCodes.sol &\\\hline
../utils/libraries/MsgFlag.sol &\\\hline
../utils/interfaces/IUpgradableContract.sol &\\\hline
../utils/TIP3/RootTokenContract.sol &\\\hline
\end{tabular}


\subsection{Contract Definitions}

\begin{itemize}
\item TIP3TokenDeployer
\end{itemize}


\bigskip

\section{Module "IUserAccount.sol"}

\subsection{Struct UserMarketInfo}

\begin{lstlisting}[firstnumber=10]
  struct UserMarketInfo {
      bool exists;
      uint32 _marketId;
      uint256 suppliedTokens;
      fraction accountHealth;
      BorrowInfo borrowInfo;
  }
  \end{lstlisting}

\issueMinor{Unused struct field in {\tt UserMarketInfo}}{The field {\tt accountHealth} is unused.}






\chapter{Contract UserAccountManager}

\minitoc

\section{Overview}


In file {\tt UserAccountManager.sol}

\section{Contract Inheritance}


\noindent\begin{tabular}{|l|p{5cm}|}\hline
IRoles & \\\hline
IUpgradableContract & \\\hline
IUserAccountManager & \\\hline
IUAMUserAccount & \\\hline
IUAMMarket & \\\hline
\end{tabular}


\section{Event Definitions}


\begin{lstlisting}[firstnumber=33]
    event AccountCreated(address tonWallet, address userAddress);
\end{lstlisting}

\section{Variable Definitions}


\ifsoltables
\noindent\begin{tabular}{|l|l|p{5cm}|}\hline
uint32 & contractCodeVersion &  \\\hline
 & & assigned in @14.UserAccountManager.onCodeUpgrade\\\hline
 & & used in @14.UserAccountManager.onCodeUpgrade\\\hline
address & marketAddress &  \\\hline
 & & used in @14.UserAccountManager.upgradeContractCode\\\hline
 & & assigned in @14.UserAccountManager.setMarketAddress\\\hline
 & & used in @14.UserAccountManager.setMarketAddress\\\hline
 & & used in @14.UserAccountManager.returnAndSupply\\\hline
 & & used in @14.UserAccountManager.returnAndSupply\\\hline
 & & used in @14.UserAccountManager.returnAndSupply\\\hline
 & & used in @14.UserAccountManager.requestWithdraw\\\hline
 & & used in @14.UserAccountManager.requestTokenPayout\\\hline
 & & used in @14.UserAccountManager.requestIndexUpdate\\\hline
 & & assigned in @14.UserAccountManager.onCodeUpgrade\\\hline
 & & used in @14.UserAccountManager.onCodeUpgrade\\\hline
 & & used in @14.UserAccountManager.calculateUserAccountHealth\\\hline
mapping (uint8 =$>$ address) & modules &  \\\hline
 & & used in @14.UserAccountManager.upgradeContractCode\\\hline
 & & assigned in @14.UserAccountManager.removeModule\\\hline
 & & used in @14.UserAccountManager.removeModule\\\hline
 & & used in @14.UserAccountManager.removeModule\\\hline
 & & used in @14.UserAccountManager.receiveWithdrawInfo\\\hline
 & & used in @14.UserAccountManager.receiveRepayInfo\\\hline
 & & used in @14.UserAccountManager.receiveLiquidationInformation\\\hline
 & & used in @14.UserAccountManager.passBorrowInformation\\\hline
 & & assigned in @14.UserAccountManager.onCodeUpgrade\\\hline
 & & used in @14.UserAccountManager.onCodeUpgrade\\\hline
 & & assigned in @14.UserAccountManager.addModule\\\hline
 & & used in @14.UserAccountManager.addModule\\\hline
mapping (address =$>$ bool) & existingModules &  \\\hline
 & & used in @14.UserAccountManager.upgradeContractCode\\\hline
 & & assigned in @14.UserAccountManager.removeModule\\\hline
 & & used in @14.UserAccountManager.removeModule\\\hline
 & & assigned in @14.UserAccountManager.onCodeUpgrade\\\hline
 & & used in @14.UserAccountManager.onCodeUpgrade\\\hline
 & & assigned in @14.UserAccountManager.addModule\\\hline
 & & used in @14.UserAccountManager.addModule\\\hline
 & & assigned in @14.UserAccountManager.addModule\\\hline
 & & used in @14.UserAccountManager.addModule\\\hline
mapping (uint32 =$>$ TvmCell) & userAccountCodes &  \\\hline
 & & assigned in @14.UserAccountManager.uploadUserAccountCode\\\hline
 & & used in @14.UserAccountManager.uploadUserAccountCode\\\hline
 & & used in @14.UserAccountManager.upgradeContractCode\\\hline
 & & used in @14.UserAccountManager.updateUserAccount\\\hline
 & & assigned in @14.UserAccountManager.onCodeUpgrade\\\hline
 & & used in @14.UserAccountManager.onCodeUpgrade\\\hline
 & & used in @14.UserAccountManager.getUserAccountCode\\\hline
 & & used in @14.UserAccountManager.createUserAccount\\\hline
 & & used in @14.UserAccountManager.createUserAccount\\\hline
 & & used in @14.UserAccountManager.\_{}buildUserAccountData\\\hline
\end{tabular}
\fi


\begin{lstlisting}[firstnumber=26]
    uint32 public contractCodeVersion;
\end{lstlisting}

\begin{lstlisting}[firstnumber=28]
    address public marketAddress;
\end{lstlisting}

\begin{lstlisting}[firstnumber=29]
    mapping(uint8 => address) public modules;
\end{lstlisting}

\begin{lstlisting}[firstnumber=30]
    mapping(address => bool) public existingModules;
\end{lstlisting}

\begin{lstlisting}[firstnumber=31]
    mapping(uint32 => TvmCell) public userAccountCodes;
\end{lstlisting}

\section{Modifier Definitions}


\subsection{Modifier onlyMarket}


\begin{lstlisting}[firstnumber=557]
    modifier onlyMarket() {
        require(
            msg.sender == marketAddress,
            UserAccountErrorCodes.ERROR_NOT_MARKET
        );
        tvm.rawReserve(msg.value, 2);
        _;
    }
\end{lstlisting}

\subsection{Modifier onlyTrusted}


\begin{lstlisting}[firstnumber=566]
    modifier onlyTrusted() {
        require(
            msg.sender == _owner ||
            msg.sender == marketAddress ||
            _canChangeParams[msg.sender],
            UserAccountErrorCodes.ERROR_NOT_TRUSTED
        );
        _;
    }
\end{lstlisting}

\subsection{Modifier onlyModules}


\begin{lstlisting}[firstnumber=576]
    modifier onlyModules() {
        require(
            existingModules.exists(msg.sender),
            UserAccountErrorCodes.ERROR_NOT_MODULE
        );
        _;
    }
\end{lstlisting}

\subsection{Modifier executor}


\begin{lstlisting}[firstnumber=584]
    modifier executor() {
        require(
            msg.sender == _owner ||
            msg.sender == marketAddress ||
            existingModules.exists(msg.sender),
            UserAccountErrorCodes.ERROR_NOT_EXECUTOR
        );
        _;
    }
\end{lstlisting}

\subsection{Modifier onlyModule}


\begin{lstlisting}[firstnumber=594]
    modifier onlyModule(uint8 operationId) {
        require(
            msg.sender == modules[operationId],
            UserAccountErrorCodes.ERROR_INVALID_MODULE
        );
        tvm.rawReserve(msg.value, 2);
        _;
    }
\end{lstlisting}

\subsection{Modifier onlySelectedExecutors}


\begin{lstlisting}[firstnumber=603]
    modifier onlySelectedExecutors(uint8 operationId, address tonWallet) {
        require(
            (msg.sender == modules[operationId]) ||
            (msg.sender == _calculateUserAccountAddress(tonWallet)),
            UserAccountErrorCodes.ERROR_INVALID_EXECUTOR
        );
        _;
    }
\end{lstlisting}

\subsection{Modifier onlyValidUserAccount}


\begin{lstlisting}[firstnumber=615]
    modifier onlyValidUserAccount(address tonWallet) {
        require(
            msg.sender == _calculateUserAccountAddress(tonWallet),
            UserAccountErrorCodes.INVALID_USER_ACCOUNT
        );
        tvm.rawReserve(msg.value, 2);
        _;
    }
\end{lstlisting}

\subsection{Modifier onlyValidUserAccountNoReserve}


\begin{lstlisting}[firstnumber=624]
    modifier onlyValidUserAccountNoReserve(address tonWallet) {
        require(
            msg.sender == _calculateUserAccountAddress(tonWallet),
            UserAccountErrorCodes.INVALID_USER_ACCOUNT
        );
        _;
    }
\end{lstlisting}

\section{Constructor Definitions}


\subsection{Constructor}

\issueCritical{Constructor for UserAccountManager (fake)}{loren ipsum  loren ipsum  loren ipsum loren ipsum loren ipsum loren ipsum loren ipsum loren ipsum loren ipsum loren ipsum loren ipsum loren ipsum loren ipsum loren ipsum loren ipsum loren ipsum loren ipsum loren ipsum

loren ipsum loren ipsum loren ipsum loren ipsum loren ipsum loren ipsum
loren ipsum loren ipsum loren ipsum }
\noindent\begin{itemize}
\item TODO
\end{itemize}

\begin{lstlisting}[firstnumber=38]
    constructor(address _newOwner) public {
        tvm.accept();
        _owner = _newOwner;
    }
\end{lstlisting}

\section{Public Method Definitions}


\subsection{Function abortLiquidation}

\noindent\begin{itemize}
\item TODO
\end{itemize}

\begin{lstlisting}[firstnumber=386]
    function abortLiquidation(
        address tonWallet, 
        address targetUser, 
        address tip3UserWallet, 
        uint32 marketId, 
        uint256 tokensProvided
    ) external override view onlyModule(OperationCodes.LIQUIDATE_TOKENS) {
        address userAccount = _calculateUserAccountAddress(targetUser);
        IUserAccountData(userAccount).abortLiquidation{
            flag: MsgFlag.REMAINING_GAS
        }(tonWallet, tip3UserWallet, marketId, tokensProvided);
    }
\end{lstlisting}

\subsection{Function addModule}

\noindent\begin{itemize}
\item TODO
\end{itemize}

\begin{lstlisting}[firstnumber=543]
    function addModule(uint8 operationId, address module) external override onlyTrusted {
        delete existingModules[module];
        modules[operationId] = module;
        existingModules[module] = true;
    }
\end{lstlisting}

\subsection{Function calculateUserAccountAddress}

\noindent\begin{itemize}
\item TODO
\end{itemize}

\begin{lstlisting}[firstnumber=123]
    function calculateUserAccountAddress(address tonWallet) external override responsible view returns (address) {
        return { value: 0, bounce: false, flag: MsgFlag.REMAINING_GAS } _calculateUserAccountAddress(tonWallet);
    }
\end{lstlisting}

\subsection{Function calculateUserAccountHealth}

\noindent\begin{itemize}
\item TODO
\end{itemize}

\begin{lstlisting}[firstnumber=446]
    function calculateUserAccountHealth(
        address tonWallet, 
        address gasTo,
        mapping(uint32 => uint256) supplyInfo,
        mapping(uint32 => BorrowInfo) borrowInfo,
        TvmCell dataToTransfer
    ) external override view onlyValidUserAccount(tonWallet) {
        tvm.rawReserve(msg.value, 2);
        IMarketOperations(marketAddress).calculateUserAccountHealth{
            flag: MsgFlag.REMAINING_GAS
        }(tonWallet, gasTo, supplyInfo, borrowInfo, dataToTransfer);
    }
\end{lstlisting}

\subsection{Function createUserAccount}

\noindent\begin{itemize}
\item TODO
\end{itemize}

\begin{lstlisting}[firstnumber=97]
    function createUserAccount(address tonWallet) external override view {
        tvm.rawReserve(msg.value, 2);

        TvmSlice ts = userAccountCodes[0].toSlice();
        require(!ts.empty());

        address userAccount = new UserAccount{
            value: UserAccountCostConstants.useForUADeploy,
            code: userAccountCodes[0],
            pubkey: 0,
            varInit: {
                owner: tonWallet
            }
        }();

        emit AccountCreated(tonWallet, userAccount);

        IUserAccountManager(this).updateUserAccount{
            value: msg.value - UserAccountCostConstants.useForUADeploy - UserAccountCostConstants.estimatedExecCost
        }(tonWallet);
    }
\end{lstlisting}

\subsection{Function disableUserAccountLock}

\noindent\begin{itemize}
\item TODO
\end{itemize}

\begin{lstlisting}[firstnumber=525]
    function disableUserAccountLock(address tonWallet) external view onlyOwner {
        tvm.rawReserve(msg.value, 2);
        address userAccount = _calculateUserAccountAddress(tonWallet);
        IUserAccountData(userAccount).disableBorrowLock{
            flag: MsgFlag.REMAINING_GAS
        }();
    }
\end{lstlisting}

\subsection{Function getUserAccountCode}

\noindent\begin{itemize}
\item TODO
\end{itemize}

\begin{lstlisting}[firstnumber=521]
    function getUserAccountCode(uint32 version) external override view responsible returns(TvmCell) {
        return {flag: MsgFlag.REMAINING_GAS} userAccountCodes[version];
    }
\end{lstlisting}

\subsection{Function grantVTokens}

\noindent\begin{itemize}
\item TODO
\end{itemize}

\begin{lstlisting}[firstnumber=363]
    function grantVTokens(
        address tonWallet, 
        address targetUser,
        address tip3UserWallet,
        uint32 marketId, 
        uint32 marketToLiquidate,
        uint256 vTokensToGrant, 
        uint256 tokensToReturn,
        uint256 tokensFromReserve
    ) external override view onlyValidUserAccountNoReserve(targetUser) {
        tvm.rawReserve(msg.value - UserAccountCostConstants.updateHealthCost, 2);
        
        address targetAccount = _calculateUserAccountAddress(targetUser);
        IUserAccountData(targetAccount).checkUserAccountHealth{
            value: UserAccountCostConstants.updateHealthCost
        }(tonWallet);

        address userAccount = _calculateUserAccountAddress(tonWallet);
        IUserAccountData(userAccount).grantVTokens{
            flag: MsgFlag.REMAINING_GAS
        }(tip3UserWallet, marketId, marketToLiquidate, vTokensToGrant, tokensToReturn, tokensFromReserve);
    }
\end{lstlisting}

\subsection{Function passBorrowInformation}

\noindent\begin{itemize}
\item TODO
\end{itemize}

\begin{lstlisting}[firstnumber=246]
    function passBorrowInformation(
        address tonWallet, 
        address userTip3Wallet, 
        uint32 marketId, 
        uint256 tokensToBorrow, 
        mapping(uint32 => uint256) supplyInfo, 
        mapping(uint32 => BorrowInfo) borrowInfo
    ) external override view onlyValidUserAccount(tonWallet) {
        IBorrowModule(modules[OperationCodes.BORROW_TOKENS]).borrowTokensFromMarket{
            flag: MsgFlag.REMAINING_GAS
        }(tonWallet, userTip3Wallet, tokensToBorrow, marketId, supplyInfo, borrowInfo);
    }
\end{lstlisting}

\subsection{Function receiveLiquidationInformation}

\noindent\begin{itemize}
\item TODO
\end{itemize}

\begin{lstlisting}[firstnumber=331]
    function receiveLiquidationInformation(
        address tonWallet, 
        address targetUser, 
        address tip3UserWallet, 
        uint32 marketId, 
        uint32 marketToLiquidate,
        uint256 tokensProvided, 
        mapping(uint32 => uint256) supplyInfo, 
        mapping(uint32 => BorrowInfo) borrowInfo
    ) external override view onlyValidUserAccount(targetUser) {
        ILiquidationModule(modules[OperationCodes.LIQUIDATE_TOKENS]).liquidate{
            flag: MsgFlag.REMAINING_GAS
        }(tonWallet, targetUser, tip3UserWallet, marketId, marketToLiquidate, tokensProvided, supplyInfo, borrowInfo);
    }
\end{lstlisting}

\subsection{Function receiveRepayInfo}

\noindent\begin{itemize}
\item TODO
\end{itemize}

\begin{lstlisting}[firstnumber=288]
    function receiveRepayInfo(
        address tonWallet, 
        address userTip3Wallet, 
        uint256 tokensForRepay,
        uint32 marketId,
        BorrowInfo borrowInfo
    ) external override view onlyValidUserAccount(tonWallet) {
        IRepayModule(modules[OperationCodes.REPAY_TOKENS]).repayLoan{
            flag: MsgFlag.REMAINING_GAS
        }(tonWallet, userTip3Wallet, tokensForRepay, marketId, borrowInfo);
    }
\end{lstlisting}

\subsection{Function receiveWithdrawInfo}

\noindent\begin{itemize}
\item TODO
\end{itemize}

\begin{lstlisting}[firstnumber=194]
    function receiveWithdrawInfo(
        address tonWallet, 
        address userTip3Wallet,
        uint256 tokensToWithdraw,
        uint32 marketId,
        mapping(uint32 => uint256) supplyInfo,
        mapping(uint32 => BorrowInfo) borrowInfo
    ) external override view onlyValidUserAccount(tonWallet) {
        IWithdrawModule(modules[OperationCodes.WITHDRAW_TOKENS]).withdrawTokensFromMarket{
            flag: MsgFlag.REMAINING_GAS
        }(tonWallet, userTip3Wallet, tokensToWithdraw, marketId, supplyInfo, borrowInfo);
    }
\end{lstlisting}

\subsection{Function removeMarket}

\noindent\begin{itemize}
\item TODO
\end{itemize}

\begin{lstlisting}[firstnumber=533]
    function removeMarket(address tonWallet, uint32 marketId) external view canChangeParams {
        tvm.rawReserve(msg.value, 2);
        address userAccount = _calculateUserAccountAddress(tonWallet);
        IUserAccountData(userAccount).removeMarket{
            flag: MsgFlag.REMAINING_GAS
        }(marketId);
    }
\end{lstlisting}

\subsection{Function removeModule}

\noindent\begin{itemize}
\item TODO
\end{itemize}

\begin{lstlisting}[firstnumber=549]
    function removeModule(uint8 operationId) external override onlyTrusted {
        delete existingModules[modules[operationId]];
        delete modules[operationId];
    }
\end{lstlisting}

\subsection{Function requestIndexUpdate}

\noindent\begin{itemize}
\item TODO
\end{itemize}

\begin{lstlisting}[firstnumber=223]
    function requestIndexUpdate(
        address tonWallet, 
        uint32 marketId, 
        TvmCell args
    ) external override view onlyValidUserAccount(tonWallet) {
        IMarketOperations(marketAddress).performOperationUserAccountManager{
            flag: MsgFlag.REMAINING_GAS
        }(OperationCodes.BORROW_TOKENS, marketId, args);
    }
\end{lstlisting}

\subsection{Function requestLiquidationInformation}

\noindent\begin{itemize}
\item TODO
\end{itemize}

\begin{lstlisting}[firstnumber=316]
    function requestLiquidationInformation(
        address tonWallet, 
        address targetUser, 
        address tip3UserWallet, 
        uint32 marketId, 
        uint32 marketToLiquidate,
        uint256 tokensProvided,
        mapping(uint32 => fraction) updatedIndexes
    ) external override view onlyModule(OperationCodes.LIQUIDATE_TOKENS) {
        address userAccount = _calculateUserAccountAddress(targetUser);
        IUserAccountData(userAccount).requestLiquidationInformation{
            flag: MsgFlag.REMAINING_GAS
        }(tonWallet, tip3UserWallet, marketId, marketToLiquidate, tokensProvided, updatedIndexes);
    }
\end{lstlisting}

\subsection{Function requestRepayInfo}

\noindent\begin{itemize}
\item TODO
\end{itemize}

\begin{lstlisting}[firstnumber=275]
    function requestRepayInfo(
        address tonWallet, 
        address userTip3Wallet, 
        uint256 tokensForRepay, 
        uint32 marketId,
        mapping(uint32 => fraction) updatedIndexes
    ) external override view onlyModule(OperationCodes.REPAY_TOKENS) {
        address userAccount = _calculateUserAccountAddress(tonWallet);
        IUserAccountData(userAccount).sendRepayInfo{
            flag: MsgFlag.REMAINING_GAS
        }(userTip3Wallet, marketId, tokensForRepay, updatedIndexes);
    }
\end{lstlisting}

\subsection{Function requestTokenPayout}

\noindent\begin{itemize}
\item TODO
\end{itemize}

\begin{lstlisting}[firstnumber=476]
    function requestTokenPayout(address tonWallet, address userTip3Wallet, uint32 marketId, uint256 toPayout) external override view onlySelectedExecutors(OperationCodes.LIQUIDATE_TOKENS, tonWallet) {
        IMarketOperations(marketAddress).requestTokenPayout{
            flag: MsgFlag.REMAINING_GAS
        }(tonWallet, userTip3Wallet, marketId, toPayout);
    }
\end{lstlisting}

\subsection{Function requestUserAccountHealthCalculation}

\noindent\begin{itemize}
\item TODO
\end{itemize}

\begin{lstlisting}[firstnumber=438]
    function requestUserAccountHealthCalculation(address tonWallet) external override view executor {
        tvm.rawReserve(msg.value, 2);
        address userAccount = _calculateUserAccountAddress(tonWallet);
        IUserAccountData(userAccount).checkUserAccountHealth{
            flag: MsgFlag.REMAINING_GAS
        }(tonWallet);
    }
\end{lstlisting}

\subsection{Function requestWithdraw}

\noindent\begin{itemize}
\item TODO
\end{itemize}

\begin{lstlisting}[firstnumber=166]
    function requestWithdraw(
        address tonWallet, 
        address userTip3Wallet, 
        uint32 marketId, 
        uint256 tokensToWithdraw
    ) external override view onlyValidUserAccount(tonWallet) {
        TvmBuilder tb;
        tb.store(tonWallet);
        tb.store(userTip3Wallet);
        tb.store(tokensToWithdraw);
        IMarketOperations(marketAddress).performOperationUserAccountManager{
            flag: MsgFlag.REMAINING_GAS
        }(OperationCodes.WITHDRAW_TOKENS, marketId, tb.toCell());
    }
\end{lstlisting}

\subsection{Function requestWithdrawInfo}

\noindent\begin{itemize}
\item TODO
\end{itemize}

\begin{lstlisting}[firstnumber=181]
    function requestWithdrawInfo(
        address tonWallet, 
        address userTip3Wallet,
        uint256 tokensToWithdraw, 
        uint32 marketId, 
        mapping(uint32 => fraction) updatedIndexes
    ) external override view onlyModule(OperationCodes.WITHDRAW_TOKENS) {
        address userAccount = _calculateUserAccountAddress(tonWallet);
        IUserAccountData(userAccount).requestWithdrawInfo{
            flag: MsgFlag.REMAINING_GAS
        }(userTip3Wallet, marketId, tokensToWithdraw, updatedIndexes);
    }
\end{lstlisting}

\subsection{Function returnAndSupply}

\noindent\begin{itemize}
\item TODO
\end{itemize}

\begin{lstlisting}[firstnumber=399]
    function returnAndSupply(
        address tonWallet,
        address tip3UserWallet,
        uint32 marketId,
        uint32 marketToLiquidate,
        uint256 tokensToReturn,
        uint256 tokensFromReserve
    ) external override view onlyValidUserAccountNoReserve(tonWallet) {
        if (tokensToReturn != 0) {
            uint128 tonsToUse = msg.value / 4;
            tvm.rawReserve(tonsToUse, 2);

            TvmBuilder tb;
            tb.store(tonWallet);
            tb.store(tokensFromReserve);

            IMarketOperations(marketAddress).performOperationUserAccountManager{
                value: msg.value - tonsToUse
            }(OperationCodes.SUPPLY_TOKENS, marketToLiquidate, tb.toCell());

            IMarketOperations(marketAddress).requestTokenPayout{
                flag: MsgFlag.REMAINING_GAS
            }(tonWallet, tip3UserWallet, marketId, tokensToReturn);
        } else {
            tvm.rawReserve(msg.value, 2);

            TvmBuilder tb;
            tb.store(tonWallet);
            tb.store(tokensFromReserve);

            IMarketOperations(marketAddress).performOperationUserAccountManager{
                flag: MsgFlag.REMAINING_GAS
            }(OperationCodes.SUPPLY_TOKENS, marketToLiquidate, tb.toCell());
        }
    }
\end{lstlisting}

\subsection{Function seizeTokens}

\noindent\begin{itemize}
\item TODO
\end{itemize}

\begin{lstlisting}[firstnumber=346]
    function seizeTokens(
        address tonWallet,
        address targetUser,
        address tip3UserWallet,
        uint32 marketId,
        uint32 marketToLiquidate,
        uint256 tokensToSeize, 
        uint256 tokensToReturn, 
        uint256 tokensFromReserve,
        BorrowInfo borrowInfo
    ) external override view onlyModule(OperationCodes.LIQUIDATE_TOKENS) {
        address userAccount = _calculateUserAccountAddress(targetUser);
        IUserAccountData(userAccount).liquidateVTokens{
            flag: MsgFlag.REMAINING_GAS
        }(tonWallet, tip3UserWallet, marketId, marketToLiquidate, tokensToSeize, tokensToReturn, tokensFromReserve, borrowInfo);
    }
\end{lstlisting}

\subsection{Function setMarketAddress}

\noindent\begin{itemize}
\item TODO
\end{itemize}

\begin{lstlisting}[firstnumber=493]
    function setMarketAddress(address _market) external override canChangeParams {
        tvm.accept();
        marketAddress = _market;
    }
\end{lstlisting}

\subsection{Function updateUserAccount}

\noindent\begin{itemize}
\item TODO
\end{itemize}

\begin{lstlisting}[firstnumber=506]
    function updateUserAccount(address tonWallet) external override {
        tvm.rawReserve(msg.value, 2);
        address userAccount = _calculateUserAccountAddress(tonWallet);
        optional(uint32, TvmCell) latestVersion = userAccountCodes.max();
        if (latestVersion.hasValue()) {
            TvmCell empty;
            (uint32 codeVersion, TvmCell code) = latestVersion.get();
            IUpgradableContract(userAccount).upgradeContractCode{
                flag: MsgFlag.REMAINING_GAS
            }(code, empty, codeVersion);
        } else {
            address(msg.sender).transfer({value: 0, flag: MsgFlag.REMAINING_GAS});
        }
    }
\end{lstlisting}

\subsection{Function updateUserAccountHealth}

\noindent\begin{itemize}
\item TODO
\end{itemize}

\begin{lstlisting}[firstnumber=459]
    function updateUserAccountHealth(
        address tonWallet, 
        address gasTo,
        fraction accountHealth, 
        mapping(uint32 => fraction) updatedIndexes,
        TvmCell dataToTransfer
    ) external override view onlyMarket {
        tvm.rawReserve(msg.value, 2);
        address userAccount = _calculateUserAccountAddress(tonWallet);
        IUserAccountData(userAccount).updateUserAccountHealth{
            flag: MsgFlag.REMAINING_GAS
        }(gasTo, accountHealth, updatedIndexes, dataToTransfer);
    }
\end{lstlisting}

\subsection{Function updateUserIndexes}

\noindent\begin{itemize}
\item TODO
\end{itemize}

\begin{lstlisting}[firstnumber=233]
    function updateUserIndexes(
        address tonWallet, 
        address userTip3Wallet, 
        uint256 tokensToBorrow, 
        uint32 marketId,
        mapping(uint32 => fraction) updatedIndexes
    ) external override view onlyModule(OperationCodes.BORROW_TOKENS) {
        address userAccount = _calculateUserAccountAddress(tonWallet);
        IUserAccountData(userAccount).borrowUpdateIndexes{
            flag: MsgFlag.REMAINING_GAS
        }(marketId, updatedIndexes, userTip3Wallet, tokensToBorrow);
    }
\end{lstlisting}

\subsection{Function upgradeContractCode}

\noindent\begin{itemize}
\item TODO
\end{itemize}

\begin{lstlisting}[firstnumber=56]
    function upgradeContractCode(TvmCell code, TvmCell updateParams, uint32 codeVersion) override external canUpgrade {
        tvm.accept();

        tvm.setcode(code);
        tvm.setCurrentCode(code);

        onCodeUpgrade(
            _owner,
            marketAddress,
            modules,
            existingModules,
            userAccountCodes,
            updateParams,
            codeVersion
        );
    }
\end{lstlisting}

\subsection{Function uploadUserAccountCode}

\noindent\begin{itemize}
\item TODO
\end{itemize}

\begin{lstlisting}[firstnumber=500]
    function uploadUserAccountCode(uint32 version, TvmCell code) external override canChangeParams {
        userAccountCodes[version] = code;
        
        address(msg.sender).transfer({flag: MsgFlag.REMAINING_GAS, value: 0});
    }
\end{lstlisting}

\subsection{Function withdrawExtraTons}

\noindent\begin{itemize}
\item TODO
\end{itemize}

\begin{lstlisting}[firstnumber=482]
    function withdrawExtraTons(address tonWallet) external onlyOwner {
        tvm.accept();
        address(tonWallet).transfer({value: 0, flag: 160});
    }
\end{lstlisting}

\subsection{Function writeBorrowInformation}

\noindent\begin{itemize}
\item TODO
\end{itemize}

\begin{lstlisting}[firstnumber=259]
    function writeBorrowInformation(
        address tonWallet, 
        address userTip3Wallet, 
        uint256 tokensToBorrow, 
        uint32 marketId, 
        fraction index
    ) external override view onlyModule(OperationCodes.BORROW_TOKENS) {
        address userAccount = _calculateUserAccountAddress(tonWallet);
        IUserAccountData(userAccount).writeBorrowInformation{
            flag: MsgFlag.REMAINING_GAS
        }(marketId, tokensToBorrow, userTip3Wallet, index);
    }
\end{lstlisting}

\subsection{Function writeRepayInformation}

\noindent\begin{itemize}
\item TODO
\end{itemize}

\begin{lstlisting}[firstnumber=300]
    function writeRepayInformation(
        address tonWallet, 
        address userTip3Wallet, 
        uint32 marketId,
        uint256 tokensToReturn, 
        BorrowInfo bi
    ) external override view onlyModule(OperationCodes.REPAY_TOKENS) {
        address userAccount = _calculateUserAccountAddress(tonWallet);
        IUserAccountData(userAccount).writeRepayInformation{
            flag: MsgFlag.REMAINING_GAS
        }(userTip3Wallet, marketId, tokensToReturn, bi);
    }
\end{lstlisting}

\subsection{Function writeSupplyInfo}

\noindent\begin{itemize}
\item TODO
\end{itemize}

\begin{lstlisting}[firstnumber=151]
    function writeSupplyInfo(
        address tonWallet,
        uint32 marketId, 
        uint256 tokensToSupply, 
        fraction index
    ) external override view onlyModule(OperationCodes.SUPPLY_TOKENS) {
        address userAccount = _calculateUserAccountAddress(tonWallet);
        IUserAccountData(userAccount).writeSupplyInfo{
            flag: MsgFlag.REMAINING_GAS
        }(marketId, tokensToSupply, index);
    }
\end{lstlisting}

\subsection{Function writeWithdrawInfo}

\noindent\begin{itemize}
\item TODO
\end{itemize}

\begin{lstlisting}[firstnumber=207]
    function writeWithdrawInfo(
        address tonWallet, 
        address userTip3Wallet, 
        uint32 marketId, 
        uint256 tokensToWithdraw, 
        uint256 tokensToSend
    ) external override view onlyModule(OperationCodes.WITHDRAW_TOKENS) {
        address userAccount = _calculateUserAccountAddress(tonWallet); 
        IUserAccountData(userAccount).writeWithdrawInfo{
            flag: MsgFlag.REMAINING_GAS
        }(userTip3Wallet, marketId, tokensToWithdraw, tokensToSend);
    }
\end{lstlisting}

\section{Internal Method Definitions}


\subsection{Function \_{}buildUserAccountData}

\noindent\begin{itemize}
\item TODO
\end{itemize}

\begin{lstlisting}[firstnumber=137]
    function _buildUserAccountData(address tonWallet) private view returns (TvmCell data) {
        return tvm.buildStateInit({
            contr: UserAccount,
            varInit: {
                owner: tonWallet
            },
            pubkey: 0,
            code: userAccountCodes[0]
        });
    }
\end{lstlisting}

\subsection{Function \_{}calculateUserAccountAddress}

\noindent\begin{itemize}
\item TODO
\end{itemize}

\begin{lstlisting}[firstnumber=130]
    function _calculateUserAccountAddress(address tonWallet) internal view returns(address) {
        return address(tvm.hash(_buildUserAccountData(tonWallet)));
    }
\end{lstlisting}

\subsection{Function onCodeUpgrade}

\noindent\begin{itemize}
\item TODO
\end{itemize}

\begin{lstlisting}[firstnumber=73]
    function onCodeUpgrade(
        address owner,
        address _marketAddress,
        mapping(uint8 => address) _modules,
        mapping(address => bool) _existingModules,
        mapping(uint32 => TvmCell) _userAccountCodes,
        TvmCell,
        uint32 _codeVersion
    ) private {
        tvm.accept();
        tvm.resetStorage();
        contractCodeVersion = _codeVersion;
        _owner = owner;
        marketAddress = _marketAddress;
        modules = _modules;
        existingModules = _existingModules;
        userAccountCodes = _userAccountCodes;
    }
\end{lstlisting}


\section{Contract WalletController}

In file {\tt WalletController.sol}


\subsection{Modifier onlyMarket}

\begin{lstlisting}[firstnumber=295]
    modifier onlyMarket() {
        require(msg.sender == marketAddress, WalletControllerErrorCodes.ERROR_MSG_SENDER_IS_NOT_MARKET);
        _;
    }
\end{lstlisting}

\noindent\begin{itemize}
  \item \unusedModifier{WalletController.onlyMarket}
\end{itemize}






%\section{Abstract Contract IRoles}

%In file {\tt IRoles.sol}




\section{Contract Platform}

In file {\tt Platform.sol}.

\subsection{Function initializeContract}

\begin{lstlisting}[firstnumber=18]
  function initializeContract(TvmCell contractCode, TvmCell params) private {
      tvm.accept();
      TvmBuilder builder;

      builder.store(root);
      builder.store(platformType);

      builder.store(platformCode); // ref 1
      builder.store(initialData);  // ref 2
      builder.store(params);       // ref 3

      tvm.setcode(contractCode);
      tvm.setCurrentCode(contractCode);

      onCodeUpgrade(builder.toCell());
  }
\end{lstlisting}

\noindent\begin{itemize}
  \item \issueMajor{{\tt tvm.accept} in a private function}{Private and internal functions should not have a {\tt tvm.accept}, especially without checks.}
\end{itemize}



\section{Module "RootTokenContract.sol"}


\subsection{Pragmas}


\noindent\begin{tabular}{|l|l|p{5cm}|}\hline
ton & -solidity $>$= 0.39.0 &\\\hline
AbiHeader &  expire &\\\hline
AbiHeader &  pubkey &\\\hline
\end{tabular}


\subsection{Imports}


\noindent\begin{tabular}{|l|l|p{5cm}|}\hline
./interfaces/IBurnableByRootTokenWallet.sol &\\\hline
./interfaces/IBurnableTokenRootContract.sol &\\\hline
./interfaces/IBurnableByRootTokenRootContract.sol &\\\hline
./interfaces/IExpectedWalletAddressCallback.sol &\\\hline
./interfaces/IBurnTokensCallback.sol &\\\hline
./interfaces/IRootTokenContract.sol &\\\hline
./interfaces/ITONTokenWallet.sol &\\\hline
./interfaces/IReceiveSurplusGas.sol &\\\hline
./interfaces/ISendSurplusGas.sol &\\\hline
./TONTokenWallet.sol &\\\hline
./interfaces/IPausable.sol &\\\hline
./interfaces/IPausedCallback.sol &\\\hline
./interfaces/ITransferOwner.sol &\\\hline
./libraries/RootTokenContractErrors.sol &\\\hline
./interfaces/IVersioned.sol &\\\hline
\end{tabular}


\subsection{Contract Definitions}

\begin{itemize}
\item RootTokenContract
\end{itemize}


\section{Module "TONTokenWallet.sol"}


\subsection{Pragmas}


\noindent\begin{tabular}{|l|l|p{5cm}|}\hline
ton & -solidity $>$= 0.39.0 &\\\hline
AbiHeader &  expire &\\\hline
AbiHeader &  pubkey &\\\hline
\end{tabular}


\subsection{Imports}


\noindent\begin{tabular}{|l|l|p{5cm}|}\hline
./interfaces/IDestroyable.sol &\\\hline
./interfaces/ITONTokenWallet.sol &\\\hline
./interfaces/IBurnableByOwnerTokenWallet.sol &\\\hline
./interfaces/IBurnableByRootTokenWallet.sol &\\\hline
./interfaces/IBurnableTokenRootContract.sol &\\\hline
./interfaces/ITokenWalletDeployedCallback.sol &\\\hline
./interfaces/ITokensReceivedCallback.sol &\\\hline
./interfaces/ITokensBouncedCallback.sol &\\\hline
./libraries/TONTokenWalletErrors.sol &\\\hline
./libraries/TONTokenWalletConstants.sol &\\\hline
./interfaces/IVersioned.sol &\\\hline
\end{tabular}


\subsection{Contract Definitions}

\begin{itemize}
\item TONTokenWallet
\end{itemize}




\bigskip

\section{Module "FloatingPointOperations.sol"}

\subsection{Struct fraction}

\begin{lstlisting}[firstnumber=3]
  struct fraction {
      uint256 nom;
      uint256 denom;
  }
\end{lstlisting}

\issueMinor{Unintuitive struct field name}{The name of the field {\tt nom} should be {\tt num} for ``numerator''.}

\bigskip

\section{Library FPO}

In file {\tt FloatingPointOperations.sol}

\subsection{Function eq}

\begin{lstlisting}[firstnumber=53]
    function eq(fraction a, fraction b) internal pure returns(bool) {
        return ((a.nom == b.nom) && (a.denom == b.denom));
    }
\end{lstlisting}

\issueMajor{Math error in {\tt FPO.eq}}{Comparing numerators and denominators when testing if fractions are equal is incorrect. $eq(\frac{a}{b}, \frac{a \times 2}{b \times 2})$ will return {\tt false} while it should return {\tt true}. The fractions need to be normalized before checking if they are equal.}

\bigskip

\subsection{Function simplify}

\begin{lstlisting}[firstnumber=69]
  function simplify(fraction a) internal pure returns(fraction) {
      // loosing ??? % of presicion at most
      if (a.nom / a.denom > 100e9) {
          return fraction(a.nom / a.denom, 1);
      } else {
          // using bitshift for simultaneos division
          // leaving up to 64 bits of information if nom & denom > 2^64
          if ( (a.nom >= bits224) && (a.denom >= bits224) ) {
              return fraction(a.nom / bits160, a.denom / bits160);
          }

          if ( (a.nom >= bits192) && (a.denom >= bits192) ) {
              return fraction(a.nom / bits128, a.denom / bits128);
          }

          if ( (a.nom >= bits160) && (a.denom >= bits160) ) {
              return fraction(a.nom / bits96, a.denom / bits96);
          }

          if ( (a.nom >= bits128) && (a.denom >= bits128) ) {
              return fraction(a.nom / bits64, a.denom / bits64);
          }

          if ( (a.nom >= bits96) && (a.denom >= bits96) ) {
              return fraction(a.nom / bits32, a.denom / bits32);
          }

          return a;
      }
  }
\end{lstlisting}

\issueMajor{Math issue in {\tt FPO.simplify}}{Dividing the numerator and denominator by their greatest common divisor might make it unnecessary to do the bitshift and avoid losing precision.}


\section{Library TvmCellOperations}

In file {\tt TvmCellOperations.sol}

\issueMinor{Unused functions}{All the functions in the file are unused.}

