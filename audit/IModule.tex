
\bigskip

\section{Library Utilities}

In file {\tt IModule.sol}

\subsection{Function calculateSupplyBorrow}

\begin{lstlisting}[firstnumber=90]
    function calculateSupplyBorrow(
        mapping(uint32 => uint256) supplyInfo,
        mapping(uint32 => BorrowInfo) borrowInfo,
        mapping(uint32 => MarketInfo) marketInfo,
        mapping(address => fraction) tokenPrices
    ) internal returns (fraction) {
        fraction accountHealth = fraction(0, 0);
        fraction tmp;
        fraction nom = fraction(0, 1);
        fraction denom = fraction(0, 1);

        // Supply:
        // 1. Calculate real token amount: vToken*exchangeRate
        // 2. Calculate real token amount in USD: realTokens/tokenPrice
        // 3. Multiply by collateral factor: usdValue*collateralFactor
        for ((uint32 marketId, uint256 supplied): supplyInfo) {
            tmp = supplied.numFMul(marketInfo[marketId].exchangeRate);
            tmp = tmp.fDiv(tokenPrices[marketInfo[marketId].token]);
            tmp = tmp.fMul(marketInfo[marketId].collateralFactor);
            nom = nom.fAdd(tmp);
            nom = nom.simplify();
        }

        // Borrow:
        // 1. Recalculate borrow amount according to new index
        // 2. Calculate borrow value in USD
        // NOTE: no conversion from vToken to real tokens required, as value is stored in real tokens
        for ((uint32 marketId, BorrowInfo _bi): borrowInfo) {
            if (_bi.tokensBorrowed != 0) {
                if (!_bi.index.eq(marketInfo[marketId].index)) {
                    tmp = borrowInfo[marketId].tokensBorrowed.numFMul(marketInfo[marketId].index);
                    tmp = tmp.fDiv(borrowInfo[marketId].index);
                } else {
                    tmp = borrowInfo[marketId].tokensBorrowed.toF();
                }
                tmp = tmp.fDiv(tokenPrices[marketInfo[marketId].token]);
                tmp = tmp.simplify();
                denom = denom.fAdd(tmp);
                denom = denom.simplify();
            }
        }

        accountHealth = nom.fDiv(denom);

        return accountHealth;
    }
\end{lstlisting}

\issueMinor{Unintuitive function name}{The function is called ``calculateSupplyBorrow'' but it calculates a user's account health. It should be named accordingly, e.g. ``calculateAccountHealth''.}
