


\bigskip

\section{Module "FloatingPointOperations.sol"}

\subsection{Struct fraction}

\begin{lstlisting}[firstnumber=3]
  struct fraction {
      uint256 nom;
      uint256 denom;
  }
\end{lstlisting}

\issueMinor{Unintuitive struct field name}{The name of the field {\tt nom} should be {\tt num} for ``numerator''.}

\bigskip

\section{Library FPO}

In file {\tt FloatingPointOperations.sol}

\subsection{Function eq}

\begin{lstlisting}[firstnumber=53]
    function eq(fraction a, fraction b) internal pure returns(bool) {
        return ((a.nom == b.nom) && (a.denom == b.denom));
    }
\end{lstlisting}

\issueMajor{Math error in {\tt FPO.eq}}{Comparing numerators and denominators when testing if fractions are equal is incorrect. $eq(\frac{a}{b}, \frac{a \times 2}{b \times 2})$ will return {\tt false} while it should return {\tt true}. The fractions need to be normalized before checking if they are equal.}

\bigskip

\subsection{Function simplify}

\begin{lstlisting}[firstnumber=69]
  function simplify(fraction a) internal pure returns(fraction) {
      // loosing ??? % of presicion at most
      if (a.nom / a.denom > 100e9) {
          return fraction(a.nom / a.denom, 1);
      } else {
          // using bitshift for simultaneos division
          // leaving up to 64 bits of information if nom & denom > 2^64
          if ( (a.nom >= bits224) && (a.denom >= bits224) ) {
              return fraction(a.nom / bits160, a.denom / bits160);
          }

          if ( (a.nom >= bits192) && (a.denom >= bits192) ) {
              return fraction(a.nom / bits128, a.denom / bits128);
          }

          if ( (a.nom >= bits160) && (a.denom >= bits160) ) {
              return fraction(a.nom / bits96, a.denom / bits96);
          }

          if ( (a.nom >= bits128) && (a.denom >= bits128) ) {
              return fraction(a.nom / bits64, a.denom / bits64);
          }

          if ( (a.nom >= bits96) && (a.denom >= bits96) ) {
              return fraction(a.nom / bits32, a.denom / bits32);
          }

          return a;
      }
  }
\end{lstlisting}

\issueMajor{Math issue in {\tt FPO.simplify}}{Dividing the numerator and denominator by their greatest common divisor might make it unnecessary to do the bitshift and avoid losing precision.}
