
\chapter{Library MarketOperations}

\minitoc

\section{Overview}


In file {\tt MarketOperations.sol}

\section{Internal Method Definitions}


\subsection{Function calculateBorrowInterestRate}

\noindent\begin{itemize}
\item TODO
\end{itemize}

\begin{lstlisting}[firstnumber=13]
    function calculateBorrowInterestRate(fraction baseRate, uint256 realTokenBalance, uint256 totalBorrowed, fraction utilizationMultiplier) internal returns (fraction) {
        fraction bir;

        fraction utilizationRate = fraction(totalBorrowed, totalBorrowed + realTokenBalance);

        bir = utilizationRate.fMul(utilizationMultiplier);
        bir = bir.fAdd(baseRate);

        return bir;
    }
\end{lstlisting}

\subsection{Function calculateExchangeRate}

\noindent\begin{itemize}
\item TODO
\end{itemize}

\begin{lstlisting}[firstnumber=24]
    function calculateExchangeRate(uint256 currentPoolBalance, uint256 totalBorrowed, uint256 totalReserve, uint256 vTokenSupply) internal pure returns(fraction) {
        return fraction(currentPoolBalance + totalBorrowed - totalReserve, vTokenSupply);
    }
\end{lstlisting}

\subsection{Function calculateNewIndex}

\noindent\begin{itemize}
\item TODO
\end{itemize}

\begin{lstlisting}[firstnumber=37]
    function calculateNewIndex(fraction index, fraction bir, uint256 dt) internal returns (fraction) {
        fraction index_;
        index_ = bir.fNumMul(dt);
        index_ = index_.fNumAdd(1);
        index_ = index_.fAdd(index);
        return index_;
    }
\end{lstlisting}

\subsection{Function calculateReserves}

\noindent\begin{itemize}
\item TODO
\end{itemize}

\begin{lstlisting}[firstnumber=52]
    function calculateReserves(uint256 reserveOld, uint256 totalBorrowedOld, fraction bir, fraction reserveFactor, uint256 dt) internal returns (uint256) {
        fraction res = bir;
        res = res.fNumMul(dt);
        res = res.fMul(reserveFactor);
        res = res.fNumMul(totalBorrowedOld);
        res = res.fNumAdd(reserveOld);
        return res.toNum();
    }
\end{lstlisting}

\subsection{Function calculateTotalBorrowed}

\noindent\begin{itemize}
\item TODO
\end{itemize}

\begin{lstlisting}[firstnumber=45]
    function calculateTotalBorrowed(uint256 totalBorrowed, fraction oldIndex, fraction newIndex) internal returns (uint256) {
        fraction tb_;
        tb_ = totalBorrowed.numFDiv(oldIndex);
        tb_ = tb_.fMul(newIndex);
        return tb_.toNum();
    }
\end{lstlisting}

\subsection{Function calculateTotalReserves}

\noindent\begin{itemize}
\item TODO
\end{itemize}

\begin{lstlisting}[firstnumber=28]
    function calculateTotalReserves(uint256 totalReserve, uint256 totalBorrowed, fraction r, fraction reserveFactor, uint256 t) internal returns (fraction) {
        fraction tr;
        tr = r.fNumMul(t);
        tr = tr.fMul(reserveFactor);
        tr = tr.fNumMul(totalBorrowed);
        tr = tr.fNumAdd(totalReserve);
        return tr;
    }
\end{lstlisting}

\subsection{Function calculateU}

\noindent\begin{itemize}
\item TODO
\end{itemize}

\begin{lstlisting}[firstnumber=9]
    function calculateU(uint256 totalBorrowed, uint256 realTokens) internal pure returns (fraction) {
        return fraction(totalBorrowed, totalBorrowed + realTokens);
    }
\end{lstlisting}
\paragraph{Some functions inherited by using}
