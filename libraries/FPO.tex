
\chapter{Library FPO}

\minitoc

\section{Overview}


In file {\tt FloatingPointOperations.sol}

\section{Constant Definitions}


\begin{lstlisting}[firstnumber=9]
    uint256 constant bits224 = 2**224;
\end{lstlisting}

\begin{lstlisting}[firstnumber=10]
    uint256 constant bits192 = 2**192;
\end{lstlisting}

\begin{lstlisting}[firstnumber=11]
    uint256 constant bits160 = 2**160;
\end{lstlisting}

\begin{lstlisting}[firstnumber=12]
    uint256 constant bits128 = 2**128;
\end{lstlisting}

\begin{lstlisting}[firstnumber=13]
    uint256 constant bits96 = 2**96;
\end{lstlisting}

\begin{lstlisting}[firstnumber=14]
    uint256 constant bits64 = 2**64;
\end{lstlisting}

\begin{lstlisting}[firstnumber=15]
    uint256 constant bits32 = 2**32;
\end{lstlisting}

\section{Internal Method Definitions}


\subsection{Function eq}

\noindent\begin{itemize}
\item TODO
\end{itemize}

\begin{lstlisting}[firstnumber=53]
    function eq(fraction a, fraction b) internal pure returns(bool) {
        return ((a.nom == b.nom) && (a.denom == b.denom));
    }
\end{lstlisting}

\subsection{Function fAdd}

\noindent\begin{itemize}
\item TODO
\end{itemize}

\begin{lstlisting}[firstnumber=33]
    function fAdd(fraction a, fraction b) internal pure returns (fraction) {
        return fraction (a.nom * b.denom + b.nom * a.denom, a.denom * b.denom);
    }
\end{lstlisting}

\subsection{Function fDiv}

\noindent\begin{itemize}
\item TODO
\end{itemize}

\begin{lstlisting}[firstnumber=29]
    function fDiv(fraction a, fraction b) internal pure returns(fraction) {
        return fraction(a.nom * b.denom, a.denom * b.nom);
    }
\end{lstlisting}

\subsection{Function fMul}

\noindent\begin{itemize}
\item TODO
\end{itemize}

\begin{lstlisting}[firstnumber=17]
    function fMul(fraction a, fraction b) internal pure returns (fraction) {
        return fraction(a.nom*b.nom, a.denom*b.denom);
    }
\end{lstlisting}

\subsection{Function fNumAdd}

\noindent\begin{itemize}
\item TODO
\end{itemize}

\begin{lstlisting}[firstnumber=37]
    function fNumAdd(fraction a, uint256 b) internal pure returns (fraction) {
        return fraction (a.nom + b*a.denom, a.denom);
    }
\end{lstlisting}

\subsection{Function fNumDiv}

\noindent\begin{itemize}
\item TODO
\end{itemize}

\begin{lstlisting}[firstnumber=25]
    function fNumDiv(fraction a, uint256 b) internal pure returns (fraction) {
        return fraction(a.nom, a.denom * b);
    }
\end{lstlisting}

\subsection{Function fNumMul}

\noindent\begin{itemize}
\item TODO
\end{itemize}

\begin{lstlisting}[firstnumber=21]
    function fNumMul(fraction a, uint256 b) internal pure returns (fraction) {
        return fraction(a.nom * b, a.denom);
    }
\end{lstlisting}

\subsection{Function fSub}

\noindent\begin{itemize}
\item TODO
\end{itemize}

\begin{lstlisting}[firstnumber=41]
    function fSub(fraction a, fraction b) internal pure returns (fraction) {
        return fraction(a.nom * b.denom - b.nom * a.denom, a.denom * b.denom);
    }
\end{lstlisting}

\subsection{Function getMin}

\noindent\begin{itemize}
\item TODO
\end{itemize}

\begin{lstlisting}[firstnumber=57]
    function getMin(fraction a, fraction b) internal pure returns(fraction) {
        if (a.nom * b.denom < b.nom * a.denom) {
            return a;
        } else {
            return b;
        }
    }
\end{lstlisting}

\subsection{Function isLarger}

\noindent\begin{itemize}
\item TODO
\end{itemize}

\begin{lstlisting}[firstnumber=45]
    function isLarger(fraction a, fraction b) internal pure returns (bool) {
        return a.nom * b.denom > b.nom * a.denom;
    }
\end{lstlisting}

\subsection{Function lessThan}

\noindent\begin{itemize}
\item TODO
\end{itemize}

\begin{lstlisting}[firstnumber=65]
    function lessThan(fraction a, fraction b) internal pure returns(bool) {
        return a.nom * b.denom < b.nom * a.denom;
    }
\end{lstlisting}

\subsection{Function simplify}

\noindent\begin{itemize}
\item TODO
\end{itemize}

\begin{lstlisting}[firstnumber=69]
    function simplify(fraction a) internal pure returns(fraction) {
        // loosing ??? of presicion at most
        if (a.nom / a.denom > 100e9) {
            return fraction(a.nom / a.denom, 1);
        } else {
            // using bitshift for simultaneos division
            // leaving up to 64 bits of information if nom & denom > 2^64
            if ( (a.nom >= bits224) && (a.denom >= bits224) ) {
                return fraction(a.nom / bits160, a.denom / bits160);
            }

            if ( (a.nom >= bits192) && (a.denom >= bits192) ) {
                return fraction(a.nom / bits128, a.denom / bits128);
            }

            if ( (a.nom >= bits160) && (a.denom >= bits160) ) {
                return fraction(a.nom / bits96, a.denom / bits96);
            }

            if ( (a.nom >= bits128) && (a.denom >= bits128) ) {
                return fraction(a.nom / bits64, a.denom / bits64);
            }

            if ( (a.nom >= bits96) && (a.denom >= bits96) ) {
                return fraction(a.nom / bits32, a.denom / bits32);
            }

            return a;
        }
    }
\end{lstlisting}

\subsection{Function toNum}

\noindent\begin{itemize}
\item TODO
\end{itemize}

\begin{lstlisting}[firstnumber=49]
    function toNum(fraction a) internal pure returns(uint256) {
        return a.nom / a.denom;
    }
\end{lstlisting}
